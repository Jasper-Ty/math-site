\documentclass{article}

% See https://github.com/Jasper-Ty/dotfiles
\usepackage[garamond]{jaspercommon}

\newcommand{\fps}[2]{#1 \left\lBrack {#2} \right\rBrack}
\newcommand{\pring}[2]{#1 \left[{#2} \right]}
\newcommand{\bigcomma}{\ensuremath{\mathop{\scalebox{3}{\:,\:}}\limits}}

\DeclareMathOperator{\EGF}{EGF}
\DeclareMathOperator{\Sym}{Sym}

\title{Trees and the Composition of Generating Functions}
\author{Jasper Ty}
\date{}

\titleauthorhead

\begin{document}

\maketitle

\section*{What is this?}

These are notes based on my study of Chapter 5 in Richard P. Stanley's ``Enumerative Combinatorics''.

\tableofcontents

\section{The exponential formula}

We recall the mechanics of \textit{formal power series composition}. 

\begin{definition}
    Let $f,g \in \fps{\KK}{x}$ such that $[x^0]g = 0$.
    The \textit{composition} $f \circ g$ is defined to be
    \begin{align*}
        (f \circ g)(x) &\coloneq \sum_{n\geq0} f_ng^n \\
                       &= f_0g^0 + f_1g^1 + \cdots,
    \end{align*}
    where we define $f_n \coloneq [x^n]f$ for all $n \geq 0$.
\end{definition}

The condition that $[x^0]g = 0$ ensures that $f \circ g$ is well-defined, as this means that, for any $n \geq 0$, $[x^n](f \circ g)$ is determined by only \textit{finitely many} of the terms $f_kg^k$, namely those where $k \leq n$.

Now, let $f$ and $g$ be \textit{exponential} generating functions, whose definition we recall.

\begin{definition}
    Let $a = \{a_i\}_{i\geq0}$ be a sequence of numbers (say, integers).
    We define its \textit{exponential generating function} $\EGF_a$ to be the formal power series
    \[
        \EGF_a(x) \coloneq \sum_{n \geq 0}a_n \frac{x^n}{n!}.
    \]
\end{definition}

Is there any special structure, then, to a \textit{composition of exponential generating functions} $\EGF_g \circ \EGF_f$? We first answer this ``directly''.

\begin{theorem}[The compositional formula, take zero]
    Let $\{f_i\}_{i\geq0}$ be a sequence of numbers such that $f_0 = 0$, and let $\{g_i\}_{i\geq0}$ be a sequence of numbers such that $g_0 = 1$.
    Then
    \begin{align*}
        (\EGF_g \circ \EGF_f)(x)
        &=
        \sum_{m\geq1}
        \left[
            \sum_{n\geq0}
            \frac{g_n}{n!}
            \left[
                    \sum_{\substack{(m_1,\ldots,m_n)\in\PP^n\\m_1+\cdots+m_n=m}}
                    \frac{m!}{m_1!\cdots m_n!}f_{m_1}\cdots f_{m_n}
            \right]
        \right]
        \frac{x^m}{m!}.
    \end{align*}
\end{theorem}

\begin{proof}
    This is just a long calculation.
    \begin{align*}
        (\EGF_g \circ \EGF_f)(x) &= 
        \sum_{n\geq0}
        g_n \frac{\big[\EGF_f(x)\big]^n}{n!} \\
                                 &= 
                                 \sum_{n\geq0} 
                                 \frac{g_n}{n!}
                                 \underbrace{
                                     \left[
                                         \sum_{m\geq1}f_m\frac{x^m}{m!}
                                     \right]^n
                                 }_{\text{expand product of sums}} \\
                                 &= 
                                 \sum_{n\geq0}
                                 \frac{g_n}{n!}
                                 \underbrace{
                                     \left[
                                         \sum_{m_1,\ldots,m_n\geq1}
                                         f_{m_1}\frac{x^{m_1}}{m_1!} \cdots f_{m_n}\frac{x^{m_n}}{m_n!}
                                     \right]
                                 }_{\text{group terms by their index sum }m =\sum_i{m_i}} \\
                                 &= 
                                 \sum_{n\geq0}
                                 \frac{g_n}{n!}
                                 \left[
                                     \sum_{m\geq1}
                                     \sum_{
                                         \substack{
                                             m_1,\ldots,m_n\geq1 \\
                                             m_1+\cdots+m_n=m
                                         }
                                     }
                                     \underbrace{
                                        f_{m_1}\frac{x^{m_1}}{m_1!} \cdots f_{m_n}\frac{x^{m_n}}{m_n!}
                                    }_{\text{organize terms}}
                                 \right] \\
                                 &= 
                                 \sum_{n\geq0}
                                 \frac{g_n}{n!}
                                 \left[
                                     \sum_{m\geq1}
                                     \sum_{
                                         \substack{
                                             m_1,\ldots,m_n\geq1 \\
                                             m_1+\cdots+m_n=m
                                         }
                                     }
                                     \frac
                                     {x^{m_1}\cdots x^{m_n}}
                                     {m_1!\cdots m_n!}
                                     f_{m_1}\cdots f_{m_n}
                                 \right] \\
                                 &= 
                                 \sum_{n\geq0}
                                 \frac{g_n}{n!}
                                 \left[
                                     \sum_{m\geq1}
                                     \sum_{
                                         \substack{
                                             m_1,\ldots,m_n\geq1 \\
                                             m_1+\cdots+m_n=m
                                         }
                                     }
                                     \frac
                                     {\overbrace{x^{m_1+\cdots+m_n}}^{\text{equals }x^m}}
                                     {m_1!\cdots m_n!}
                                     f_{m_1}\cdots f_{m_n}
                                 \right] \\
                                 &= 
                                 \sum_{n\geq0}
                                 \frac{g_n}{n!}
                                 \left[
                                     \sum_{m\geq1}
                                     x^m
                                     \left[
                                         \sum_{
                                             \substack{
                                                 m_1,\ldots,m_n\geq1 \\
                                                 m_1+\cdots+m_n=m
                                             }
                                         }
                                         \frac
                                         {1}
                                         {m_1!\cdots m_n!}
                                         f_{m_1}\cdots f_{m_n}
                                     \right]
                                 \right] \\
                                 &= 
                                 \sum_{n\geq0}
                                 \frac{g_n}{n!}
                                 \left[
                                     \sum_{m\geq1}
                                     \frac{x^m}{m!}
                                     \left[
                                         \sum_{
                                             \substack{
                                                 m_1,\ldots,m_n\geq1 \\
                                                 m_1+\cdots+m_n=m
                                             }
                                         }
                                         \frac
                                         {m!}
                                         {m_1!\cdots m_n!}
                                         f_{m_1}\cdots f_{m_n}
                                     \right]
                                 \right] \\
                                 &= 
                                 \underbrace{
                                     \sum_{n\geq0}
                                     \frac{g_n}{n!}
                                     \sum_{m\geq1}
                                     \frac{x_m}{m!} 
                                 }_{\text{interchange sums}}
                                 \sum_{
                                     \substack{
                                         m_1,\ldots,m_n\geq1 \\
                                         m_1+\cdots+m_n=m
                                     }
                                 }
                                 \frac
                                 {m!}
                                 {m_1!\cdots m_n!}
                                 f_{m_1}\cdots f_{m_n} \\
                                 &= 
                                 \sum_{m\geq1}
                                 \frac{x_m}{m!} 
                                 \sum_{n\geq0}
                                 \frac{g_n}{n!}
                                 \sum_{
                                     \substack{
                                         m_1,\ldots,m_n\geq1 \\
                                         m_1+\cdots+m_n=m
                                     }
                                 }
                                 \frac
                                 {m!}
                                 {m_1!\cdots m_n!}
                                 f_{m_1}\cdots f_{m_n}.
    \end{align*}
\end{proof}

If, given $\{f_i\}_{i\geq0}$ and $\{g_i\}_{i\geq0}$, we defined the sequence $\{h_i\}_{i\geq0}$ to be
\begin{align*}
    h_n 
    &\coloneq 
    \sum_{k\geq0}
    \frac{g_k}{k!}
    \left[
            \sum_{\substack{(n_1,\ldots,n_k)\in\PP^k\\n_1+\cdots+n_k=n}}
            \frac{n!}{n_1!\cdots n_k!}f_{n_1}\cdots f_{n_k}
    \right] \\
    h_0
    &\coloneq
    1,
\end{align*}
then the theorem can be restated as
\[
    \EGF_g \circ \EGF_f = \EGF_h.
\]
But what \textit{is} $h$?
Can we \textit{breathe some combinatorial life into it}?
The answer is yes.

First, you will have noticed that our manipulations have introduced a \textit{multinomial coefficient}, so we can rewrite
\begin{align*}
    h_n 
    &=
    \sum_{k\geq0}
    \frac{g_k}{k!}
    \sum_{\substack{(n_1,\ldots,n_k)\in\PP^k\\n_1+\cdots+n_k=n}}
    \overbrace{
        \frac{n!}{n_1!\cdots n_k!}
    }^{=\binom{i}{i_1,\ldots,i_n}}
    f_{n_1}\cdots f_{n_k}
    \\
    &=
    \sum_{k\geq0}
    \frac{g_k}{k!}
    \sum_{\substack{(n_1,\ldots,n_k)\in\PP^k\\n_1+\cdots+n_k=n}}
    \binom{n}{n_1,\ldots,n_k}
    f_{n_1}\cdots f_{n_k}
    \\
    &=
    \sum_{k\geq0}
    g_k
    \sum_{\substack{(n_1,\ldots,n_k)\in\PP^k\\n_1+\cdots+n_k=n}}
    \left[
        \frac{1}{k!}
        \binom{n}{n_1,\ldots,n_k}
    \right]
    f_{n_1}\cdots f_{n_k}.
\end{align*}

We recall $\displaystyle\binom{n}{n_1,\ldots,n_k}$'s most direct combinatorial interpretation: ways of putting $n$ balls into $k$ \textit{labeled} boxes, such that the first box has $n_1$ balls, the second has $n_2$ balls, and so on.
Equivalently, we are considering \textit{ordered partitions of a set of size $n$ into $k$ parts}.

Dividing this quantity by $k!$ \textit{forgets} the labeling on the boxes.
In terms of partitions, this means we are considering \textit{unordered partitions of a set of size $n$} now.

\begin{definition}
    Let $X$ be a finite set.
    Denote by $\mathbf{Par}(X)$ the set of all \textit{ordered partitions of $X$ into $k$ parts}.
    Denote by $\mathbf{Par}^{\Sym}(X)$ the set of all \textit{unordered partitions of $X$ into $k$ parts}.
\end{definition}

Then
\begin{align*}
    h_n 
    &=
    \sum_{k\geq0}
    g_k
    \sum_{\substack{(n_1,\ldots,n_k)\in\PP^k\\n_1+\cdots+n_k=n}}
    \left[
        \frac{1}{k!}
        \binom{n}{n_1,\ldots,n_k}
    \right]
    f_{n_1}\cdots f_{n_k}
    \\
    &=
    \sum_{k\geq0}
    \frac{g_k}{k!}
    \sum_{\substack{(n_1,\ldots,n_k)\in\PP^k\\n_1+\cdots+n_k=n}}
    \left(
        \substack{
            \#\text{ of ordered partitions of }[n]\\
            \text{with part sizes }\{n_1,\ldots,n_k\}
        }
    \right)
    f_{n_1}\cdots f_{n_k}
    \\
    &=
    \sum_{k\geq0}
    \frac{g_k}{k!}
    \sum_{\substack{(n_1,\ldots,n_k)\in\PP^k\\n_1+\cdots+n_k=n}}
    \left(
        \sum_{\substack{
                \pi = (\pi_1,\ldots,\pi_k) \in \mathbf{P}(n)\\
            |\pi_i| = n_i
        }}
        1
    \right)
    f_{n_1}\cdots f_{n_k}
    \\
    &=
    \sum_{k\geq0}
    \frac{g_k}{k!}
    \sum_{\substack{(n_1,\ldots,n_k)\in\PP^k\\n_1+\cdots+n_k=n}}
    \left(
        \sum_{\substack{
                \pi = (\pi_1,\ldots,\pi_k) \in \mathbf{P}(n) \\
            \#\pi_i = n_i
        }}
        f_{n_1}\cdots f_{n_k}
    \right)
    \\
    &=
    \sum_{k\geq0}
    g_k
    \frac{1}{k!}
    \sum_{\substack{(n_1,\ldots,n_k)\in\PP^k\\n_1+\cdots+n_k=n}}
    \left(
        \sum_{\substack{
                \pi = (\pi_1,\ldots,\pi_k) \in \mathbf{P}(n) \\
            \#\pi_i = n_i
        }}
        f_{\#\pi_1}\cdots f_{\#\pi_k}
    \right)
    \\
    &=
    \sum_{k\geq0}
    g_k
    \frac{1}{k!}
    \sum_{\pi = (\pi_1,\ldots,\pi_k) \in \mathbf{P}(n)}
    f_{\#\pi_1}\cdots f_{\#\pi_k}
    \\
    &=
    \sum_{k\geq0}
    g_k
    \sum_{\pi = (\pi_1,\ldots,\pi_k) \in \mathbf{S}(n)}
    f_{\#\pi_1}\cdots f_{\#\pi_k}
    \\
    &=
    \sum_{\pi = (\pi_1,\ldots,\pi_k) \in \mathbf{S}(n)}
    f_{\#\pi_1}\cdots f_{\#\pi_k} g_k
\end{align*}


Now, we make a psychological shift--- we consider our sequences not as sequences, but as \textit{weight functions}, which gives us a rule.

Namely, we will consider counting the number of \textit{structures}


\begin{theorem}[The compositional formula]
    Let $f(n)$ and $g(n)$ be two weight functions.
    Define
    \[
        h(n)
        \coloneq
        \sum_{\pi = (\pi_1,\ldots,\pi_k) \in \mathbf{S}(n)}
        f(\#\pi_1)\cdots f(\#\pi_k) g(k)
    \]

    Then
    \[
        \EGF_g \circ \EGF_f = \EGF_h.
    \]
\end{theorem}

The compositional formula reflects putting a $f$-structure on a partition of $X$, and a $g$-structure on the partitions.

\begin{theorem}[The exponential formula]
    Let $f: \PP \to K$.

    Then if we define
    \[
        h(n)
        \coloneq
        \sum_{\pi = (\pi_1,\ldots,\pi_k) \in \mathbf{S}(n)}
        f(\#\pi_1)\cdots f(\#\pi_k) g(k)
    \]
    We have that
    \[
        \EGF_h = \exp \EGF_f 
    \]
\end{theorem}

\end{document}
