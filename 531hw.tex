\documentclass{article}

\usepackage[a5paper, margin=0.7in]{geometry}
\usepackage[garamond]{jaspercommon}

\DeclareMathOperator{\sign}{sign}

\makeatletter
\fancyhead[L]{\@title}
\fancyhead[R]{\@author}
\makeatother

\title{MATH-531 Homework Solutions}

\newcommand{\pts}[1]{\fbox{#1}}

\author{Jasper Ty}
\date{}

\begin{document}

\maketitle

\section{Before we start...}

\subsection{Binomial coefficients and elementary counting}

\begin{exercise}\pts{3} Let $n \in \NN$. Prove that 
    \[
        \sum_{k=0}^n \binom{-2}{k} = (-1)^n \left\lfloor\frac{n+2}{2}\right\rfloor
    \]
\end{exercise}

\section{Generating functions}
\section{Integer partitions and \texorpdfstring{$q$}{q}-binomial coefficients}
\section{Permutations}
\section{Alternating sums, signed counting and determinants}

\begin{exercise} ($3$ points) \\
    Prove a generalization of Lemma $6.1.4$ in which $f$ is only required to be a bijection, not an involution, but the assumption ``$\sign I = 0$ for all $I \in \mathcal{X}$ satisfying $f(I) = I$'' is replaced by the stronger assumption  ``$\sign I = 0$ for all $I \in \mathcal{X}$ and all \textbf{odd} $k \in \NN$ satisfying $f^k(I) = I$''
\end{exercise}

Since $\mathcal{X}$ is finite, there exists a smallest $k$.

\begin{exercise}
    Recall the concepts of Dyck words and Dyck paths defined in Example 2 in Section 3.1.
    Let $n \in \N$.
    If $w \in {0,1}^{2n}$ is a $2n$-tuple, and if $k \in \{0,1,\ldots,2n\}$, then we define \textit{$k$-height} $h_k(w)$ of $w$ to be the number 
    \begin{align*}
        (\text{\# of $1$'s }&\text{among the first $k$ entries of $w$}) \\
        &- (\text{\# of $0$'s among the first $k$ entries of $w$}) 
    \end{align*}
    If $w$ is a Dyck word, then this $k$-height $h_k(w)$ is a nonnegative integer.
    [For example, if $n=4$ and $w=(1,0,0,1)$ then $h_3(w)$ is a nonnegative integer.
\end{exercise}

\section{Symmetric functions}



\end{document}
