% !TEX TS-program = xelatex

\documentclass{article}

% See https://github.com/Jasper-Ty/dotfiles
\usepackage[garamond]{jaspercommon}
\usepackage{bm}

% Real part
\renewcommand*\Re{\operatorname{Re}}

\usepackage{scrextend}
\usepackage{tikz}
\usepackage{tikz-cd}

\DeclarePairedDelimiter\bra{\langle}{\rvert}
\DeclarePairedDelimiter\ket{\lVert}{\rangle}
\DeclarePairedDelimiterX\braket[2]{\langle}{\rangle}{#1\,\delimsize\vert\,\mathopen{}#2}

% Wildcard
\newcommand{\innerproduct}[2]{\ensuremath{\left\langle #1 , #2 \right\rangle}}
\newcommand{\ip}[1]{\ensuremath{\left\langle{#1}\right\rangle}}

% The set of positive semidefinite matrices
\DeclareMathOperator{\PSD}{PSD}

% The set of matrices
\DeclareMathOperator{\Mat}{Mat}

% The set of matrices
\newcommand*\Pol{\ensuremath{\mathfrak{pol}}}

% Set of all Hermitian matrices
\newcommand*\Herm{\ensuremath{\text{Herm}}}

%
\DeclareMathOperator{\End}{End}

% Trace
\DeclareMathOperator{\tr}{tr}

% Torus
\newcommand*\TT{\ensuremath{\mathbb{T}}}

\title{Positive Polynomials and Sums}
\author{Jasper Ty}
\date{}

\titleauthorhead

\begin{document}

\maketitle

\section*{What is this?}


These are notes I am taking for the class ``Positive Polynomials and Sums'' at Drexel University, taught by Hugo Woerdeman.

\tableofcontents

\section{Cones of Hermitian matrices and trigonometric polynomials}

\subsection{Cones}

\begin{definition}
    Let $\scrH$ be a real Hilbert space. 

    A \defstyle{cone} $\scrC \subseteq \scrH$ is a nonempty subset of $\scrH$ such that
    \begin{enumerate}[label=(\roman*)]
        \item 
            $\scrC + \scrC \subseteq \scrC$,
        \item 
            $\alpha\scrC \subseteq \scrC$ for all $\alpha > 0$.
    \end{enumerate}
\end{definition}

Namely, a cone $\scrC$ is a set closed under taking \defstyle{conical combinations}.

\begin{proposition}
    The set of all cones of a Hilbert space $\scrH$ forms a lattice, with join and meet given by sum and intersection respectively.
\end{proposition}
\begin{proof}
    Let $\scrC_1$ and $\scrC_2$ be cones in $\scrH$.

    Let $x, y \in \scrC_1 \cap \scrC_2$.
    Then $x + y \in \scrC_1 + \scrC_1$
\end{proof}

\begin{definition}
    Let $\scrC$ be a cone.

    The \defstyle{dual cone} $\scrC^\ast$ of $\scrC$ is 
    \[
        \scrC^\ast
        \coloneq
        \Big\{
            L \in \scrH
            :
            \ip{L,K} \geq 0 \text{ for all } K \in \scrC
        \Big\}.
    \]
    We say that $\scrC$ is \defstyle{self-dual} if $\scrC^\ast = \scrC$.
\end{definition}

\begin{proposition}
    Let $\scrC$ be a cone.
    \begin{enumerate}[label=(\roman*)]
        \item 
            $\scrC^\ast$ is a cone.
        \item 
            $\scrC^{\ast\ast} = \overline{\scrC}$.
    \end{enumerate}
\end{proposition}

\begin{definition}
    A \defstyle{extreme ray} of a cone $\scrC$ is
\end{definition}

\subsection{The cone \texorpdfstring{$\PSD_n$}{PSDn}}

\begin{definition}
    Let $\Mat_n$ be the Hilbert space over $\CC$ consisting of $n \times n$ complex matrices, with inner product given by
    \[
        \ip{A,B}_{\Mat_n}
        \coloneq
        \tr (AB^\ast)
    \]
\end{definition}

Let $\Herm_n$ be the \textit{real} Hilbert space consisting of Hermitian matrices.

Namely, $\Herm_n \coloneq \RR.\Mat_n$.

It's a neat fact that $\Herm_n \oplus i\Herm_n = \Mat_n$.

\begin{definition}
    The \defstyle{cone of positive semidefinite matrices} $\PSD_n$ is the subset of $\Herm_n$ consisting of positive semidefinite matrices.
\end{definition}

\subsection{Trigonometric polynomials}

\begin{definition}
    Let $\FF$ be a field and let $X$ be any set.
    We denote the $\FF$-vector space with basis $X$ by $\FF.X$.

    Moreover, if $M \in \PSD_{|X|}$
\end{definition}

\begin{definition}
    Let $\bfx_d$ denote a column vector of variables
    \[
        \bfx_d
        =
        \begin{pmatrix}
            x_1 \\
            \vdots \\
            x_d
        \end{pmatrix}.
    \]
\end{definition}

\iffalse
\begin{definition}
    Let $\bfK \subseteq \ZZ^d$ be a set of \defstyle{exponent vectors}.

    The space $<\bfz^\bfK>$ is a \textit{complex} Hilbert space endowed with the inner product whose structure constants are given by
    \[
        \ip{\bfz^\lambda,\bfz^\mu}
        =
        \delta_{\lambda\mu}
    \]

    The \defstyle{set of trigonometric polynomials} $\Re<e^{i\pm\bfK\bfx}>$ is define to be the following the \textit{real} Hilbert space:

    The \defstyle{cone of nonnegative $\bfK$-supported trigonometric polynomials is} $\CC[\TT^{\bfK\bfx}]^{+}$.

\end{definition}

Let $\CC[e^{i(\bm{\Lambda} - \bm{\Lambda})\bfx}]$
\fi

\end{document}
