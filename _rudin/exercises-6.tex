%   Principles of Mathematical Analysis Chapter 6 Solutions
%   Jasper Ty
%
%   to compile:
%       pdflatex [filename]
\documentclass{article}

% + preamble
\usepackage{mathrsfs}
\usepackage{amsmath}
\usepackage{amssymb}
\usepackage[margin=0.5in]{geometry}

\newenvironment{myboxed}{\noindent\begin{tabular}{|p{.975\linewidth}|}\hline \\}{\\\\\hline\end{tabular}}
\newenvironment{mytitle}{\noindent\large\begin{flushright}}{\end{flushright}\normalsize}

\newcounter{Question}
\newenvironment{Question} 
{\bigskip\begin{myboxed}\refstepcounter{Question}\par\noindent\textbf{Q\theQuestion.}}
{\end{myboxed}\bigskip}
\newenvironment{Answer} {\par\noindent\textbf{A:}} {}
% - preamble

\begin{document}

\begin{mytitle}
    Principles of Mathematical Analysis \\
    Chapter 6 \\
    \normalsize Jasper Ty
\end{mytitle}

\noindent
\footnotesize 
(Note: no guarantee of correctness. Just an undergrad writing their answers down and archiving because notebooks have become unwieldy.)
\normalsize


\begin{Question}
    Suppose $\alpha$ increases on $[a, b]$, $a \leq x_0 \leq b$, $\alpha$ is continuous at $x_0$, $f(x_0) = 1$, and $f(x) = 0$ if $x \neq x_0$. Prove that $f \in \mathscr{R}(\alpha)$ and that $\int f d\alpha = 0$.
\end{Question}
\begin{Answer}
    \textsc{Idea}

    $x_0$ may be covered with an interval of arbitrarily small width, and this segment will have an arbitrarily small weight in the sum due to $\alpha$'s continuity at $x_0$.

    \textsc{Proof that $f \in \mathscr{R}(\alpha)$:}

    Let $\varepsilon > 0$.

    Since $\alpha$ is continuous at $x_0$, choose $\delta$ such that $|x-x_0| \leq 2\delta$ implies $|\alpha(x) - \alpha(x_0)| \leq \varepsilon$. 

    Then let $P = [a, x_0-\delta, x_0 + \delta, b]$. Then
    \begin{align*}
        U(P,f,\alpha) &= \sum_{i=0}^n M_i \Delta\alpha_i \\
        &= M_0 \Delta\alpha_0 + M_1 \Delta\alpha_1 + M_2 \Delta\alpha_2 
    \end{align*}
    Since $f(x) = 0$ on $[a, x_0 - \delta]$ and $[x_0 + \delta, b]$, $M_0 = 0$ and $M_2 = 0$. Moreover, $M_1 = 1$ 
    \[U(P,f,\alpha) = 0 + 1 \cdot \Delta\alpha_1 + 0 \Delta = \alpha_1\]
    As for $L$, the following holds for any partition at all, as any nonempty interval must contain a point that is not $x_0$
    \[L(P,f,\alpha) = 0\]
    So
    \[
        U(P,f,\alpha) - L(P, f, \alpha) = \Delta\alpha_1
        = \alpha(x+\delta) - \alpha(x-\delta) < \varepsilon\]
    Then $f \in \mathscr{R}(\alpha)$.

    \textsc{Proof that $\int fd\alpha = 0$:}

    Since $L(P,f,\alpha) = 0$ for all $P$, 
    \[\underline{\int}fd\alpha = 0\]
    Which lets us conclude 
    \[\int fd\alpha = 0\]
\end{Answer}


\begin{Question}
    Suppose $f \geq 0$, $f$ is continuous on $[a,b]$, and $\int_a^b f(x)dx = 0$. Prove that $f(x) = 0$ for all $x \in [a, b]$. (Compare this with Exercise 1.)
\end{Question}
\begin{Answer}
    \textsc{Idea}

    We show a contradiction that a ``bump'' of nonzero area must exist if $f$ is not identically zero, hence the integral must be nonzero.

    \textsc{Proof}
    
    Choose any arbitrary point $q$ in $[a, b]$. Suppose $f(q) > 0$. Let $m$ be a number such that $0 < m < f(q)$, and use the continuity of $f$ to create an interval $[s, t]$ containing $q$ such that $x \in [s, t]$ implies $f(x) \geq m$. 

    By additivity of the integral,
    \[\int_a^b f dx = \int_a^s f dx + \int_s^t f dx + \int_t^b f dx\]
    Hence
    \[\int_a^b f dx \geq \int_s^t f dx \]
    But $\int_s^t f dx \geq m(t-s) > 0$, so
    \[\int_a^b f dx > 0\]
    A contradiction to the assertion that $\int_a^b fdx = 0$
\end{Answer}


\newpage
\begin{Question}
    Define three functions $\beta_1$, $\beta_2$, $\beta_3$ as follows: $\beta_j(x) = 0$ if $x < 0$, $\beta_j(x) = 1$ if $x > 0$ for $j = 1, 2, 3$; and $\beta_1(0) = 0$, $\beta_2(0) = 1$, $\beta_3(0) = \frac{1}{2}$. Let $f$ be a bounded function on $[-1, 1]$.
    \begin{itemize}
        \item [(a)] Prove that $f \in \mathscr{R}(\beta_1)$ if and only if $f(0^+) = f(0)$ and that then
        \[\int f d\beta_1 = f(0)\]
        \item [(b)] State and prove a similar result for $\beta_2$
        \item [(c)] Prove that $f \in \mathscr{R}(\beta_3)$ if and only if $f$ is continuous at $0$
        \item [(d)] If $f$ is continuous at $0$ prove that
        \[\int f d\beta_1 = \int f d\beta_2 = \int f d\beta_3 = f(0)\]
    \end{itemize}
\end{Question}
\begin{Answer}
        \textsc{Idea}

        With $\beta_i$, the only intervals with non-zero weight in any partition will be those which contain $0$. Moreover, these segments will have fixed weights due to our definitions of $\beta_i$. Then, bounds on the variation of the upper and lower Riemann sums automatically pass through as bounds on $f$ in the segment itself.

        Hence we can convert convergence of Riemann sums into convergence of $f$.
    \begin{itemize}
        \item [(a)]
        \textsc{Proof that $f \in \mathscr{R}(\beta_1)$ implies $f(0+) = f(0)$}

        Let $f \in \mathscr{R}(\beta_1)$. Let $\varepsilon > 0$. Choose a partition $P$ such that $0 \in P$, and
        \[U(P, f, \beta_1) - L(P, f, \beta_1) < \varepsilon\]
        The only nonzero term in either of the sums $U(P,f,\beta_1)$ or $L(P,f,\beta_1)$ comes from the segment containing $x = 0$ as a left endpoint. Let this segment be $[0, x_i]$. Then 
        \[U(P, f, \beta_1) = M_i \cdot \Delta x_i = M_i \cdot (\beta(x_i) - \beta(0)) = M_i \cdot 1 = M_i\]
        \[L(P, f, \beta_1) = M_i \cdot \Delta x_i = m_i \cdot (\beta(x_i) - \beta(0)) = m_i \cdot 1 = m_i\]
        So $U(P, f, \beta_1) - L(P, f, \beta_1) = M_i - m_i < \varepsilon$. This tells us that on $[0, x_i]$, $f$ stays within $\varepsilon$ of $f(0)$. Since $\varepsilon$ was arbitrary, we conclude that $f(0+) = f(0)$
        
        \textsc{Proof that $f(0+) = f(0)$ implies $f \in \mathscr{R}(\beta_1)$}

        Let $f(0+) = f(0)$. Let $\varepsilon > 0$ Choose $\delta$ such that $x \in (0, \delta)$ implies $f(x) \in (f(0) - \varepsilon, f(0) + \varepsilon)$. Let $P$ be the partition $\{a, 0, \delta, b\}$. Then for the same reasons as above,
        \[U(P, f, \beta_1) - L(P, f, \beta_1) < \varepsilon\]
        So $f \in \mathscr{R}(\beta_1)$.

        \textsc{Proof that $\int fd\beta_1 = 0$}

        By Theorem 6.7(b), letting $t_i = 0$, we have that
        \[\left|f(0) - \int f d\beta_1 \right| < \varepsilon\]
        Hence we conclude that
        \[\int f d\beta_1 = f(0)\]
        \item [(b)] 
            \textsc{Statement}

            $f \in \mathscr{R}(\beta_1)$ if and only if $f(0-) = f(0)$, and if it exists,

        \[\int f d\beta_1 = f(0)\]

        \textsc{Proof}

        Extremely similar as before, but with $[x_{i-1}, 0]$ instead of $[0, x_i]$

        \item [(c)]         
            \textsc{Proof that $f \in \mathscr{R}(\beta_3)$ implies $f$ is continuous at $0$}

            Let $f \in \mathscr{R}(\beta_3)$. Let $\varepsilon > 0$. Let $P$ be a partition that such that $0 \in P$ and $U(P, f, \beta_3) - L(P, f, \beta_3) < \varepsilon$

            Then, consider the intervals $[x_{i-1}, 0]$, $[0, x_i]$. We have that
            \[\beta_3(0) - \beta_3(x_{i-1}) = \frac{1}{2}\]
            \[\beta_3(x_i) - \beta_3(0) = \frac{1}{2}\]
            Then 
            \[U(P, f, \beta_3) = \frac{M_i + M_{i-1}}{2}\]
            \[L(P, f, \beta_3) = \frac{m_i + m_{i-1}}{2}\]
            Let $M = \min \{ M_i, M_{i-1} \}$ and $m = \max \{ m_i, m_{i-1}\}$, then
            \[U(P, f, \beta_3) \geq M \]
            \[L(P, f, \beta_3) \leq m\]
            Then 
            \[M - m \leq U(P, f, \beta_3) - L(P, f, \beta_3) < \varepsilon\]
            Hence $f(x)$ is within $\varepsilon$ of $f(0)$ if $f(x)$ is in $[x_{i-1}, x_i]$. Hence $f$ is continuous.

            \textsc{Proof that $f$ continuous at $0$ implies $f \in \mathscr{R}(\beta_3)$}

            Let $f$ be continuous at $0$. Let $\varepsilon > 0$. Choose $\delta$ such that $|x| \leq 2\delta$ implies $|f(x) - f(0)| \leq \varepsilon/2$ then take the partition $P = \{a, -\delta, \delta, b\}$. Then
        \[U(P, f, \beta_3) - L(P, f, \beta_3) < \varepsilon\]
        Hence $f \in \mathscr{R}(\beta_3)$. Also, with a similar argument as in (a), 
        \[\int f d\beta_3 = f(0)\]


        \item [(d)] 
            $f$ being continuous at zero is a strong enough condition for all three previous proofs above to apply. Since all of them show equality of the integral with respect to $\beta_i$ and $f(0)$, the result follows.
    \end{itemize}
\end{Answer}


\begin{Question}
    If $f(x) = 0$ for all irrational $x$, $f(x) = 1$ for all rational $x$, prove that $f \notin \mathscr{R}$ on $[a, b]$ for any $a < b$
\end{Question}
\begin{Answer}
    \textsc{Idea}

    Any nonempty interval will contain both irrational and rational numbers, so the ``gap'' between the lower and upper sums never closes.

    \textsc{Proof}

    Let $P$ be a partition. Since $\mathbb{Q}$ is dense in $\mathbb{R}$, for all $x_i$ we can find two rational numbers $p$ and $q$ such that $x_{i-1} < p < q < x_i$. This tells us that $M_i = 1$ for all $x_i$. Furthermore, the number $s = p + (\sqrt{2}/2)(q-p)$ is irrational and between $p$ and $q$. This tells us that $m_i = 1$ for all $x_i$. Hence it must be that
    \[U(P, f) = 1\]
    \[L(P, f) = 0\]
    for any partition $P$. Then $f$ is not Riemann-integrable.
\end{Answer}


\begin{Question}
    Suppose $f$ is a bounded real function on $[a, b]$, and $f^2 \in \mathscr{R}$ on $[a,b]$. Does it follow that $f \in \mathscr{R}$? Does the answer change if we assume that $f^3 \in \mathscr{R}$?
\end{Question}
\begin{Answer}
    \textsc{Idea}

    Because of squaring's non-injectivity, we can construct ``terrible'' discontinuities which can be undone via squaring. On the other hand, cubing is much nicer (it is a continuous bijection of the real line to itself).

    \textsc{Proof}

    No. Consider the function
    \[
        f(x) := \begin{cases}
        1 & x \text{ rational} \\
        -1 & x \text{ irrational} \\
        \end{cases}
    \]
    We have that $(f^2)(x) = 1$, so $\int_a^b f^2 dx = (b-a)$. But $f$ itself is not Riemann-integrable, as the exercise above shows.

    For $f^3$, use Theorem 6.11, which tells us that the composition of continuous functions with integrable functions is integrable. Let $\phi = \sqrt[3]{x}$. Then $\phi \circ (f^3) \in \mathscr{R}$ Since $\phi \circ f^3 = f$, $f \in \mathscr{R}$.
\end{Answer}


\begin{Question}
    Let $P$ be the Cantor set constructed in Sec. 2.44. Let $f$ be a bounded real function on $[0, 1]$ which is continuous at every point outside $P$. Prove that $f \in \mathscr{R}$ on $[0,1]$. \textit{Hint:} $P$ can be covered by finitely many segments whose total length can be made as small as desired. Proceed as in Theorem 6.10.
\end{Question}
\begin{Answer}
    Each set of the sequence that constructs the Cantor set has $2^n$ intervals, each of width $1/3^n$. Let $E_n$ be the $n$th set in the sequence. Cover each interval in $E_n$ with an open interval whose endpoints have a distance of $1.1\cdot(1/3^n)$. Then, the sum of all lengths of each cover will be $1.1 \cdot (2/3)^n$. This also covers the Cantor set itself, since the Cantor set $E$ is a subset of $E_n$ for all $n$. Hence, we have made a cover of the Cantor set with arbitrarily small area.

    Take the endpoints of each interval and construct a partition, and proceed as in Theorem 6.10. 
    This sequence of partitions eventually.
\end{Answer}


\newpage
\begin{Question}
    Suppose $f$ is a real function on $(0, 1]$ and $f \in \mathscr{R}$ on $[c,1]$ for every $c > 0$. Define
    \[\int_0^1 f(x) dx = \lim_{c\to 0} \int_c^1 f(x)dx\]
    if this limit exists (and is finite).
    \begin{itemize}
        \item[(a)] If $f \in \mathscr{R}$ on $[0, 1]$, show that this definition of the integral agrees with the old one.
        \item[(b)] Construct a function $F$ such that the above limit exists, although it fails to exist with $|f|$ in place of $f$
    \end{itemize}
\end{Question}
\begin{Answer}
    \begin{itemize}
        \item[(a)]
            Lemma: If $f \in \mathscr{R}$ on $[a ,b]$, then
            \[\lim_{h \to a} \int_a^h f(x)dx = 0\]
            Proof: Let $|f(x)| \leq M$. Let $\varepsilon > 0$. Choose $\delta = \varepsilon/M$. Then, letting $h \leq \delta$
            \[\left|\int_a^h f(x) dx\right| \leq M((a+h) - a) = Mh \leq M\delta = \varepsilon\]
            Since the $\varepsilon$ was arbitrary, the lemma holds.
            Then, by Theorem 6.12, if $f \in \mathscr{R}$ on $[0, 1]$,
            \[\int_0^1 f(x)dx = \int_0^c f(x)dx + \int_c^1 f(x)dx\]
            Then, taking limits
            \begin{align*}
                \lim_{c \to 0} \int_0^1 f(x)dx &= \lim_{c \to 0}\int_0^c f(x)dx + \lim_{c \to 0}\int_c^1 f(x)dx \\
                \int_0^1 f(x)dx &= 0 + \lim_{c \to 0}\int_c^1 f(x)dx \\
                \int_0^1 f(x)dx &= \lim_{c \to 0}\int_c^1 f(x)dx \\
            \end{align*}
            Hence showing that the definition agrees.
        \item[(b)]
            Let $n = 1, 2, \ldots$. Consider the function (with infinitely many cases)
            \[f(x) = \begin{cases}
                (-1)^{n+1}n+1 & \text{if } \frac{1}{n+1} \leq x < \frac{1}{n}
            \end{cases}\]
            Then the graph of $f$ consists of boxes of width $(1/n) - (1/(n+1))$ and height $n+1$. The area of each box is then $1/n$. Then as the lower bound of the integral approaches $0$, the integral approximates the alternating sum $S = 1 - 1/2 + 1/3 - 1/4 + \cdots$, which converges to $\ln 2$.
            However, in the case of $|f|$, the integral apporximates the sum $S = 1 + 1/2 + 1/3 + 1/4 + \cdots$, which diverges.
    \end{itemize}
\end{Answer}


\newpage
\begin{Question}
    Suppose $f \in \mathscr{R}$ on $[a, b]$ for every $b > a$ where $a$ is fixed. Define 
    \[\int_a^\infty f(x) dx = \lim_{b \to \infty} \int_a^b f(x) dx\]
    if this limit exists (and is finite). In that case, we say that the integral on the left \textit{converges}. If it also converges after $f$ has been replaced by $|f|$, it is said to converge \textit{absolutely}.
    Assume that $f(x) \geq 0$ and that $f$ decreases monotonically on $[1, \infty)$. Prove that
    \[\int_1^\infty f(x) dx\] 
    converges if and only if
    \[\sum_{n=1}^\infty f(n)\] 
    converges. (This is the so-called ``integral test" for convergence of series.)
\end{Question}
\begin{Answer}
    Suppose $\int_1^\infty f(x) dx$ converges. Then if we show that, for any positive integer $b$,
    \[\sum_{n=1}^b f(n) \leq \int_1^b f(x) dx\]
    we can prove convergence of the sum, as the partial sums form a monotonic sequence ($f$ is non-negative).
    Let $P$ be the partition consisting of the points $0, 1, 2, \ldots, b-1, b$. Then, since $f$ is monotonically decreasing, 
    \[\inf_{x \in [n-1, n]} f(x) = f(n)\]
    Hence 
    \[\sum_{n=1}^b f(n) = L(P, f) \leq \int_1^b f(x) dx \]
    which was the inequality we wanted. Then the sum converges.
    
    Suppose $\int_1^\infty f(x) dx$ converges. Then we make a similar argument, with $U(P, f)$ bounding $\int_1^b f(x) dx$ from above. Take the same partition of $[a, b]$, then 
    \[\sup_{x \in [n, n+1]} f(x) = f(n)\]
    Hence 
    \[U(P, f) = \sum_{n=0}^{b-1} f(n)\]
    Which proves that the integral converges.
\end{Answer}


\begin{Question}
    Show that integration by parts can sometimes be applied to the ``improper" integrals defined in Exercises $7$ and $8$. (State appropriate hypotheses, formulate a theorem, and prove it.) For instance show that

    \[\int_0^\infty \frac{\cos x}{1+x} dx = \int_0^\infty \frac{\sin x}{(1+x)^2}\]
\end{Question}


\begin{Question}
    Let $p$ and $q$ be positive real numbers such that
    \[\frac{1}{q} + \frac{1}{q} = 1\]
    Prove the following statements.
    \begin{itemize}
        \item[(a)] 
            If $u \geq 0$ and $v \geq 0$, then
            \[uv \leq \frac{u^p}{p} + \frac{v^q}{q}\]
            Equality holds if and only if $u^p = v^q$.
        \item[(b)]
            If $f \in \mathscr{R}(\alpha)$, $g \in \mathscr{R}(\alpha)$, $f \geq 0$, $g \geq 0$, and
            \[\int_a^b f^p \: d\alpha = 1 = \int_a^b g^q \: d\alpha\]
            then
            \[\int_a^b fg \: d\alpha \leq 1\]
        \item[(c)] 
            If $f$ and $g$ are complex functions in $\mathscr{R}(\alpha)$, then 
            \[\left|\int_a^b fg \: d\alpha\right| \leq \left\{\int_a^b |f|^p \: d\alpha \right\}^{\frac{1}{p}}\left\{\int_a^b |g|^q \: d\alpha\right\}^{\frac{1}{q}}\]
            This is \textit{Hölder's inequality}. When $p = q = 2$ it is usually called the Schwarz inequality. (Note that Theorem 1.35 is a very special case of this.)
        \item[(d)]
            Show that Hölder's inequality is also true for the ``improper'' integrals described in Exercises $7$ and $8$.
    \end{itemize}
\end{Question}
\begin{Answer}
    \begin{itemize}
        \item[(a)]
            Let, for $x \geq 0$ and $y \geq 0$,
            \[f(x) := x^{p-1}\]
            \[g(y) := y^{q-1}\]
            Then $f$ and $g$ are both strictly increasing functions. Furthermore, since $(p-1)(q-1) = 1$, they are inverses to one another.

        \textsc{Lemma}
        
        Let $f$ be a strictly increasing, invertible function on $[a, b]$ that is also $\mathscr{R}(\alpha)$ on $[a ,b]$. Then
        \[(b-a)(f(b)-f(a)) = \int_a^b f dx + \int_{f(a)}^{f(b)}f^{-1} dx - (b-a)(f(a)) - (f(b)-f(a))(a)\]

        \textbf{Proof}

        Let $P$ be any partition at all of $[a, b]$. Construct a corresponding partition $P'$ of $[f(a), f(b)]$ made of points $y_i := f(x_i)$. This is a valid partition as $f$ is increasing, and the endpoints map correctly. 
        Now consider the facts 
        \[b-a = \sum_{i=1}^n \Delta x_i\]
        \[f(b)-f(a) = \sum_{i=1}^n \Delta y_i\]
        Hence 
        \[(b-a)(f(b)-f(a)) = \sum_{1 \leq i , j \leq n} \Delta x_i \Delta y_j\]
        Let $m_i$ denote the inf of $f$ in $[x_{i-1}, x_i]$, and let $m'_i$ denote the inf of $f^{-1}$ on $[y_{i-1}, y_i]$. Define $M_i$ and $M'_i$ similarly.
        By the monotonicity of $f$, we know that 
        \[m_i = y_{i-1} = f(x_{i-1})\]
        \[m'_i = x_{i-1} = f^{-1}(y_{i-1})\]
        \[M_i = y_i = f(x_i)\]
        \[M'_i = x_i = f^{-1}(y_i)\]
        We also know that
        \[x_i - a = \sum_{j=1}^i \Delta x_j\]
        \[y_i - f(a) = \sum_{j=1}^i \Delta y_j\]
        Hence $U(P, f)$ may be written as
        \[U(P, f) = \sum_{i=1}^n M_i \Delta x_i = \sum_{i=1}^n y_i \Delta x_i = \sum_{i=1}^n \left[\sum_{j=1}^i \Delta y_j + f(a)\right] \Delta x_i\]
        Simplifying the sums
        \[U(P, f) = \left[\sum_{1 \leq j \leq i \leq n} \Delta x_i \Delta y_j \right] + (b-a)(f(a))\]
        Similarly,
        \[U(P', f^{-1}) = \left[\sum_{1 \leq i \leq j \leq n} \Delta x_i \Delta y_j \right] + (f(b)-f(a))(a)\]
        Consider the term $\Delta x_i \Delta y_k$. Either $j < i$, in which case it is part of the sum of $U(P, f)$ but not of $U(P', f^{-1})$, or $i > j$ in which case it is part of the sum of $U(P',f^{-1})$ but not of $U(P, f)$. Or $i = j$, in which case it is part of both.
        Hence, by inclusion-exclusion,
    \[U(P, f) + U(P', f^{-1}) = \left[ \sum_{1 \leq i , j \leq n} \Delta x_i \Delta y_j - \sum_{k=1}^n \Delta x_k \Delta y_k \right] + (b-a)\cdot f(a) + (f(b)-f(a))\cdot a\]
        But since $\Delta y_k = y_k - y_{k-1} = M_k - m_k$, 
        \[\sum_{k=1}^n \Delta x_k \Delta y_k = U(P, f) - L(P,f)\]
        And in the same way, with $\Delta x_k = x_k - x_{k-1} = M'_i - m'i$
        \[\sum_{k=1}^n \Delta x_k \Delta y_k = U(P', f^{-1}) - L(P',f^{-1})\]
        Hence choose a partition $P$ such that 
        \[U(P, f) - L(P,f) \leq \varepsilon\]
        We automatically get a partition such that
        \[U(P', f) - L(P',f) \leq \varepsilon\]
        and we show that
    \[U(P, f) + U(P', f^{-1}) = \left[ (b-a)(f(b)-f(a)) - \varepsilon \right] + (b-a)\cdot f(a) + (f(b)-f(a))\cdot a\]
        Similarly,
        \[L(P, f) + L(P', f^{-1}) = \left[ (b-a)(f(b)-f(a)) + \varepsilon \right] + (b-a)\cdot f(a) + (f(b)-f(a))\cdot a\]
        Then we have the inequalities
    \[U(P, f) + U(P', f^{-1}) \leq (b-a)(f(b)-f(a)) + (b-a)\cdot f(a) + (f(b)-f(a))\cdot a\]
        \[L(P, f) + L(P', f^{-1}) \geq (b-a)(f(b)-f(a)) + (b-a)\cdot f(a) + (f(b)-f(a))\cdot a\]

        Hence
        \[\int_a^b f dx + \int_{f(a)}^{f(b)} f^{-1} dx = (b-a)(f(b)-f(a)) + (b-a)\cdot f(a) + (f(b) - f(a))\cdot a\]
        Which is the result we wanted.

            Consider the quantities $u^{p-1}$ and $v^{q-1}$. If they are not equal, then there are two possibilities:
            \[u^{p-1} > v \text{ and } u > v^{q-1}\]
            \[u^{p-1} < v \text{ and } u < v^{q-1}\]
        Suppose, without loss of generality, that $u^{p-1} \geq v$

        Then 
        \[ uv = (u - v^{q-1} + v^{q-1}) v = v^{q-1} \cdot v + (u-v^{q-1})v\] 
        By the lemma,
        \[v^{q-1} \cdot v = \int_0^{v^{q-1}} x^{p-1} dx + \int_0^v y^{q-1} dy\]
        so
        \[uv = \int_0^{v^{q-1}} x^{p-1} dx + \int_0^v y^{q-1} dy + (u - v^{q-1}) v\]
        By the monotonicity of $x^{p-1}$, the $\inf$ of $x^{p-1}$ on $[v^{q-1}, u]$ is $v^{(q-1)(p-1)} = v$, so 
        \[(u - v^{q-1})v \leq \int_{v^{q-1}}^u x^{p-1}dx\]
        Then 
        \[uv \leq \int_0^{v^{q-1}} x^{p-1} dx + \int_0^v y^{q-1} dy + \int_{v^{q-1}}^u x^{p-1} dx\] 
        Combining the integrals with the common integrand $x^{p-1}$,
        \[uv \leq \int_0^u x^{p-1} + \int_0^v y^{q-1} dy\]
        Finally, evaluating the integrals,
        \[uv \leq \frac{u^p}{p} + \frac{y^q}{q}\]

    \item[(b)]
        By the previous part, we have that
        \[fg \leq \frac{f^p}{p} + \frac{g^q}{q}\]
        Then
        \[\int_a^b fg \: d\alpha \leq \int_a^b \left[\frac{f^p}{p} + \frac{g^q}{q}\right]dx\]
        Using properties of the integral,
        \[\int_a^b fg \: d\alpha \leq \frac{1}{p} \int_a^b f^p \: dx + \frac{1}{q}\int_a^b g^q \: dx\]
        Since the integrals on the right hand side equal $1$
        \[\int_a^b fg \: d\alpha \leq \frac{1}{p} + \frac{1}{q} = 1\]

    \item[(c)]
        Let $f$ and $g$ be any $\mathscr{R}(\alpha)$ functions at all. Then $|f|$ and $|g|$ are also $\mathscr{R}(\alpha)$. Then the integrals.
        \[\int_a^b |f|^p \: d\alpha\]
        \[\int_a^b |g|^q \: d\alpha\]
        exist and are finite.  Suppose they are both positive. Define 
        \[P := \int_a^b |f|^p \: d\alpha\]
        \[Q := \int_a^b |g|^q \: d\alpha\]
        Then define the functions
        \[h := \frac{|f|}{P^{\frac{1}{p}}}\]
        \[k := \frac{|g|}{Q^{\frac{1}{q}}}\]
        We also have that $\overline{f}$ and $\overline{g}$ are nonnegative, hence the inequality in (b) applies, and
        \[\int_a^b hk \: d\alpha \leq 1\]
        Then
        \[\int_a^b \frac{|f||g|}{P^{\frac{1}{p}}Q^{\frac{1}{q}}} \: d\alpha \leq 1\]
        \[\frac{1}{P^{\frac{1}{p}}Q^{\frac{1}{q}}}\int_a^b |f||g| d\alpha \leq 1\]
        \[\int_a^b |f||g| d\alpha \leq P^{\frac{1}{p}}Q^{\frac{1}{q}} \]
        Since we know that
        \[\left|\int_a^b fg \: d\alpha \right| \leq \int_a^b |fg| \: d\alpha \leq \int_a^b |f||g| \: d\alpha\]
        and that
        \[P^{\frac{1}{p}}Q^{\frac{1}{q}} = \left\{\int_a^b |f|^p \: d\alpha \right\}^{\frac{1}{p}}\left\{\int_a^b |g|^q \: d\alpha\right\}^{\frac{1}{q}}\]
        we conclude that
        \[\left|\int_a^b fg d\alpha\right| \leq \left\{\int_a^b |f|^p \: d\alpha \right\}^{\frac{1}{p}}\left\{\int_a^b |g|^q \: d\alpha\right\}^{\frac{1}{q}} \]

    \item[(d)]

        
    \end{itemize}
\end{Answer}
\end{document}
