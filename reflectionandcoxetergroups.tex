\documentclass{article}

% See https://github.com/Jasper-Ty/dotfiles
\usepackage[garamond]{jaspercommon}

\newcommand*\wc{{\mkern 3mu\_\mkern 3mu}}
\newcommand{\innerproduct}[2]{\ensuremath{\left\langle #1 , #2 \right\rangle}}

\DeclareMathOperator{\s}{s}
\DeclareMathOperator{\ogroup}{O}
\DeclareMathOperator{\Dih}{Dih}
\DeclareMathOperator{\Sym}{Sym}

\title{Reflection groups and Coxeter groups}
\author{Jasper Ty}
\date{}

\titleauthorhead

\begin{document}

\maketitle

\section*{What is this?}

These are notes based on my reading of Humphreys's ``Reflection groups and Coxeter groups''.

\tableofcontents

\section{Finite reflection groups}

We define \textit{reflections}.

\begin{definition}
    Let $V$ be a real Euclidean space, equipped with an inner product $\innerproduct{\wc}{\wc}$.
    Let $\alpha \in V$ be some nonzero vector.
    The \textit{reflection $\s_\alpha$} is defined to be.
    \[
        \s_\alpha \lambda \coloneq \lambda - \frac{2\innerproduct{\lambda}{\alpha}}{\innerproduct{\alpha}{\alpha}}\alpha.
    \]
\end{definition}

This has the effect of sending $\alpha$ to $-\alpha$, and fixing the hyperplane orthogonal to $\alpha$.

We do some paperwork to show that this is all defined correctly.

\begin{proposition}
    Let $V$ be a real vector space, and let $\alpha \in V$.
    The reflection $\s_\alpha$ is an orthogonal linear operator, which sends $\alpha$ to $-\alpha$, and fixes the hyperplane
    \[
        H_\alpha \coloneq \{w \in V: \innerproduct{w}{\alpha} = 0\}.
    \]
\end{proposition}

\begin{proof}
    Linearity follows from the bilinearity of the inner product.
    Let $a,b \in \RR$.
    Then
    \begin{align*}
        \s_\alpha (a\lambda+b\beta)
        &=
        (a\lambda+b\beta) - \frac{2\innerproduct{a\lambda+b\beta}{\alpha}}{\innerproduct{\alpha}{\alpha}}\alpha \\
        &=
        (a\lambda+b\beta) - \frac{2\Big(a\innerproduct{\lambda}{\alpha}+b\innerproduct{\beta}{\alpha}\Big)}{\innerproduct{\alpha}{\alpha}}\alpha \\
        &=
        (a\lambda+b\beta) - \left(\frac{2a\innerproduct{\lambda}{\alpha}}{\innerproduct{\alpha}{\alpha}}\alpha + \frac{2b\innerproduct{\beta}{\alpha}}{\innerproduct{\alpha}{\alpha}}\alpha\right) \\
        &=
        \left(a\lambda - \frac{2a\innerproduct{\lambda}{\alpha}}{\innerproduct{\alpha}{\alpha}}\alpha\right) +
        \left(b\beta - \frac{2b\innerproduct{\beta}{\alpha}}{\innerproduct{\alpha}{\alpha}}\alpha\right) \\
        &=
        a\left(\lambda - \frac{2\innerproduct{\lambda}{\alpha}}{\innerproduct{\alpha}{\alpha}}\alpha\right) +
        b\left(\beta - \frac{2\innerproduct{\beta}{\alpha}}{\innerproduct{\alpha}{\alpha}}\alpha\right) \\
        &=
        a\s_\alpha\lambda + b \s_\alpha\beta.
    \end{align*}
    
    Next, we compute that it is orthogonal.
    \begin{align*}
        &\innerproduct{\s_\alpha\lambda}{\s_\alpha\beta} \\
        &=
        \innerproduct{\lambda - \frac{2\innerproduct{\lambda}{\alpha}}{\innerproduct{\alpha}{\alpha}}\alpha}{\beta- \frac{2\innerproduct{\beta}{\alpha}}{\innerproduct{\alpha}{\alpha}}\alpha} \\
        &= 
        \innerproduct{\lambda}{\beta}
        +
        \innerproduct{\lambda}{-\frac{2\innerproduct{\beta}{\alpha}}{\innerproduct{\alpha}{\alpha}}\alpha}
        +
        \innerproduct{-\frac{2\innerproduct{\lambda}{\alpha}}{\innerproduct{\alpha}{\alpha}}\alpha}{\beta}
        +
        \innerproduct{-\frac{2\innerproduct{\lambda}{\alpha}}{\innerproduct{\alpha}{\alpha}}\alpha}{-\frac{2\innerproduct{\beta}{\alpha}}{\innerproduct{\alpha}{\alpha}}\alpha} \\
        &= 
        \innerproduct{\lambda}{\beta}
        -
        \frac{2\innerproduct{\beta}{\alpha}}{\innerproduct{\alpha}{\alpha}}
        \innerproduct{\lambda}{\alpha}
        -
        \frac{2\innerproduct{\lambda}{\alpha}}{\innerproduct{\alpha}{\alpha}}
        \innerproduct{\alpha}{\beta}
        +
        \frac{4\innerproduct{\lambda}{\alpha}\innerproduct{\beta}{\alpha}}{\innerproduct{\alpha}{\alpha}^2}
        \innerproduct{\alpha}{\alpha} \\
        &=
        \innerproduct{\lambda}{\beta}
        -
        \frac{4\innerproduct{\lambda}{\alpha}\innerproduct{\beta}{\alpha}}{\innerproduct{\alpha}{\alpha}}
        +
        \frac{4\innerproduct{\lambda}{\alpha}\innerproduct{\beta}{\alpha}}{\innerproduct{\alpha}{\alpha}} \\
        &= 
        \innerproduct{\lambda}{\beta}.
    \end{align*}
    Finally, we verify its ``reflection'' properties.
    First,
    \[
        \s_\alpha \alpha = \alpha - \frac{2\innerproduct{\alpha}{\alpha}}{\innerproduct{\alpha}{\alpha}}\alpha = \alpha - 2\alpha = -\alpha.
    \]
    And, if $\innerproduct{\lambda}{\alpha} = 0$
    \[
        \s_\alpha \lambda = \lambda - \frac{2\innerproduct{\lambda}{\alpha}}{\innerproduct{\alpha}{\alpha}}\alpha = \lambda - \frac{2 \cdot 0}{\innerproduct{\alpha}{\alpha}} = \lambda.
    \]
\end{proof}

And finally, 

\begin{remark}
    Let $\s_\alpha$ be a reflection in $V$.
    Then it is involutory, i.e $\s_\alpha^2$ is the identity operator on $V$.
\end{remark}

\begin{proof}
    We know that $V = \RR\alpha \oplus H_\alpha$.
    Then if we write down a vector $w \in V$ as $p\alpha + q\lambda$, where $p,q \in \RR$ and $\lambda \in H_\alpha$,
    \[
        \s_\alpha \s_\alpha w = \s_\alpha \s_\alpha (p\alpha) + \s_\alpha \s_\alpha (q\lambda) = \s_\alpha(-p\alpha) + \s_\alpha(q\lambda) = p\alpha + q\beta = w.
    \]
\end{proof}

\begin{definition}
    A \textit{finite reflection group} is a finite subgroup of $\ogroup(V)$ generated by reflections.
\end{definition}

\subsection{Some examples}

\begin{definition}
    Let $V = \RR^2$, and fix some integer $m \geq 3$.
    The \textit{dihedral group $\Dih_{m}$} is the finite reflection group consisting of all orthogonal transformations which fix a regular $m$-sided polygon centered at the origin.
\end{definition}

\begin{definition}
    The \textit{symmetric group} $\Sym_n$ is the finite reflection group consisting of all orthogonal transformations which swap two basis vectors.
\end{definition}




\end{document}
