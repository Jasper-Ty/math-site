\documentclass{article}

\usepackage[garamond, tableau]{jaspercommon}

\newcommand{\frkS}{\ensuremath\mathfrak{S}}
\DeclareMathOperator{\sgn}{sgn}

\title{Schubert polynomials}
\author{Jasper Ty}
\date{}

\titleauthorhead

\begin{document}

\maketitle

These are notes based on my study of Schubert polynomials. My main references are \cite{MacdonaldSchubertPolynomials} and \cite{KnutsonSchubertPolynomials}.

\tableofcontents

\section{Notation and conventions}

\subsection{Sets}

We take $\NN$ to be the set of natural numbers \textit{including} zero,
\[
    \NN := \{0,1,2,\ldots\}.
\]
We take $\PP$ to be the set of \textit{positive integers},
\[
    \PP := \{1,2,\ldots\}.
\]
$\ZZ,\QQ,\RR,\CC$ are defined as usual.

We denote the set $\{1,\ldots,n\}$ by $[n]$.

\subsection{Partitions and compositions}

A \textit{weak composition} $\alpha$ of $n \in \NN$ is an infinite tuple of nonnegative integers 
\[
    (\alpha_1, \alpha_2, \ldots)
\]
such that $\sum_i \alpha_i = n$. 
We define $|\alpha| = \sum_i \alpha_i$ to have notation for recovering $n$ given $\alpha$.

A \textit{partition} $\lambda$ of $n$ is a weak composition whose entries are \textit{weakly decreasing}. 
That a particular partition $\lambda$ is a partition of a particular $n$ is denoted $\lambda \vdash n$. 
We define $|\lambda|$ the exact same way.

I use English notation when drawing diagrams and tableaux, meaning, row index increases \textit{north to south}, and column index increases \textit{west to east}.

\subsection{Rings, polynomials, and formal power series}

The following notation is (mostly) in accordance with the notation in \cite{DarijAC}, with a few additions. 

All rings considered are commutative and unital. An arbitrary ring will be denoted $\KK$. 

$\KK[[t]]$ will denote the formal power series ring over $\KK$ in the indeterminate $t$.

We will fix notation for the following sets of indeterminates, which we will use when convenient:
\begin{enumerate}[label=(\alph*)]
    \item $X_N := (x_1, x_2, \ldots x_N)$ for a set of $N$ indeterminates.
    \item $X := (x_1, x_2, \ldots)$ for a set of countably many indeterminates.
    \item $Y$, $Y_N$, $Z$, $Z_N$, $Q$, $Q_N$ and so on are defined similarly.
\end{enumerate}

With compositions, partitions, or otherwise any finitely supported tuple of nonnegative integers $\alpha$, we define \textit{multi-index notation} for compactly writing down monomials.
\[
    x^\alpha := x_1^{\alpha_1}x_2^{\alpha_2}x_3^{\alpha_3}\cdots.
\]
We will let $[x^\alpha]f$ denote the coefficient of $[x^\alpha]$ in the polynomial or formal power series $f$.

\subsection{Permutations and the symmetric group}

$S_n$ will denote the symmetric group on $n$ letters.

I use cycle notation, so e.g the cycle that sends $1$ to $7$, $7$ to $4$, and $4$ to $1$ will be written as $(174)$.

The simple transpositions $(i\:i+1)$ will be denoted $s_i$.

The identity permutation will be denoted $1$.

The length of a permutation $w$ will be denoted $\ell(w)$.

Permutations will act on polynomials or power series by permuting \textit{places}, meaning that if $\sigma \in S_n$ and $f(x_1, \ldots, x_n) \in \KK[X_n]$, we define
\[
    \sigma f(x_1, \ldots, x_n) := f(x_{\sigma(1)}, \ldots x_{\sigma(n)}).
\]

\section{Permutations}

We recall here relevant tidbits about permutations. 

\begin{definition}
    Let $w \in S_n$. 
    An \textit{inversion} of $w$ is a pair $i < j$ such that $w(i) > w(j)$.
    The \textit{inversion number} of $w$ is the number of inversions of $w$, and we denote this with $\ell(w)$.
\end{definition}

We note that it's particularly easy to see that $\ell(w)$ is well-defined (as the size of a well-defined subset of $[n]\times[n]$). 

It gives us, then, an easy way to define the simplest, most famous permutation statistic:

\begin{definition}
    We define the \textit{sign} of a permutation $w$ to be
    \[
        (-1)^w := (-1)^{\ell(w)}.
    \]
\end{definition}

This coincides with more typical definitions.

\begin{remark}
    Let $w \in S_n$.
    The quantity $(-1)^{\ell(w)}$ agrees with the following
    \begin{enumerate}[label=(\alph*)]
        \item $\sgn(w)$, where $\sgn$ is the usual \textit{sign homomorphism} $\sgn: S_n \to \{-1, 1\}$. 
        \item $(-1)^k$, where $k$ is the length of \textit{any} decomposition of $w$ into a product of transpositions.
    \end{enumerate}
\end{remark}

\begin{proof}
    See section 5.4 in \cite{DarijAC}.
\end{proof}

We happen to be interested in a particular kind of decomposition of a permutation $w$ as a product of transpositions:

\begin{definition}
    Let $w \in S_n$. 
    A \textit{Coxeter word} for $w$ is a sequence of simple transpositions $s_{i_1}, \ldots, s_{i_k}$ such that
    \[
        w = s_{i_1}\cdots s_{i_k}.
    \]
    We call a Coxeter word a \textit{reduced word} if it's of minimal length, that is, there is no shorter Coxeter word for $w$.
\end{definition}

The following theorem is important, and has a detailed proof, as Theorem 5.3.17, in \cite{DarijAC}.

\begin{theorem}
    Let $w \in S_n$.
    Then there exist Coxeter words for $w$, and their minimal length is $\ell(w)$, i.e reduced words for $w$ have length $\ell(w)$.
\end{theorem}

\begin{proof}[Proof (sketch)]
    We kill one and a half birds with one stone by first showing existence of Coxeter words for $w$ with length $\ell(w)$. 
    The remaining half a bird is showing that it is a reduced word.

    The key fact is that simples $s_i$, when multiplied on the right, either increment or decrement the inversion number--- if $(i, i+1)$ is an inversion, then $s_i$ \textit{deletes} it, otherwise, $s_i$ \textit{creates} an inversion $(i, i+1)$.

    This makes existence amenable to proof by induction on $\ell(w)$. 

    For the base case, the only permutation $w$ with $\ell(w) = 0$ is the identity permutation, a product of zero simples.

    For the induction step, let $w$ be a permutation and suppose $\ell(w) = h > 0$ and assume (induction hypothesis) existence of Coxeter words for all permutations $w'$ where $\ell(w') = h-1$.
    Then we hit $w$ with a simple $s_k$ that cancels out one of its inversions.
    
    Then $\ell(ws_k) = \ell(w)-1 = h-1$, so there exists a Coxeter word $s_{i_1}\cdots s_{i_{h-1}}$ for $ws_k$.
    Then $s_{i_1}\cdots s_{i_{h-1}}s_k$ is a Coxeter word of length $h = \ell(w)$ for $w$.

    The fact that we have a reduced word follows from $s_i$'s at most only incrementing inversion number--- you can't get $\ell(w) = h$ with fewer than $h$ simples!
\end{proof}

Then, what do we know about $\ell(w)$?

\begin{definition}
    Let
    \[
        w_0 := n,n-1,\ldots,1.
    \]
    Equivalently, it's the permutation that maximizes the number of inversions, which happens to be 
    \[
        \ell(w_0) = \frac{n(n-1)}{2}.
    \]
\end{definition}

\begin{theorem}\label{thm:simplecoxeterrelations}
    The simple transpositions satisfy the \textit{Coxeter-Moore relations}
    \begin{enumerate}[label=(\alph*)]
        \item Braid relation 
            \begin{equation}\label{eq:simplebraid}
                s_is_{i+1}s_i = s_{i+1}s_is_{i+1}
            \end{equation}
        \item Far commutativity
            \begin{equation}\label{eq:simplecommute}
                s_is_j = s_js_i\qquad \text{whenever}\qquad|i-j| > 1
            \end{equation}
        \item Contraction
            \begin{equation}\label{eq:simplecontract}
                s_i^2 = 1
            \end{equation}
    \end{enumerate}
\end{theorem}

\section{Divided difference operators}

\subsubsection{Definition}

\begin{definition}
    Let $f$ be a polynomial. 
    We define the \textit{divided difference operator} $\partial_i$ by 
    \begin{equation}
        \partial_i f := \frac{f-s_if}{x_i-x_{i+1}}
    \end{equation}
\end{definition}

\begin{example}
    If $f(x_1, x_2, x_3) = x_1x_2$, then
    \begin{align*}
        \partial_2 f(x_1,x_2,x_3) &= \frac{x_1x_2 - x_1x_3}{x_2 - x_3} \\
                                  &= x_1\left(\frac{x_2-x_3}{x_2-x_3}\right) \\
                                  &= x_1.
    \end{align*}
\end{example}

\subsubsection{Basic facts}

We have the following characterization of $\partial_i$ that does not invoke division.

\begin{lemma}
    Fix $i$. Consider some monomial $f = \cdots x_i^ax_{i+1}^b \cdots$. Then
    \[
        \partial_i(\cdots x_i^a x_{i+1}^b \cdots) = 
        \varepsilon_{ba}
        \sum_{\substack{u,v \geq \min\{a,b\} \\ u+v = a+b-1}} \cdots x_i^u x_{i+1}^v \cdots,
    \]
    where $\varepsilon$ is defined to be
    \[
        \varepsilon_{rs} := \begin{cases}
            0 & \text{if } r = s \\
            1 & \text{if } r < s \\
            -1 & \text{if }r > s 
        \end{cases}.
    \]
\end{lemma}

\begin{proof}
    The proof is not hard but it's a slog. We compute
    \begin{align*}
        \partial_i(\cdots x_i^a x_{i+1}^b \cdots) &= \frac{(\cdots x_i^a x_{i+1}^b \cdots) - (\cdots x_i^b x_{i+1}^a \cdots)}{x_i-x_{i+1}} \\
                                                                          &= (\cdots) \frac{x_i^ax_{i+1}^b - x_i^bx_{i+1}^a}{x_i-x_{i+1}}.
    \end{align*}
    Recall that in any commutative ring we have that
    \[
        \frac{x^n-y^n}{x-y} = x^{n-1}y^0 + x^{n-2}y^1 + \cdots + x^1y^{n-2} + x^0y^{n-1},
    \]
    which we will modify a little
    \[
        \frac{x^{n+m}y^m-x^my^{n+m}}{x-y} = x^{m+n-1}y^m + x^{m+n-2}y^{m+1} + \cdots + x^{m+1}y^{m+n-2}x^my^{m+n-1},
    \]
    and we note that the pairs $(u,v) \in \{(m+n-1,m),\ldots,(m,m+n-1)\}$ are precisely those such that $u,v \geq \min\{a,b\}$ and $u+v = 2m+n-1$. We then put $a=m+n$ and $b=m$, to get that
    \[
        \frac{x^ay^b-x^ay^b}{x-y} = \sum_{\substack{u,v \geq \min\{a,b\}\\u+v=a+b-1}} x^uy^v, \qquad \text{given }a \geq b.
    \]
    Then, to forget $a \geq b$, we pick up a $\varepsilon_{ba}$ term to keep track of sign. 
    Applying this identity now to our computation, we finish the lemma.
\end{proof}

Then the following properties of the operator $\partial_i$ can be read off

\begin{corollary} 
    Let $f$ be a polynomial.
    \begin{enumerate}[label=(\alph*)]
        \item $\partial_if$ is a polynomial.
        \item If $f$ is homogeneous of degree $d$, then $\partial_if$ is homogeneous of degree $d-1$.
    \end{enumerate}
\end{corollary}

\begin{proof}
    Left to reader.
\end{proof}

The following theorem gives us an analogy between the divided difference operators and the simple transpositions.
In particular, it tells us that sequences of $\partial_i$'s structurally behave like reduced words when the corresponding sequence of $s_i$'s are reduced words (see Definition \ref{def:ddword} and Theorem \ref{thm:ddreducedwordequiv}), but that the $\partial_i$'s degenerate and collapse to nothing in the case for non-reduced words (see Theorem \ref{thm:ddnonreducedzero}).

\begin{theorem}\label{thm:ddnilcoxeterrelations}
    The divided difference operators satsify the \textit{nil-Coxeter relations}:
    \begin{enumerate}[label=(\alph*)]
        \item The braid relation
            \begin{equation}\label{eq:ddbraid}
                \partial_i\partial_{i+1}\partial_i = \partial_{i+1}\partial_i\partial_{i+1}
            \end{equation}
        \item Far commutativity
            \begin{equation}\label{eq:ddcommute}
                \partial_i\partial_j = \partial_j\partial_i\qquad \text{whenever}\qquad|i-j| > 1
            \end{equation}
        \item Nilpotence
            \begin{equation}\label{eq:ddcontract}
                \partial_i^2 = 0
            \end{equation}
    \end{enumerate}
\end{theorem}

\begin{proof}[Proof (sketch)]
    For (a), without loss of generality we prove the case
    \[
        \partial_1\partial_2\partial_1 = \partial_2\partial_1\partial_2.
    \]
    Which we just have to grind out (see Appendix).

    It turns out that both sides equal
    \[
        \frac{1 - s_1 - s_2 + s_1s_2 + s_2s_1 - s_1s_2s_1}{(x_1-x_2)(x_1-x_3)(x_2-x_3)}.
    \]
    The proofs of (b), (c) are straightforward.
\end{proof}

    Note that the numerator appearing in the proof happens to be
    \[
        \nabla^{-} := \sum_{w\in S_3}(-1)^w w,
    \]
    the antisymmetrizer of $\ZZ[S_3]$.

Given that the definition of $\partial_i$ takes in some $s_i$ as an input, we can naturally come up with a broader definition of $\partial$ that takes in Coxeter words.

\begin{definition}\label{def:ddword}
    Let $w \in S_n$, and let $a = (a_1, \ldots a_k)$ be a Coxeter word for $w$, i.e $k = \ell(w)$ and $s_{a_1}\ldots s_{a_k} = w$. Then define
    \[
        \partial_a := \partial_{a_1}\ldots\partial_{a_k}.
    \]
\end{definition}

We'll actually use this to bootstrap another definition-- divided difference operators parametrized by permutations. That doesn't quite come for free, so we need to first prove the following fact:

\begin{theorem}\label{thm:ddreducedwordequiv}
    Let $w \in S_n$.
    If $a=(a_1,\ldots,a_k)$ and $b=(b_1,\ldots,b_k)$ are reduced words for $w$, then $\partial_a = \partial_b$.
\end{theorem}

\begin{proof}
    This follows from the fact any two reduced words for a permutation $w$ are equivalent modulo far commutativity and the braid relation. 
    Then recall Theorem \ref{thm:ddnilcoxeterrelations}--- Equations \ref{eq:ddbraid} and \ref{eq:ddcommute} tell us exactly that the divided difference operators also satisfy those relations.
\end{proof}

So we can now properly define the following:

\begin{theorem}
    Let $w \in S_n$, and let $a=(a_1,\ldots,a_k)$ be some reduced word for $w$.
    Then define
    \[
        \partial_w := \partial_a = \partial_{a_1}\ldots\partial_{a_k}.
    \]
\end{theorem}

In the case for sequences that \textit{do not} correspond to reduced words, we have the following reason to not really care about them:
\begin{theorem}\label{thm:ddnonreducedzero}
    Let $a=(a_1,\ldots,a_k)$ be a sequence that is not a reduced word for any $w \in S_n$. Then
    \[
        \partial_a = 0.
    \]
\end{theorem}

\begin{proof}
    Because $a$ is not a reduced word, it is possible to do a sequence of moves on the Coxeter word which contains a contraction.
    Mapping these moves over to the divided difference word results in an application of Equation \ref{eq:ddcontract}, killing the whole term.
\end{proof}

\subsection{The definition of a Schubert polynomial}

\begin{definition}
    The \textit{Schubert polynomials} $\frkS_w$ are defined by the rules
    \[
        \begin{cases}
        \frkS_{w_0} := x_1^{n-1}x_2^{n-2}\cdots x_{n-1}^1, \\
        \partial_i\frkS_w := \frkS_{ws_i}
        \end{cases}
    \]
\end{definition}

\subsection{Definition}


\section{Appendix}

\begin{proof}[Detailed proof of Theorem \ref{thm:ddnilcoxeterrelations}] 
    Define $[ij] := x_i - x_j$.
    We have the following relations:
    \begin{align*}
        s_1[12] &= -[12] & s_2[12] &= [13] \\
        s_1[13] &= [23] & s_2[13] &= [12] \\
        s_1[23] &= [13] & s_2[23] &= -[23]
    \end{align*}
    First, we expand the left hand side, which is
    \[
        \partial_1\partial_2\partial_1 = \left(\frac{1 - s_1}{[12]}\right)\left(\frac{1 - s_2}{[23]}\right)\left(\frac{1 - s_1}{[12]}\right).
    \]
    We do the first application, which is the $\partial_2$ hitting the $\partial_1$,
    \begin{align*}
        &\partial_2\partial_1 = \left(\frac{1 - s_2}{[23]}\right) {\color{red} \left(\frac{1 - s_1}{[12]}\right)} \\
        &= \left(\frac{{\color{red}\left(\frac{1 - s_1}{[12]}\right)} - s_2{\color{red}\left(\frac{1 - s_1}{[12]}\right)}}{[23]}\right),
    \end{align*}
    then we apply the $s_2$,
    \begin{align*}
        &= \left(\frac{\left(\frac{1 - s_1}{[12]}\right) - {\color{blue}s_2}\left(\frac{1 - s_1}{[12]}\right)}{[23]}\right) \\
        &= \left(\frac{\frac{1 - s_1}{[12]} - \frac{{\color{blue}s_2} - {\color{blue}s_2}s_1}{{\color{blue} s_2}[12]}}{[23]}\right) \\
        &= \left(\frac{\frac{1 - s_1}{[12]} - \frac{{\color{blue}s_2} - {\color{blue}s_2}s_1}{\color{blue}[13]}}{[23]}\right) \\
        &= \left(\frac{1 - s_1}{[12][23]} - \frac{s_2 - s_2s_1}{[13][23]}\right).
    \end{align*}
    Now we apply $\partial_1$ to our just computed $\partial_2\partial_1$,
    \begin{align*}
        \partial_1(\partial_2\partial_1)
        &= \left(\frac{1 - s_1}{[12]}\right){\color{red}\left(\frac{1 - s_1}{[12][23]} - \frac{s_2 - s_2s_1}{[13][23]}\right)} \\
        &= \left(\frac{\left(\frac{1 - s_1}{[12][23]} - \frac{s_2 - s_2s_1}{[13][23]}\right) - s_1\left(\frac{1 - s_1}{[12][23]} - \frac{s_2 - s_2s_1}{[13][23]}\right)}{[12]}\right) \\
        &= \left(\frac{\frac{1 - s_1}{[12][23]} - \frac{s_2 - s_2s_1}{[13][23]} - \frac{s_1 - s_1s_1}{s_1[12]s_1[23]} + \frac{s_1s_2 - s_1s_2s_1}{s_1[13]s_1[23]}}{[12]}\right) \\
        &= \left(\frac{\frac{1 - s_1}{[12][23]} - \frac{s_2 - s_2s_1}{[13][23]} - \frac{s_1 - 1}{(-[12])[13]} + \frac{s_1s_2 - s_1s_2s_1}{[23][13]}}{[12]}\right) \\
        &= \left(\frac{\frac{1 - s_1}{[12][23]} - \frac{s_2 - s_2s_1}{[13][23]} - \frac{1 - s_1}{[12][13]} + \frac{s_1s_2 - s_1s_2s_1}{[23][13]}}{[12]}\right) \\
        &= \left(\frac{\frac{1-s_1}{[12][23]} - \frac{1-s_1}{[12][13]}}{[12]}\right) +\left(\frac{-\frac{s_2 - s_2s_1}{[13][23]} + \frac{s_1s_2 - s_1s_2s_1}{[23][13]}}{[12]}\right) \\
        &= \left(\frac{1-s_1}{[12]^2[23]} - \frac{1-s_1}{[12]^2[13]}\right) + \left(\frac{-(s_2 - s_2s_1) + s_1s_2 - s_1s_2s_1}{[12][13][23]}\right)\\
        &= \left(\frac{1-s_1}{[12]^2}\left(\frac{1}{[23]}-\frac{1}{[13]}\right)\right) + \left(\frac{-s_2 + s_2s_1 + s_1s_2 - s_1s_2s_1}{[12][13][23]}\right)\\
        &= \left(\frac{1-s_1}{[12]^2}\left(\frac{[13]-[23]}{[23][13]}\right)\right) + \left(\frac{-s_2 + s_2s_1 + s_1s_2 - s_1s_2s_1}{[12][13][23]}\right)\\
        &= \left(\frac{1-s_1}{[12]^2}\left(\frac{[12]}{[23][13]}\right)\right) + \left(\frac{-s_2 + s_2s_1 + s_1s_2 - s_1s_2s_1}{[12][13][23]}\right)\\
        &= \left(\frac{1-s_1}{[12][13][23]}\right) + \left(\frac{-s_2 + s_2s_1 + s_1s_2 - s_1s_2s_1}{[12][13][23]}\right)\\
        &= \frac{1-s_1-s_2+s_2s_1-s_1s_2s_1}{[12][13][23]}    
    \end{align*}
\end{proof}



\begin{thebibliography}{999999999}
    \footnotesize \raggedright
    \bibitem[StanleyEC2]{StanleyEC2}
    Richard P. Stanley, \textit{Enumerative Combinatorics. Volume 2}, Cambridge University Press 2023.

    \bibitem[GrinbergAC]{DarijAC}
    Darij Grinberg, \textit{An Introduction to Algebraic Combinatorics}, \url{http://www.cip.ifi.lmu.de/~grinberg/t/21s/lecs.pdf}

    \bibitem[KnutsonSP]{KnutsonSchubertPolynomials}
    Allen Knutson, \textit{Schubert Polynomials and Symmetric Functions}, \url{https://pi.math.cornell.edu/~allenk/schubnotes.pdf}

    \bibitem[Mac91]{MacdonaldSchubertPolynomials}
    Ian Macdonald, \textit{Notes on Schubert Polynomials}
\end{thebibliography}

\end{document}
