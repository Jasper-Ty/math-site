\documentclass{article}

\usepackage[garamond,nosubsections]{jaspercommon}
\usepackage{centernot}
\DeclareMathOperator{\interior}{int}
\DeclareMathOperator{\closure}{cl}

\title{Topology}
\author{Jasper Ty}
\date{}

\titleauthorhead

\begin{document}

\maketitle

\section*{What is this?}

Here lies an attempt to learn topology from Munkres.
I'm trying to do all exercises and prove all the theorems myself.
I'm also changing notations / conventions to what I like.

\tableofcontents

\setcounter{section}{11}
\section{Topological spaces}

\begin{definition}[Topological spaces]
    A \textit{topology} on a set $X$ is a collection $\scrT$ of subsets of $X$ such that
    \begin{enumerate}[label=(T\arabic*)]
        \item $\varnothing, X \in \scrT$. 
            \label{item:topbtmelems}
        \item $\scrT$ is closed under taking arbitrary unions.
            \label{item:arbunion}
        \item $\scrT$ is closed under taking \textit{finite} intersections.
            \label{item:finintersect}
    \end{enumerate}
    The elements of $\scrT$ are called the \textit{open sets of $X$}.
\end{definition}

\begin{example}[Some finite topologies]
    Consider the three-element set $X=\{a,b,c\}$.
    We will write a subset compactly with a string, so we write $ab$ for $\{a,b\}$.
    The following collections are topologies on $X$:
    \begin{enumerate}[label=(\alph*)]
        \item $\{\varnothing,abc\}$
        \item $\{\varnothing,a,ab,abc\}$
        \item $\{\varnothing,b,ab,bc,abc\}$
        \item $\{\varnothing,b,abc\}$
        \item $\{\varnothing,a,bc,abc\}$
        \item $\{\varnothing,b,c,ab,bc,abc\}$
        \item $\{\varnothing,ab,abc\}$
        \item $\{\varnothing,a,b,ab,abc\}$
        \item $\{\varnothing,a,b,c,ab,bc,ac,abc\}$
    \end{enumerate}
\end{example}

\begin{example}[Discrete and indiscrete topology]
    Let $X$ be a set.
    Then $\mathscr{P}(X)$ is a topology on $X$, called the \textit{discrete topology}.

    The set $\{X, \varnothing\}$ is also a topology on $X$, called the \textit{indiscrete topology}.
\end{example}

\begin{example}[Cofinite topology]
    Let $X$ be a set.
    Let $\scrT_f$ be the collection of all subsets $U$ of $X$ such that $X \setminus U$ is either finite or $X$ itself.
    Then $\scrT_f$ is called the \textit{cofinite topology} on $X$.
\end{example}

\begin{proof}[Proof (the cofinite topology is a topology)]
    $X \setminus X$ is finite, namely $\varnothing$, and $X \setminus \varnothing$ is $X$, so we have that $X, \varnothing \in \scrT_f$.
    Then $\scrT_f$ satisfies condition \ref{item:topbtmelems}.

    Let $\{U_\alpha\}$ be a subcollection of $\scrT_f$.
    It's a set theoretic identity that
    \[
        X \setminus \bigcup_\alpha U_\alpha = \bigcap_\alpha (X \setminus U_\alpha).
    \]
    Then it's clear that the complement of $\bigcup_\alpha U_\alpha$ is either finite (some $X \setminus U_\alpha$ is finite) or equal to $X$ (none of the $X \setminus U_\alpha$ are finite--- they all equal $X$).
    So, $\scrT_f$ satisfies condition \ref{item:arbunion}.


    Let $U_\alpha$ and $U_\beta$ be two arbitrary elements of $\scrT_f$.
    The relevant computation here is 
    \[
        X \setminus (U_\alpha \cap U_\beta) =  (X \setminus U_\alpha) \cap (X \setminus U_\beta).
    \]
    And the same casework applies: if both $X \setminus U_\alpha$ and $X \setminus U_\beta$ are $X$, then the left hand side is $X$. 
    If either of them are finite, then the left hand side is finite.
    Then, $\scrT_f$ satisfies condition \ref{item:finintersect}.
\end{proof}

\begin{convention}
    That a set is \textit{countable} means that it's either finite or countably infinite.
\end{convention}

\begin{example}[Cocountable topology]\label{ex:cocountabletop}
    Let $X$ be a set.
    Let $\scrT_c$ be the collection of all subsets $U$ of $X$ such that $X \setminus U$ is either \textit{countable} or $X$ itself.
    Then $\scrT_c$ is called the \textit{cocountable topology} on $X$.
\end{example}

\begin{proof}[Proof (the cocountable topology is a topology)] \
    This is proven the same way, using the same computations.

    The ``juice'' for both proofs is that the subset relationship respects the total order defined by cardinality, so a subset of a countable (resp. finite) set can never be more than countable (resp. finite).
    Combine this with the fact that the intersection of a collection of sets is always a subset of any of the individual sets in the collection.
\end{proof}

\begin{definition}[Finer and coarser topologies]
    Let $\scrT$ and $\scrT'$ be two topologies on a set $X$.
    We say that $\scrT$ is \textit{coarser} then $\scrT'$ when $\scrT \subset \scrT'$.
    $\scrT$ is \textit{finer} than $\scrT'$ when $\scrT \supset \scrT'$.
    We say that two topologies $\scrT$ and $\scrT'$ are \textit{comparable} whenever one is finer or coarser than the other.
\end{definition}

\begin{example}
    For all sets $X$, the discrete topology on $X$ is finer than the indiscrete topology on $X$.
    Conversely, the indiscrete topology is always coarser than the discrete topology.

    In fact, the discrete topology on $X$ is finer than \textit{any} other topology on $X$, and the indiscrete topology is coarser than \textit{any other} topology on $X$.
    In this sense, the former is also defined as the \textit{finest} topology on $X$, and the latter the coarsest.
\end{example}

\begin{example}
    The cocountable topology is finer than the cofinite topology.
\end{example}

\section{Basis for a topology}

\begin{definition}[Basis for a topology]
    Let $X$ be a set.
A \textit{basis for a topology} on $X$ is a collection $\scrB$ such that
    \begin{enumerate}[label=(B\arabic*)]
        \item For all $x \in X$, there exists $B \in \scrB$ such that $x \in B$. 
            \label{basis:cover}
        \item If $x \in B_1 \cap B_2$ for some $B_1, B_2 \in \scrB$, then there exists $B_3 \in \scrB$ such that $x \in B_3$ and $B_3 \subset B_1 \cap B_2$. 
            \label{basis:intersect}
    \end{enumerate}
    We refer to the sets $B \in \scrB$ as \textit{basis elements}.

    The \textit{topology $\scrT$ generated by $\scrB$} is defined by letting the open sets $U$ be the sets such that, for all $x \in U$, there exists a basis element $B \in \scrB$ such that $x \in B$ and $B \subset U$.
\end{definition}

\begin{proof}[Proof (the topology generated by a basis is a topology)]
    Let $\scrT$ be the topology generated by $\scrB$, as in the above definition.

    $X \in \scrT$, almost directly from \ref{basis:cover}, since we note that $B \subseteq X$ for all $B \in \scrB$.
    Vacuously, $\varnothing \in \scrT$.
    Then we've shown that $\scrT$ satisfies \ref{item:topbtmelems}.

    Let $\{U_\alpha\}$ be some subcollection of $\scrT$.
    If $x \in \bigcup_\alpha U_\alpha$, then it is in $U_\alpha$ for some $\alpha$.
    Then there exists some $B \in \scrB$ such that $x \in B$ and $B \subseteq U_\alpha$.
    Noting that $U_\alpha \subseteq \bigcup_\alpha U_\alpha$ tells us that $B \subseteq \bigcup_\alpha U_\alpha$.
    Since we picked an arbitrary $x$ from $\bigcup_\alpha U_\alpha$, we've shown that $\bigcup_\alpha U_\alpha \in \scrT$.
    Then $\scrT$ satisfies \ref{item:arbunion}.

    Finally, let $U_1, U_2 \in \scrT$.
    Pick some $x \in U_1 \cap U_2$.
    Then we can take basis elements $B_1$ and $B_2$ such that $x \in B_1$, $x \in B_2$, $B_1 \subseteq U_1$, $B_2 \subseteq B_2$.
    Use \ref{basis:intersect} now to produce a $B_3$.
    We will have that $x \in B_3$, and that $B_3 \subseteq B_1 \cap B_2$.
    Since $B_1 \cap B_2 \subseteq U_1 \cap U_2$, we have that $B_3 \subseteq U_1 \cap U_2$.
    This shows \ref{item:finintersect}.
\end{proof}

\begin{example}\label{ex:openballsopenrectsbasis}
    Consider the plane, $\RR^2$.
    The set of all open balls 
    \[
        \{(x,y): (x-c)^2 + (y-c)^2 < r^2\}
    \]
    form a basis.

    Similarly, the set of all open rectangles
    \[
        \{(x,y): x_0 < x < x_1, y_0 < y < y_1\}
    \]
    form a basis.
\end{example}

\begin{figure}
    \centering
    \begin{tikzpicture}
        \fill (-0.2,0.2) circle (1pt) node[right] {$x$};
        \draw[dashed] (0,0) circle (0.5cm) node[below] {$B_3$};
        \draw[dashed] (-0.5,-1) circle (1.7cm) node[below left] {$B_1$};
        \draw[dashed] (1,0) circle (1.6cm) node[above right] {$B_2$};
    \end{tikzpicture}
    \caption{Pictorial explanation of why open balls satisfy \ref{basis:intersect}.}
    \label{fig:openballb2}
\end{figure}
\begin{figure}
    \centering
    \begin{tikzpicture}
        \fill (0,0) circle (1pt) node[right] {$x$};
        \draw[dashed] (-0.5,-0.5) rectangle ++(1,1)
            node[right] {$B_3$};
        \draw[dashed] (-1,-1) rectangle ++(4,3.5)
            node[right] {$B_1$};
        \draw[dashed] (-3,-1.5) rectangle ++(5,3)
            node[right] {$B_2$};
    \end{tikzpicture}
    \caption{Pictorial explanation of why open rectangles satisfy \ref{basis:intersect}.}
\end{figure}

\begin{proof}
    Fix some $\varepsilon > 0$.
    Every point $x$ on the plane belongs to some open ball of radius $\varepsilon$ centered at $x$.
    Hence open balls satisfy \ref{basis:cover}.
    For \ref{basis:intersect}, consider Figure \ref{fig:openballb2}.

\end{proof}

\begin{example}
    The set of all one-element subsets of a set $X$ is a basis for the discrete topology on $X$.
\end{example}


\begin{lemma}
    The topology generated by a basis is the same as the collection of all possible unions of basis elements.
\end{lemma}

\begin{proof}
    Fix $\scrB$ to be a basis for a topology $\scrT$.

    If $U \in \scrT$, we can assign to each $x \in U$ a basis element $B_x \in \scrB$. Evidently,
    \[
        \bigcup_{x \in U} B_x = U.
    \]
    So every open set in $\scrT$ is some union of basis elements.

    Conversely, let $U$ be some union of basis elements $U_\alpha B_\alpha$.
    This is obviously an open set, since $x \in U$ if and only if $x$ came from some $B_\alpha$, and every such $B_\alpha$ will be a subset of $U$.
\end{proof}

\begin{lemma}\label{lem:basisfromsubcollection}
    Let $X$ be a topological space.
    Suppose $\scrC$ is a collection of open sets in $X$ such that for all open sets $U$, there exists a $C \in \scrC$ for each $x \in U$ such that $x \in C$ and $C \subseteq U$.
    Then $\scrC$ is a basis for the topology on $X$.
\end{lemma}

\begin{proof}
    $\scrC$ automatically satisfies \ref{basis:cover} when we take $U$ in the definition above to be $X$ (recall \ref{item:topbtmelems}).

    For \ref{basis:intersect}, we use the fact that $\scrC$ is already made up of open sets.
    Let $C_1, C_2 \in \scrC$, and let $x \in C_1 \cap C_2$.
    We have that $C_1 \cap C_2$ is an open set (recall \ref{item:finintersect}), hence we can reuse the above definition to produce $C_3 \in \scrC$, which will satisfy \ref{basis:intersect}.
\end{proof}

\begin{lemma}\label{lem:finerbasis}
    Let $\scrT$ and $\scrT'$ be topologies on $X$, generated respectively by $\scrB$ and $\scrB'$.
    Then the following are equivalent:
    \begin{enumerate}[label=(\arabic*)]
        \item $\scrT'$ is finer than $\scrT$. \label{basisfiner1}
        \item For each $x \in X$ and each basis element $B \in \scrB$, there exists $B' \in \scrB'$ such that $x \in B'$ and $B' \subseteq B$. \label{basisfiner2}
    \end{enumerate}
\end{lemma}

\begin{proof}
    Suppose \ref{basisfiner2}.
    It suffices to prove that every open set of $\scrT$ is also an open set of $\scrT'$.
    This much is clear, because if $U = \bigcup_\alpha B_\alpha$, where all $B_\alpha \in \scrB$, then every $x \in U$ belongs to some $B_\alpha$, and therefore to some $B'_x \in \scrB'$ as well, where $B'_x \subset B_\alpha$.
    Then
    \[
        U = \bigcup_{x \in U} B'_x.
    \]
    This proves that \ref{basisfiner2} implies \ref{basisfiner1}.

    For the other direction, choose some pair $x$ and $B \in \scrB$ such that $x \in \scrB$.
    $B$ is already open in $\scrT$, and therefore it's also open in $\scrT'$, since $\scrT'$ is finer than $\scrT$.
    Then, there must exist some $B' \in \scrB$ such that $x \in B'$ and $B' \subseteq B$.
    This proves that \ref{basisfiner1} implies \ref{basisfiner2}.
\end{proof}

\begin{example}
    The bases in Example \ref{ex:openballsopenrectsbasis}, that of open balls and open rectangles in the plane, generate the same topology.
\end{example}

\begin{proof}
    For any open ball and a point inside it, we can fit an open rectangle containing the point.

    Likewise, for any open rectangle and a point inside it, we can fit an open ball containing the point.

    Then the topologies generated by both are finer than each other.
    By antisymmetry, they must be the same topology.
\end{proof}

\begin{definition}
    Consider the real line $\RR$.
    The \textit{standard topology} on the real line is the topology generated by the basis of open intervals; sets of the form
    \[
        (a,b) \coloneq \{x: a < x < b\}.
    \]
    The \textit{lower limit topology} on the real line is the topology generated by the basis of sets of the form
    \[
        [a,b) \coloneq \{x: a \leq x < b\}.
    \]
    Let \textit{$K$-topology} on the real line is the topology generated by the basis of open intervals, along with the sets 
    \[
        (a,b) \setminus \left\{1, \frac{1}{2}, \frac{1}{3}, \ldots\right\}
    \]
    for all open intervals $(a,b)$.

    We will refer to $\RR$ equipped with the standard, lower limit, and the $K$ topologies by $\RR$, $\RR_\ell$, and $\RR_K$ respectively.
\end{definition}

\begin{lemma}
    $\RR_\ell$ and $\RR_K$ are strictly finer than the standard topology, but are incomparable.
\end{lemma}

\begin{proof}
    That $\RR_K$ is finer than the standard topology is obvious.
    That it is strictly finer follows from the fact that we cannot produce the set $E = (-1,1) \setminus \{1, 1/2, 1/3, \ldots\}$ with a union of open intervals.

    We would have an open interval $(a,b) \subseteq E$ containing $0$, but that's impossible, since we would be able to produce $n$ such that $0 < 1/n < b$,\footnote{this follows from the archimedean property of the reals} so that $1/n \in (a,b)$, hence $1/n \in E$.

    But $1/n \in K$, which is a contradiction to the fact that $E$ contains no points of $K$.

    $\RR_\ell$ is finer than the standard topology, since
\end{proof}

\begin{definition}[Subbasis for a topology]
    A \textit{subbasis} for a topology on $X$ is a collection $\scrS$ of subsets of $X$ whose only constraint is that they must cover $X$.

    The \textit{topology generated by the subbasis} $\scrS$ is defined to be the collection of all possible unions of all possible finite intersections of elements of $\scrS$.
\end{definition}

\begin{proof}[Proof (the topology generated by a subbasis is a topology)]
    Let $\widetilde{\scrS}$ be the collection of all possible finite intersections of elements of $S$.
    We will show that $\widetilde{\scrS}$ is a basis, which will prove that the topology generated by $\scrS$ is a topology.

    \ref{basis:cover} is automatically satisfied, since $\scrS$ covers $X$, and the individual elements of $\scrS$ are in $\widetilde{\scrS}$.

    \ref{basis:intersect} is also easily satisfied, since for any two $S_1, S_2 \in \widetilde{\scrS}$, $S_1 \cap S_2 \in \widetilde{\scrS}$ as well, since a finite intersection of finite intersections remains finite.
\end{proof}

\section*{Exercises}

\begin{exercise}
    Let $X$ be a topological space; let $A$ be a subset of $X$.
    Suppose that for each $x \in A$ there is an open set $U$ containing $x$ such that $U \subset A$.
    Show that $A$ is open in $X$.
\end{exercise}

Assign each $x \in A$ an open set $U_x$ such that $U_x \subset A$.
Then 
\[
    \bigcup_{t\in A} U_t \subset A.
\]
Since $x \in U_x \subset \bigcup_{t \in A} U_t \subset A$ for \textit{all} $x \in A$, we also have that
\[
    A \subset \bigcup_{t \in A} U_t.
\]
Then $A$ is a union of open sets in $X$, namely all the $U_x$.

\begin{exercise}
    Consider the nine topologies on the set $X = \{a,b,c\}$ indicated in Example 1 of §12.
    Compare them; that is, for each pair of topologies, determine whether they are comparable, and if so, which is the finer.
\end{exercise}

The Hasse diagram of the nine topologies, ordered by containment.

\begin{center}
    \begin{tikzpicture}[scale=1.3]
    \node (abcabbcacabc) at (-1,4) {$\{abc,ab,bc,ac,a,b,c,\varnothing\}$};
    \node (abcabab) at (1,3) {$\{abc, ab, a,b,\varnothing\}$};
    \node (abcabbcb) at (-3,2) {$\{abc, ab, bc, b,\varnothing\}$};
    \node (abcabbcbc) at (-3,3) {$\{abc, ab, bc, b,c,\varnothing\}$};
    \node (abcabc) at (-1,2) {$\{abc, ab, c,\varnothing\}$};
    \node (abcaba) at (1,2) {$\{abc, ab, a,\varnothing\}$};
    \node (abcab) at (-1,1) {$\{abc, ab,\varnothing\}$};
    \node (abcb) at (-3,1) {$\{abc, b,\varnothing\}$};
    \node (abc) at (-1,0) {$\{abc,\varnothing\}$};
    \draw (abc) -- (abcab);
    \draw (abc) -- (abcb);
    \draw (abcab) -- (abcaba);
    \draw (abcab) -- (abcabbcb);
    \draw (abcb) -- (abcabbcb);
    \draw (abcab) -- (abcabc);
    \draw (abcaba) -- (abcabab);
    \draw (abcabbcb) -- (abcabbcbc);
    \draw (abcabbcbc) -- (abcabbcacabc);
    \draw (abcabab) -- (abcabbcacabc);
    \draw (abcabc) -- (abcabbcbc);
    \end{tikzpicture}.
\end{center}

\begin{exercise}
    Show that the collection $\scrT_c$ given in Example \ref{ex:cocountabletop} is a topology on the set $X$.
    Is the collection
    \[
        \scrT_\infty = \{U : X \setminus U \text{ is infinite or empty or all of } X\}
    \]
    a topology on $X$?
\end{exercise}

I already proved $\scrT_c$ is a topology, and for reasons outlined the collection above is \textit{not} a topology.

This is because the union and intersection of two infinite sets may be finite--- having the condition cover all smaller cardinalities is crucial.

Consider $\scrT_\infty$ on $\NN = \{1,2,3,\ldots\}$.
We have that $\{n\} \in \scrT_\infty$ for all $n$.
However, the set $\{2,3,4,\ldots\}$, despite being the union $\bigcup_{n>1}\{n\}$, is evidently not in $\scrT_\infty$, since its complement in $\NN$ is $\{1\}$, a set that is neither infinite, empty, nor all of $\NN$.
Similarly, $\{2,3,4,\ldots\}$ can be written as the intersection of the two sets $\{2,4,6,\ldots\}$ and $\{3,5,7,\ldots\}$, which are both in $\scrT_\infty$.

\begin{exercise}\
    \begin{enumerate}[label=(\alph*)]
        \item 
            If $\{\scrT_\alpha\}$ is a family of topologies on $X$, show that $\bigcap_\alpha \scrT_\alpha$ is a topology on $X$.
            Is $\bigcup_\alpha \scrT_\alpha$ a topology on $X$?
        \item
            Let $\{\scrT_\alpha\}$ be a family of topologies on $X$.
            Show that there is a unique smallest topology on $X$ containing all the collections $\scrT_\alpha$, and a unique largest topology contained in all $\scrT_\alpha$.
        \item 
            If $X = \{a,b,c\}$, let
            \[
                \scrT_1 = \{\varnothing, X, \{a\},\{a,b\}\} \qquad \text{and} \qquad \scrT_2 = \{\varnothing, X, \{a\}, \{b, c\}\}.
            \]
            Find the smallest topology containing $\scrT_1$ and $\scrT_2$, and the largest topology contained in $\scrT_1$ and $\scrT_2$.
    \end{enumerate}
\end{exercise}

\begin{enumerate}[label=(\alph*)]
    \item
        Put $\scrT := \bigcap_\alpha \scrT_\alpha$.
        $X, \varnothing \in \scrT_\alpha$ for all $\alpha$, hence they are in $\scrT$.

        Similarly, for an arbitrary union of open sets $U_\mu$ in $\scrT$, we have that each $U_\mu$ is in each $\scrT_\alpha$, hence the union $\bigcup_\mu U_\mu$ is in $\scrT_\alpha$ for all $\alpha$.
        Then $\bigcup_\mu U_\mu \in \scrT$.

        Similar reasoning goes for finite intersections.

        $\bigcup_\alpha \scrT_\alpha$ is \textit{not necessarily} a topology on $X$.

        Consider $X = \{a,b,c\}$.
        The collections $\scrT_1 = \{\varnothing,a,abc\}$ and $\scrT_2 = \{\varnothing,b,abc\}$ are topologies on $X$, however their union is not a topology, since it does not contain $a \cup b = ab$.

        \item
            The unique largest topology contained in all $T_\alpha$ is $\bigcap_\alpha \scrT_\alpha$.
            It is evidently contained in all $\scrT_\alpha$, and we just proved that it's a topology.
            Suppose we had another topology contained in all $T_\alpha$, then its open sets must be contained in all $\scrT_\alpha$, and so are also open sets of $\bigcap_\alpha \scrT_\alpha$, hence it is coarser than $\scrT_\alpha$.
            This shows that $\bigcap_\alpha \scrT_\alpha$ is the largest such topology.

            The unique smallest topology on $X$ containing all the $\scrT_\alpha$ is the topology generated by $\bigcup_\alpha \scrT_\alpha$ viewed as a subbasis.
            Suppose another topology contains all the $\scrT_\alpha$, then it certainly contains all possible unions of finite intersections of elements of $\bigcup_\alpha \scrT_\alpha$, meaning that it is finer than our unique smallest topology.
            This shows that it is in fact the smallest.
        \item
            In light of the answer above, we have that the smallest topology containing both $\scrT_1$ and $\scrT_2$ is
            \[
                \{\varnothing, a, b, ab, bc, abc\},
            \]
            and the unique largest topology contained in both $\scrT$ and $\scrT_2$ is
            \[
                \{\varnothing,a,abc\}.
            \]
\end{enumerate}

\begin{exercise}
    Show that if $\scrA$ is a basis for a topology on $X$, then the topology generated by $\scrA$ equals the intersection of all topologies on $X$ that contain $\scrA$.
    Prove the same if $\scrA$ is a subbasis.
\end{exercise}

\begin{proof}
    Let $\scrA$ be a basis for a topology on $X$.
    Let $\scrT$ be the topology generated by $\scrA$.
    Let $\{\scrT_\alpha\}$ be the collection of all topologies that contain $\scrA$.
    The claim is that
    \[
        \scrT = \bigcap_\alpha \scrT_\alpha.
    \]
    Let $U \in \scrT$.
    Then $U = \bigcup_\mu U_\mu$ where $U_\mu \in \scrA$ for all $\mu$.
    But we have $U_\mu \in \scrT_\alpha$ for all $\alpha$ also, since all the $\scrT_\alpha$ contain $\scrA$.
    Then $\bigcup_\mu U_\mu \in \scrT_\alpha$, and hence $U \in \scrT_\alpha$, for all $\scrT_\alpha$.
    This shows that
    \[
        \scrT \subseteq \bigcap_\alpha \scrT_\alpha.
    \]
    Since $\scrT$ is a topology containing $\scrA$, $\scrT \in \{\scrT_\alpha\}$.
    Then
    \[
        \bigcap_\alpha \scrT_\alpha \subseteq \scrT,
    \]
    which proves equality.

    Now let $\scrS$ be a subbasis for a topology on $X$.
    Let $\scrT$ be the topology generated by $\scrS$.
    Let $\{\scrT_\alpha\}$ be the collection of all topologies that contain $\scrS$.
    Let $U \in \scrT$.
    Then $U$ is a union of finite intersections of sets in $\scrS$.
    But since $\scrS \subset \scrT_\alpha$ for any $\alpha$, it must be that $U \in \scrT_\alpha$ for any $\alpha$ as well, since topologies are closed under unions of finite intersections.
    Then
    \[
        \scrT \subseteq \bigcap_\alpha \scrT_\alpha
    \]
    again, which proves the statement.
\end{proof}

\begin{exercise}
    Show that the topologies of $\RR_\ell$ and $\RR_K$ are not comparable.
\end{exercise}

You cannot produce the set $(-1, 1) \setminus K$ by unions of $[a,b)$, hence $\RR_K$ contains an open set not in $\RR_\ell$.

Similarly, you cannot produce the set 
\[
    \left(-1, -\frac{1}{2}\right) 
    \cup \left(-\frac{1}{2}, -\frac{1}{3}\right) 
    \cup \left(-1, -\frac{1}{4}\right) 
    \cup \cdots
    \cup [0, 1)
\]
by unions of $(a,b) - K$, for the same reason that the standard topology doesn't contain the set $(-1,1) \setminus K$. Then, $\RR_\ell$ contains an open set not in $\RR_K$.

\begin{exercise}
    Consider the following topologies on $\RR$:
    \begin{align*}
        \scrT_1 &= \text{the standard topology}, \\
        \scrT_2 &= \text{the topology of }\RR_K, \\
        \scrT_3 &= \text{the finite complement topology}, \\
        \scrT_4 &= \text{the upper limit topology, having all sets }(a,b]\text{ as basis}, \\
        \scrT_5 &= \text{the topology having all sets }(-\infty,a)=\{x:x<a\}\text{ as basis}.
    \end{align*}
    Determine, for each of these topologies, which of the others it contains.
\end{exercise}

We have that $\scrT_5 \subset \scrT_1 = \scrT_3 \subset \scrT_2 \subset \scrT_4$, and  $\scrT_3 \subset \scrT_1$.

\begin{itemize}
    \item[$\scrT_5 \subset \scrT_1$]
        Use Lemma \ref{lem:finerbasis}: for any $x \in (-\infty, a)$, we can find an open interval containing $x$ contained in $(-\infty, a)$, namely $(x-\delta, a)$ for some $\delta > 0$.

        The converse relation does not hold, since you cannot produce the set $(0,1)$ by \textit{any} union of rays $(-\infty, a)$--- any such union is either yet another ray or the entirety of $\RR$.
        This follows from the least upper bound property, since if the set of right endpoints have an upper bound $M$, then the union is $(-\infty, M)$, and if not, the union is $\RR$.
    \item[$\scrT_1 \subset \scrT_2$] This was already proven.
    \item[$\scrT_2 \subset \scrT_4$]
        We know that we can write any open interval $(a,b)$ as a union of half-open intervals: 
        \[
            (a,b) = \bigcup_{k=0}\left(a,b-\frac{1}{k}\right].
        \]
        So, we can conservatively add all the open intervals to our basis without changing the generated topology.
        Then, using Lemma \ref{lem:finerbasis} again, we show that we can write the set $(a,b) \setminus K$ as follows:
        \[
            (a,b) \setminus K = (a, 0] \cup \left(\bigcup_{n=1}^\infty \left(\frac{1}{n+1},\frac{1}{n}\right)\right) \cup \left(\frac{1}{n}, b\right).
        \]
        The converse relation does not hold, since you cannot express the set $(0,1]$ as a union of $(a,b) \setminus K$.
    \item[$\scrT_3 \subset \scrT_1$]
        Take some open set of $\scrT_3$, namely, $\RR \setminus E$ where $E$ is some finite subset of $\RR$.
        If $E$ is empty, then $\RR \setminus K = \RR$ is already always open.
        Suppose $E$ is nonempty, and put $E = \{x_0, x_1, \ldots, x_n\}$.
        Then
        \[
            \RR \setminus E = (-\infty, x_0) \cup \bigcup_{k=0}^{n-1}(x_k, x_{k+1}) \cup (x_n, \infty).
        \]
        Then $\scrT_3 \subseteq \scrT_1$.
        The converse does not hold, since you cannot write $(0, 1)$ as a union of cofinite subsets of the real line--- such a union will also again be cofinite, but $(0,1)$ is not.
\end{itemize}

\begin{exercise}\
    \begin{enumerate}[label=(\alph*)]
        \item 
            Apply Lemma \ref{lem:basisfromsubcollection} to show that the countable collection
            \[
                \scrB = \{(a,b): a<b,a\text{ and }b\text{ rational}\}
            \]
            is a basis that generates the standard topology on $\RR$.
        \item 
            Show that the collection
            \[
                \scrC = \{[a,b): a<b,a\text{ and }b\text{ rational}\}
            \]
            is a basis that generates a topology different from the lower limit topology on $\RR$.
    \end{enumerate}
\end{exercise}

Let $U$ be an open set in $\RR$.
Let $x \in U$.
Then $x$ is contained in some open interval $(a,b) \subseteq U$.

Between every two real numbers there exists a rational number.
Then, we may produce rational points $p,q$ such that $a<p<x<q<b$.
Then $x \in (p,q) \subset (a,b) \subseteq U$.

By Lemma \ref{lem:basisfromsubcollection}, this means that the set of all open intervals with rational endpoints forms a basis for the standard topology on $\RR$.

For (b), consider the interval $[\sqrt{2}, 2)$.
$\sqrt{2} \in [\sqrt{2},2)$, but we cannot produce $[a,b) \subseteq [\sqrt{2},2)$ with rational endpoints such that $\sqrt{2} \in [a,b)$, since necessarily $a > \sqrt{2}$ by $a$ being rational.

Then the lower limit topology contains an set \textit{not} in the topology generated by the basis in the question.

\section{The order topology}

\begin{definition}
    Let $X$ be a totally ordered set.
    Suppose that $X$ has more than one element.
    Let $\scrB$ be the collection of all sets of the following types
    \begin{enumerate}[label=(\arabic*)]
        \item All open intervals $(a,b)$ in $X$.
        \item All intervals of the form $[a_0, b)$, where $a_0$ is the least element (if any) of $X$.
        \item All intervals of the form $(a, b_0]$, where $b_0$ is the greatest element (if any) of $X$.
    \end{enumerate}

    Then $\scrB$ is a basis for what is called the \textit{order topology} on $x$.
\end{definition}

\begin{proof}[Proof (order topology is topology)]
    Let $X$ be a totally ordered set with more than one element, and define $\scrB$ as above.

    Let $x \in X$.
    Suppose $x$ is a least element of $X$, then $x \in [x,b)$ for some $b \in X$, where $b \neq x$.
    Similarly, if $x$ is a greatest element of $X$, then $x \in (a, x]$ for some $a \in X$, where $a \neq x$.
    Now suppose $x$ is not a least or greatest element of $X$.
    This implies that there are elements $u,v$ such that $u < x < v$, in which case $x \in (u,v)$.

    This shows that $\scrB$ as defined satisfies \ref{basis:cover}.

    Now let $B_1$ and $B_2$ be two basis elements.
    Let $x \in B_1 \cap B_2$.

    If $B_1 = (a,b)$ and $B_2 = (c,d)$, then it must be that $a<x<b$ and $c<x<d$.
    Put $u = \max(a,c)$ and $v = \min(b,d)$.
    Then $a,c \leq u < x < v \leq b,d$. 
    From that we can read off that $x \in (u,v)$, $(u,v) \subseteq (a,c)$, and $(u,v) \subseteq (b,d)$.
    So put $B_3 = (u,v)$.

    There is extra casework in the case that $B_1$ or $B_2$ are of the form $[a, b)$ or $(a, b]$.

\end{proof}

\begin{example}
    The standard topology on $\RR$ is the order topology on $(\RR, \geq)$.
\end{example}

\begin{example}
    Take $\RR \times \RR$ with dictionary order.
    The order topology on $\RR \times \RR$ is the topology generated by the basis sets of the form 
    \[
        \Big((a,b), (c,d)\Big)
    \]
    where $(a,b) < (c,d)$ in dictionary order, which means that either $a<c$, or $a=c$ and $b<d$.

    The type of basis set which contain greatest and least elements are not a consideration here, since $\RR \times \RR$ with dictionary order has no greatest nor least element.
\end{example}

\begin{example}
    The order topology on the positive integers $\ZZ_+$ is the discrete topology, since we have as basis sets all singletons $\{n\}$---
    for $n > 1$, we take $(n-1,n+1)$, and for $n=1$, we take $[1,2)$.
\end{example}

\begin{example}
    The set $\{1,2\} \times \ZZ$ with the dictionary order has a least element--- $\{1,1\}$.
    Denote the elements of the form $\{1, a\}$ simply by $a$, and elements of the form $\{2, a\}$ by $a^\ast$.

    Then $1 < 2 < 3 < \cdots < 1^\ast < 2^\ast < 3^\ast < \cdots$.

    The order topology on this set is \textit{not} the discrete topology, since it does not contain the singleton $\{1^\ast\}$--- for all $x < 1^\ast$, there always exists $y$ such that $x < y < 1^\ast$ (consider $x+1$).

    Hence, any open interval $(a, b^\ast)$ that contains $1^\ast$ must necessarily contain an element $x$ such that $a < x < 1^\ast$.
\end{example}

\begin{definition}
    Let $X$ be a totally ordered set.
    The \textit{open rays} are all sets of the form
    \begin{align*}
        (a, +\infty) &\coloneq \{x: x > a\}, \\
        (-\infty, a) &\coloneq \{x: x < a\}.
    \end{align*}

    These sets form a \textit{subbasis} for the order topology on $X$.
\end{definition}

\begin{proof}
    \begin{enumerate}[label=(\arabic*)]
        \item $(a,b) = (a, +\infty) \cap (-\infty, b)$
        \item $[a_0, b) = (-\infty, b)$
        \item $(a, b_0] = (a, +\infty)$
    \end{enumerate}
\end{proof}

\section{The ``product'' topology}

\begin{convention}
    For two collections $\scrA$ and $\scrB$, we define
    \[
        \scrA \otimes \scrB \coloneq \{A \times B: A \in \scrA, B \in \scrB\}.
    \]
\end{convention}

\begin{definition}\label{def:producttopology}
    Let $(X, \scrT_X)$ and $(Y, \scrT_Y)$ be topological spaces.
    The \textit{product topology} is defined to be topology generated by the basis $\scrT_X \otimes \scrT_Y$, that is, all sets $U \times V$, where $U \in \scrT_X$ and $V \in \scrT_Y$.
\end{definition}

\begin{proof}[Proof (box topology is topology)]
    Take some point $(x,y) \in X \times Y$, then \ref{basis:cover} is automatically satisfied, since $X \times Y \in \scrT_X \otimes \scrT_Y$.
    
    For \ref{basis:intersect}, take $U_1 \times V_1 \in \scrT_X \otimes \scrT_Y$ and $U_2 \times V_2 \in \scrT_X \otimes \scrT_Y$, and let $x \in (U_1 \times V_1) \cap (U_2 \times V_2)$.
    Since $U_1, U_2 \in \scrT_X$, $U_1 \cap U_2 \in \scrT_X$, and similarly $V_1 \cap V_2 \in \scrT_Y$.
    Then $(U_1 \cap U_2) \times (V_1 \cap V_2) \in \scrT_X \otimes \scrT_Y$.
    It's an easy set-theoretic fact that
    \[
        (U_1 \times V_1) \cap (U_2 \times V_2) = (U_1 \cap U_2) \times (V_1 \cap V_2).
    \]
    From which \ref{basis:intersect} easily follows--- $x$ is a member of the LHS, so it is therefore a member of the RHS.
    Moreover, the RHS is a subset of the LHS.

    To prove the identity, the LHS consists of all elements of the form $(u,v)$ where $(u \in U_1 \wedge v \in V_1) \wedge (u \in U_2 \wedge v \in V_2)$, and the RHS consists of all elements of the form $(u,v)$ where $(u \in U_1 \wedge u \in U_2) \wedge (v \in V_1 \wedge v \in V_2)$.
    Then the identity follows from properties of $\wedge$.
\end{proof}

\begin{theorem}\label{thm:basisforproduct}
    Let $(X, \scrT_X)$ and $(Y, \scrT_Y)$ be topological spaces, and let $(X \times Y, \scrT_{X \times Y})$ be their topological product.
    Let $\scrB_X$ and $\scrB_Y$ be bases for $\scrT_X$ and $\scrT_Y$ respectively.
    Then, $\scrB_X \otimes \scrB_Y$ is a basis for $(X \times Y, \scrT_{X \times Y})$.
\end{theorem}

\begin{proof}
    Each member of $\scrB_X \times \scrB_Y$ is open, since if $B_X \in \scrB_X$ and $B_Y \in \scrB_Y$, we have that $B_X \in \scrT_X$ and $B_Y \in \scrT_Y$, and so $B_X \times B_Y \in \scrT_X \otimes \scrT_Y$, which is by definition the basis for $\scrT_{X \times Y}$. 
    Then we can use Lemma \ref{lem:basisfromsubcollection}.

    Let $U \in \scrT_{X \times Y}$.
    Let $(x,y) \in U$.
    Then $(x,y) \in U_X \times U_Y$ for some $U_X \times U_Y \in \scrT_X \otimes \scrT_Y$. Since $\scrB_X$ and $\scrB_Y$ are bases for $\scrT_X$ and $\scrT_Y$, we now produce sets $B_X$ and $B_Y$ such that $x \in B_X$, $y \in B_Y$, and $B_X \subseteq U_X$, $B_Y \subseteq U_Y$.

    Then
    \[
        (x,y) \in B_X \times B_Y \subseteq U_X \times U_Y \subseteq U.
    \]
    Since the conditions of Lemma \ref{lem:basisfromsubcollection} hold, we have shown that $\scrB_X \otimes \scrB_Y$ is a basis for $\scrT_{X \times Y}$.
\end{proof}

\begin{example}
    The standard topology on $\RR^2 \coloneq \RR \times \RR$ is the product of the standard topologies.

    By the previous theorem, this means that we have as a basis for this topology all sets of the form $(a,b) \times (c,d)$.

    Namely, this topology is the topology generated by all open rectangles in the plane, which was considered earlier in Example \ref{ex:openballsopenrectsbasis}.
\end{example}

The product topology can also be defined in terms of a particular subbasis.

\begin{definition}
    Let $(X, \scrT_X)$, $(Y, \scrT_Y)$ be topological spaces.
    We define the \textit{projections} $\pi_1$ and $\pi_2$ to be the functions
    \begin{align*}
        \pi_1: X \times Y &\to X \\
        (x,y) &\mapsto x
    \end{align*}
and
    \begin{align*}
        \pi_2: X \times Y &\to Y \\
        (x,y) &\mapsto y
    \end{align*}
respectively.
\end{definition}

\begin{remark}
    For any set $U_X \in X$,
    \[
        \pi_1^{-1}(U_X) = U_X \times Y.
    \]
    Similarly, for any set $U_Y \in Y$,
    \[
        \pi_2^{-1}(U_Y) = X \times U_Y.
    \]
\end{remark}

\begin{convention}
    Given a function $f: X \to Y$, and a collection $\scrA$ of subsets of $Y$, we define $f^{-1}(\scrA)$ to be
    \[
        f^{-1}(\scrA) \coloneq \{f^{-1}(A): A \in \scrA\}.
    \]
\end{convention}

\begin{theorem}
    Let $(X, \scrT_X)$, $(Y, \scrT_Y)$ be two topological spaces, and let $(X \times Y, \scrT_{X \times Y})$ be their topological product.
    The collection
    \[
        \pi_1^{-1}(\scrT_X)
        \cup
        \pi_2^{-1}(\scrT_Y)
    \]
    is a subbasis for $\scrT_{X \times Y}$.
\end{theorem}

\begin{proof}
    Take two sets in $\pi_1^{-1}(\scrT_X) \cup \pi_2^{-1}(\scrT_Y)$, one from $\pi_1^{-1}(\scrT_X)$, and the other from $\pi_2^{-1}(\scrT_X)$.
    Then we may write the former set as $U_X \times Y$ and the latter as $X \times U_Y$.
    Their intersection yields
    \[
        (U_X \times Y) \cap (X \times U_Y) = (U_X \cap X) \times (U_Y \cap Y) = U_X \times U_Y.
    \]
    hence, the topology generated by the subbasis $\pi_1^{-1}(\scrT_X) \cup \pi_1^{-1}(\scrT_Y)$ contains the topology generated by the basis $\scrT_X \otimes \scrT_Y$.

    Now take some set in $\scrT_X \otimes \scrT_Y$.
    This set can be written as $U_X \times U_Y$.
    Running the same equality backwards shows that it is obtained by intersecting two sets in $\pi_1^{-1}(\scrT_X) \cup \pi_2^{-1}(\scrT_Y)$, hence we've obtained the opposite inclusion.
\end{proof}

\section{The subspace topology}

\begin{definition}
    Let $(X, \scrT)$ be a topological space.
    Let $Y \subseteq X$.
    The collection
    \[
        \scrT_Y = \{Y \cap U : U \in \scrT\}
    \]
    is called the \textit{subspace topology}, and $Y$ equipped with $\scrT_Y$ is called a \textit{subspace} of $X$.
\end{definition}

\begin{proof}[Proof (subspace topology is topology)]
    Let $(X, \scrT)$ and $\scrT_Y$ be defined as above.
    We note that $X, \varnothing \in \scrT$ by definition, so
    $Y \cap X \in \scrT_Y$, and $Y \cap \varnothing \in \scrT_Y$.
    The former is $Y$ and the latter is $\varnothing$, so this shows \ref{item:topbtmelems}.
    Intersections distribute over unions \textit{in full generality}, so we have that
    \[
        \bigcup_\alpha Y \cap U_\alpha = Y \cap \left(\bigcup_\alpha U_\alpha\right),
    \]
    which shows \ref{item:arbunion}.
    Similarly, intersections distribute over intersections, so
    \[
        \bigcap_{k=1}^n Y \cap U_k = Y \cap \left(\bigcap_{k=1}^n \cap U_k \right),
    \]
    which shows \ref{item:finintersect}.
\end{proof}

\begin{lemma}
    Let $\scrB$ be a basis for a topological space $(X, \scrT)$.
    Then
    \[
        \scrB_Y \coloneq \{B \cap Y : B \in \scrB\}
    \]
    is a basis for the subspace topology on $X$.
\end{lemma}

\begin{proof}
    Take an open set in the subspace $Y$--- this is $Y \cap U$ for some $U \in \scrT$.
    Then if we pick $y \in Y \cap U$, we know that $y \in U$, and so there must exist $B \in \scrB$ such that $y \in B \subseteq U$.
    Then $y \in Y \cap B \subseteq Y \cap U$.
    By \ref{lem:basisfromsubcollection}, $\scrB_Y$ is a basis for the subspace topology on $Y$.
\end{proof}

\begin{convention}
    If $(X, \scrT_X)$ is a topological space, we denote the subspace topology on a subset $Y \subseteq X$ by $\scrT_{Y \subseteq X}$.
\end{convention}

\begin{lemma}
    Let $(Y, \scrT_{Y \subseteq X})$ be a subspace of $(X, \scrT_X)$.
    Then if $U$ is open in $\scrT_{Y \subseteq X}$, and $Y$ itself is open in $X$, then $U$ is open in $X$.
\end{lemma}

\begin{proof}
    $U = Y \cap U_X$ for some open set $U_X \in \scrT_X$.
    Since $Y$ is open, we may use \ref{item:finintersect} to know that $Y \cap U_X \in \scrT_X$.
\end{proof}

\begin{theorem}[Products of subspaces are subspaces of products]
    Take the product $X \times Y$ of two topological spaces $X$, $Y$.
    Let $A \subseteq X$ and $B \subseteq Y$, and induce the subspace topologies on them.
    Then we have that the topological product of $A$ and $B$ is the same as the subspace $A \times B$ of $X \times Y$.
\end{theorem}

\begin{proof}
    Name the subspace topologies on $A$ and $B$ $\scrT_{A \subseteq X}$ and $\scrT_{B \subseteq Y}$ respectively.
    Name the subspace topology on $A \times B$ as a subset of $X \times Y$ $\scrT_{A \times B \subseteq X \times Y}$.

    We abuse the poor little identity about intersections and products once more, namely that
    \[
        (U \cap A) \times (V \cap B) = (U \times V) \cap (A \times B).
    \]
    The left hand side belongs to $\scrT_{A \subseteq X} \otimes \scrT_{B \subseteq Y}$--- the basis for $A \times B$ as a product of subspaces.
    The right hand side belongs to $\{U \cap E: U \in X \times Y, E \in A \times B\}$--- the basis for $A \times B$ as a subspace of a product. 
\end{proof}

\begin{example}[Subspace of order topology is order topology of suborder]
    Consider the subset $[0,1]$ of $\RR$ in the \textit{subspace topology}.
    This topology is generated by sets of the form $(a,b) \cap [0,1]$.
    We compute the possible forms this set can take.
    \begin{center}
        \begin{tabular}{ c | c c }
            & $a < 0$ & $0 \leq a$ \\ 
            \hline
            $1 < b$ & $[0,1]$ & $(a,1]$ \\
            $0 \leq b$ & $[0,b)$ & $(a,b)$
        \end{tabular}
    \end{center}
    These sets are also a basis for the \textit{order topology} on $[0,1]$.
\end{example}

\begin{example}[Subspace of order topology is not order topology of suborder] 
    Consider the subspace $[0,1) \cup \{2\}$ of $\RR$.
    The set $\{2\}$ is open in this subspace, but it is not open when you consider $[0,1) \cup \{2\}$ with the order topology, since it is smaller than any basis set that can contain it.
\end{example}

\begin{theorem}
    Let $(X, <)$ be a totally ordered set, and induce the order topology on it.
    Let $Y$ be a convex subset of $X$, meaning that it contains the interval $(a,b)$ for any pair of points $a<b$ in $Y$.
    Then the order topology on the suborder $(Y, <)$ is the same as the subspace topology on $Y$, as a subspace of the order topology on $X$.
\end{theorem}

\section*{Exercises}

\begin{exercise}
    Show that if $Y$ is a subspace of $X$, and $A$ is a subspace of $Y$, then the topology $A$ inherits as a subspace of $Y$ is the same as the topology it inherits as a subspace of $X$.
\end{exercise}

Let $\scrT_{A\subseteq Y}$ and $\scrT_{A \subseteq X}$ denote the two topologies we wish to prove are the same.

We recall the fact that for any sets $A, B, C$.
\[
    A \cap (B \cap C) = (A \cap B) \cap C.
\]

Now, let $U_A \in \scrT_{A \subseteq Y}$.
Then $U_A = A \cap U_Y$ for some $U_Y \in \scrT_{Y \subseteq X}$.
But that also means $U_Y = Y \cap U_X$ for some $U_X \in \scrT_X$, so
\[
    U_A = A \cap U_Y = A \cap (Y \cap U_X) = \underbrace{(A \cap Y)}_{=A} \cap U_X = A \cap U_X,
\]
hence $U_A \in \scrT_{A \subseteq X}$.
Then $\scrT_{A \subseteq Y} \subseteq \scrT_{A \subseteq X}$.

Conversely, if $U_A \in \scrT_{A \subseteq X}$, we can run the logic backwards
\[
    U_A = A \cap U_X = (A \cap Y) \cap U_X = A \cap \underbrace{(Y \cap U_X)}_{\coloneq U_Y} = A \cap U_Y,
\]
so that hence $U_A \in \scrT_{A \subseteq Y}$.
This proves the opposite inclusion, so both topologies are the same.

\begin{exercise}
    If $\scrT$ and $\scrT'$ are topologies on $X$ and $\scrT'$ is strictly finer than $\scrT$, what can you say about the corresponding subspace topologies on the subspace $Y$ of $X$?
\end{exercise}

The subspace of the finer topology is finer, but not neccessarily strictly, than the subspace of the coarser topology.

Let $Y \cap U$ be open in $Y$ as a subspace of $\scrT$.
Then $Y \cap U$ is open in $Y$ as a subspace of $\scrT'$ as well since $U \in \scrT'$.

There might not exist an open set $\scrT_Y \cap U'$ that is an open set in $Y$ as a subspace of $\scrT'$ but not as a subspace of $\scrT$, since even if $U' \notin \scrT$, it may be the case that $Y \cap U' = Y \cap U$ for some $U \in \scrT$.

As a silly example, the case $Y = \varnothing$ always results in the same subspace topology same no matter what the bigger topology is.

\begin{exercise}
    Consider the set $Y = [-1,1]$ as a subspace of $\RR$.
    Which of the following sets are open in $Y$?
    Which are open in $\RR$?
    \begin{align*} 
        A &= \left\{x : \frac{1}{2} < |x| < 1 \right\}, \\
        B &= \left\{x : \frac{1}{2} < |x| \leq 1 \right\}, \\
        C &= \left\{x : \frac{1}{2} \leq |x| < 1 \right\}, \\
        D &= \left\{x : \frac{1}{2} \leq |x| \leq 1 \right\}, \\
        E &= \left\{x : 0 < |x| < 1 \text{ and } 1/x \notin \ZZ_+ \right\}.
    \end{align*}
\end{exercise}

The sets $A$ through $D$ can be rewritten
\begin{align*}
    A &= \left(-1, -\frac{1}{2}\right) \cap \left(\frac{1}{2}, 1\right), \\
    B &= \left[-1, -\frac{1}{2}\right) \cap \left(\frac{1}{2}, 1\right], \\
    C &= \left(-1, -\frac{1}{2}\right] \cap \left[\frac{1}{2}, 1\right), \\
    D &= \left[-1, -\frac{1}{2}\right] \cap \left[\frac{1}{2}, 1\right]. \\
\end{align*}

Then, we can see that $A$ is open in both $\RR$ and $Y$.
$B$ is open in $Y$ but not in $\RR$.
$C$ and $D$ are neither open in $Y$ nor open in $\RR$.

$E$ is open in both $Y$ and $\RR$, since it decomposes as 
\[
    (-1,1) \cup \bigcup_{n=1}^\infty \left(\frac{1}{n+1},\frac{1}{n}\right).
\]

\begin{exercise}
    A map $f: X \to Y$ is said to be an \textit{open map} if for every open set $U$ of $X$, the set $f(U)$ is open in $Y$.
    Show that $\pi_1: X \times Y \to X$ and $\pi_2: X \times Y \to Y$ are open maps.
\end{exercise}

Let $A$ be an open set of $X \times Y$.

Let $(x,y) \in A$.
Then $(x,y) \in U \times V \subseteq A$, where $U$ is open in $X$ and $V$ is open in $Y$.

We have that $U = \pi_1(U \times V) \subseteq \pi_1(A)$.

Then $\pi_1(A)$ is open, since for all $x \in \pi_1(A)$ we can find an open set $U$ of $X$ such that $x \in U \subseteq \pi_1(A)$.

Similarly, $\pi_2(A)$ is open.

\begin{exercise}
    Let $X$ and $X'$ denote a single set in the topologies $\scrT$ and $\scrT'$, respectively; let $Y$ and $Y'$ denote a single set in the topologies $\scrU$ and $\scrU'$, respectively.
    Assume these sets are nonempty.
    \begin{enumerate}[label=(\alph*)]
        \item 
            Show that if $\scrT' \supset \scrT$ and $\scrU' \supset \scrU$, then the product topology on $X' \times Y'$ is finer than the product topology on $X \times Y$.
        \item 
            Does the converse of (a) hold?
            Justify your answer.
    \end{enumerate}
\end{exercise}

I don't get it honestly.

\begin{exercise}
    Show that the countable collection
    \[
        \{(a,b)\times(c,d) : a<b \text{ and } c<d \text{ and } a,b,c,d \text{ are rational.}\}
    \]
    is a basis for $\RR^2$.
\end{exercise}

By a previous exercise, the set
\[
    \{(a,b): a<b \text{ and }a,b \text{ are rational.}\}
\]
is a basis for $\mathbb{R}$.
Then, by Theorem \ref{thm:basisforproduct}, the set in question is a basis for $\RR^2$.

\begin{exercise}
    Let $X$ be an ordered set.
    If $Y$ is a proper subset of $X$ that is convex in $X$, does it follow that $Y$ is an interval or a ray in $X$?
\end{exercise}

The set
\[
    E = \{p : p^2 < 2\}
\]
is convex in $\QQ$, but is not an interval or ray in $\QQ$.

\begin{exercise}
    If $L$ is a straight line in the plane, describe the topology $L$ inherits as a subspace of $\RR_\ell \times \RR$ and as a subspace of $\RR_\ell \times \RR_\ell$.
    In each case it is a familiar topology.
\end{exercise}

For $\RR_\ell \times \RR$, if $L$ is \textit{not} a vertical line, it has the lower limit topology, and if $L$ is a vertical line, it has the standard topology.



If $L$ is a straight line, it can be described as either the sets
\[
    \{ (x, ax+b) : x \in \RR \}
\]
or
\[
    \{(x_0, y) : y \in \RR \}.
\]

In the first case, the basis elements of $\RR_\ell \times \RR$ are of the form

\[
    (x, ax+b) \cap \Big([p,q) \times (r,s)\Big)
\]
which is equal to the conditions that $p \leq x < q$ and $r < ax+b < s$.

\[
    p \leq x < q
\]
\[
    \frac{r}{a} - b < x < \frac{s}{a} - b
\]

Which is

In the latter case, 

\todo{Exercise 16.8}

For $\RR_\ell \times \RR_\ell$, the topology is the lower limit topology for \textit{all} lines $L$.

\begin{exercise}
    Show that the dictionary order topology on the set $\RR \times \RR$ is the same as the product topology $\RR_d \times \RR$, where $\RR_d$ denotes $\RR$ in the discrete topology.
    Compare this topology with the standard topology on $\RR^2$.
\end{exercise}

To avoid nesting too many brackets, denoted a \textit{pair} of real numbers $(a,b)$ by $ab$.

The basis sets of $\RR \times \RR$ in the dictionary order are intervals of the form
\[
    (ab, cd)
\]
where either $a<c$ or $a=c$ and $b<d$.

If $a < c$, this basis element can be broken down into the union of three pieces:
\begin{align*}
    \{a\} &\times (b, \infty) \\
    (a,c) &\times \RR \\
    \{c\} &\times (-\infty, d),
\end{align*}
each corresponding to the case that the first coordinate equals $a$, is between $a$ and $c$, and equals $c$ respectively.

Each piece is open in $\RR_d \times \RR$ respectively.

If $a=c$ and $b<d$, then our basis set is
\[
    \{a\} \times (b,d).
\]
Which is again open in $\RR_d \times \RR$.
We have now shown one inclusion of topologies.

Take some basis element of $\RR_d \times \RR$.
This is a set of the form 
\[
    \{a\} \times (b,c)
\]
which is evidently the interval $(ab, ac)$ in the dictionary order.
This shows the reverse inclusion.

\begin{exercise}
    Let $I = [0,1]$.
    Compare the product topology on $I \times I$, the dictionary order topology on $I \times I$, and the topology $I \times I$ inherits as a subspace of $\RR \times \RR$ in the dictionary order topology.
\end{exercise}

\todo{Exercise 16.9}

\section{Closed sets and limit points}

\begin{convention}
    Let $X$ be a set, and let $A \subseteq X$.
    We denote the complement $X \setminus A$ to be $A^c$.
\end{convention}

\begin{definition}
    Let $X$ be a topological space.
    We say that a subset $A$ of $X$ is closed whenever $A^c$ is open.
\end{definition}

\begin{example}
    The subset $[a,b] \in \RR$ is closed, since it is the complement of the open set 
    \[
        (-\infty, a) \cap (b, \infty).
    \]
\end{example}

\begin{example}
    The subset
    \[
        \{(x,y): x \geq 0, y \geq 0\}
    \]
    of $\RR^2$ is closed, since it is the complement of the union of the two open sets
    \[
        \pi_1^{-1}\Big((-\infty, 0)\Big) \qquad \text{and} \qquad \pi_2^{-1}\Big((-\infty, 0)\Big).
    \]
    Namely, these are open half-spaces--- the former is the left hand plane, and the latter is the bottom half plane.
\end{example}

\begin{example}
    The closed sets in the cofinite topology on $X$ are the finite subsets of $X$ and $X$ itself.
\end{example}

\begin{example}
    In every discrete topology, all subsets are closed since all subsets are open.
\end{example}

\begin{example}
    Consider the subset 
    \[
        Y = [0,1] \cup (2,3)
    \]
    of the real line.
    It has $[0,1]$ as an open set, and also as a closed set.
    Similarly, it has $(2,3)$ as a closed set, and also as a nopen set.
\end{example}

\begin{definition}
    A subset of a topological space that is both closed an open is called \textit{clopen}.
\end{definition}

\begin{theorem}[Closed set axioms]
    Let $X$ be a topological space.
    Let $\scrC$ denote the collection of all closed sets of $X$.
    Then
    \begin{enumerate}[label=(C\arabic*)]
        \item $\varnothing, X \in \scrC$. \label{ax:closure:topbtmelems}
        \item $\scrC$ is closed under arbitrary intersection. \label{ax:closure:arbintersect}
        \item $\scrC$ is closed under finite unions. \label{ax:closure:finunion}
    \end{enumerate}
\end{theorem}

\begin{proof}
    For \ref{ax:closure:topbtmelems}, $X = \varnothing^c$, and $\varnothing = X^c$.

    \ref{ax:closure:arbintersect} can be read off the equation
    \[
        \left(\bigcup_\alpha U_\alpha\right)^c = \bigcap_\alpha U_\alpha^c,
    \]
    an application of DeMorgan's laws.

    Similarly, we can read off \ref{ax:closure:finunion} from the equation
    \[
        \left(\bigcap_{k=1}^n U_k\right)^c = \bigcup_{k=1}^n U_k^c.
    \]
\end{proof}

\begin{theorem}
    Let $Y$ be a subspace of $X$.
    Then $A \subseteq Y$ is closed in $Y$ if and only if it is $Y \cap C$ for some closed set $C$ of $X$.
\end{theorem}

\begin{proof}
    Let $A$ be closed in $Y$.
    Then $A$ is the complement (in $Y$) of some open set in $Y$.
    Then $A = Y \setminus (Y \cap U)$, where $U$ is an open set of $X$.
    So
    \[
        A = Y \setminus U = Y \cap (X \setminus U) = Y \cap U^c.
    \]
    Then $A$ is $Y \cap C$ for a closed set $C$ of $X$, namely $U^c$.

    Conversely, if $A = Y \cap C$, then we run the equations backwards.
\end{proof}

\begin{theorem}
    Let $Y$ be a subspace of $X$. If $A$ is closed in $Y$ and $Y$ is closed in $X$, then $A$ is closed in $X$.
\end{theorem}

\begin{proof}
    If $A$ is closed in $Y$, then $A = Y \setminus V$, where $V$ is open in $Y$.
    Then $A = Y \setminus (Y \cap U)$, where $U$ is open in $X$.
    So,
    \[
        A = Y \cap (Y \cap U)^c = Y \cap (Y^c \cup U^c) = (Y \cap Y^c) \cup (Y \cap U) = Y \cap U^c.
    \]
    Since $U^c$ is closed and $Y$ is closed, $Y \cap U^c$ is closed.
\end{proof}

\subsection{Closure and interior of a set}
\begin{definition}
    Let $X$ be a topological space, and let $A$ be a subset of $X$.
    The \textit{interior} of $A$ is the largest open set contained in $A$, and the \textit{closure} of $A$ is the smallest closed set containing $A$.

    Specifically, the interior of $A$, $\interior A$, is defined to be
    \[
        \interior A \coloneq \bigcup_{\substack{U\text{ is open} \\U \subseteq A}}U,
    \]
    and the closure of $A$, $\closure A$, is defined to be
    \[
        \closure A \coloneq \bigcap_{\substack{U\text{ is closed} \\U \supseteq A}}U.
    \]
\end{definition}

\begin{theorem}
    Let $Y$ be a subspace of $X$, and let $A$ be a subset of $Y$.
    Define $\closure_Y A$ and $\closure_X A$ to be the closure of $A$ with respect to $X$ and $Y$ respectively.
    Then
    \[
    \closure_Y A = \closure_X A \cap Y.
    \]
\end{theorem}

\begin{proof}
    The left hand side is
    \[
        \bigcap_{\substack{V \text{ is closed in } Y \\ V \supseteq A}} V 
    \]
    Since $V$ is closed in $Y$, we may write it as $Y \cap C$, where $C$ is closed in $X$.
    So we can rewrite the above to be
    \[
        \bigcap_{\substack{C \text{ is closed in } X \\ Y \cap C \supseteq A}} Y \cap C.
    \]
    Now, since $A \subseteq Y$, the condition $Y \cap C \supseteq A$ is equivalent to $C \supseteq A$.
    Moreover, we can use distributivity of intersection with itself to pull out the $Y$.
    Now, our expression is
    \[
        Y \cap \bigcap_{\substack{C \text{ is closed in } X \\ C \supseteq A}} C.
    \]
    which is evidently the right hand side of the equation we want to prove.
\end{proof}

\begin{definition}
    Fix some background topological space.
    An open set containing a point $x$ is called a \textit{neighborhood of $x$}.
    If the intersection of two sets $A$ and $B$ is nonempty, we say that \textit{$A$ intersects $B$} and vice versa.
\end{definition}

\begin{theorem}
    Let $X$ be a topological space, and let $A$ be a subset of $X$.
    \begin{enumerate}[label=(\alph*)]
        \item $x \in \closure A$ if and only if every neighborhood of $x$ intersects $A$.
        \item If $\scrB$ is a basis for the topology on $X$, then $x \in \closure A$ if and only if $A$ intersects every basis element containing $x$.
    \end{enumerate}
\end{theorem}

\begin{proof}
    Take a point $x$.
    Suppose $x$ has a neighborhood $N$ does not intersect $A$.
    Then there exists a closed set $C$ containing $A$ that does not intersect $N$ (namely, $N^c$).
    So $x$ \textit{cannot} be in the intersection of all closed sets containing $A$, since it doesn't belong in at least one of them.

    Hence, if $x$ has a neighborhood $N$ that does not intersect $A$, it cannot be in the closure of $A$.
    By contraposition, if $x \in \closure A$, it must be that every neighborhood of $x$ intersects $A$.

    Now suppose that $x$ is not in the closure of $A$.
    We can then produce a closed set $C$ that does not contain $x$.
    Then, $C^c$ is a neighborhood of $x$ not intersecting $A$.
    This proves the opposite direction, and we use contraposition again--- if every neighborhood of $x$ intersects $A$, then $x \in \closure A$.

    For (b), we note that every basis set is open, and that every open set containing $x$ contains a basis set containing $x$.
\end{proof}

\begin{example}
    In $\RR$, $\closure (0,1] = [0,1]$.

\end{example}
\todo{Closure examples}


\subsection{Limit Points}

\begin{definition}
    A \textit{limit point} $x$ of a subset $A$ of a topological space $X$ is a point in which every neighborhood of $x$ contains a point of $A$ \textit{distinct from $x$ itself}.
\end{definition}

\begin{theorem}
    Let $X$ be a topological space.
    Let $A$ be any subset of $X$.
    Let $A'$ be the set of all limit points of $A$.
    Then
    \[
        \closure A = A \cup A'.
    \]
\end{theorem}

Let $x$ be in the closure of $A$.
Then it is contained in every closed set containing $A$.
Suppose that

\subsection{Hausdorff spaces}

\begin{definition}
    A topological space $X$ is \textit{Hausdorff} whenever points can be separated with open sets, meaning that for two distinct points $x_1, x_2 \in X$, there exist \textit{disjoint} neighborhoods $U_1$ and $U_2$ of $x_1$ and $x_2$ respectively.
\end{definition}

\begin{definition}
    Let $X$ be a topological space.
    We say that a sequence $x_1, x_2, \ldots$ of points in $X$ \textit{converges to the point $x$} if, for every neighborhood $U$ of $x$, there exists a positive integer $N$ such that $x_n \in U$ for all $n \geq N$.
\end{definition}

\begin{theorem}
    In a Hausdorff space, all finite sets are closed.
\end{theorem}

\begin{proof}
    Let $x$ be any point at all in a Hausdorff space.

    The closure of $\{x\}$ is $\{x\}$, since for any other point $x'$, we may produce a neighborhood $U$ of $x'$ that does not intersect $\{x\}$--- choose a neighborhood $V$ of $x$ that is disjoint from $U$.
    A fortiori, the fact that $U$ and $V$ are disjoint means that $U$ and $\{x\}$ are disjoint, so $x'$ is not in the closure of $\{x\}$.
    Since $\{x\}$ equals its own closure, it must be closed.

    Since finite unions of closed sets are closed, it follows that all finite sets in a Hausdorff space are closed.
\end{proof}

The condition that we can separate \textit{one point from another with a neighborhood} is weaker than being Hausdorff.

\begin{definition}
    A topological space $X$ is $T_1$ if any two distinct points have neighborhoods which do not contain the other.
\end{definition}

\begin{remark}
    A topological space is $T_1$ if and only if every singleton is closed.
\end{remark}

\begin{theorem}
    Let $X$ be a $T_1$ space.
    Let $A \subseteq X$.
    Then $x$ is a limit point of $A$ if and only if every neighborhood of $x$ contains infinitely many points of $A$.
\end{theorem}

\begin{proof}
    Suppose there exists a neighborhood $U$ of $x$ which contains only finitely many points of $A$.
    Then, if we label these points $a_1, a_2, \ldots, a_n$, the set $\{a_1, a_2, \ldots, a_n\}$ is closed. Then
    \[
        A \setminus \{a_1, a_2, \ldots, a_n\}
    \]
    is an open set containing $x$ --- a neighborhood of $x$ --- containing no points of $A$.
    Then $x$ cannot be a limit point of $A$.
    By contraposition, if $x$ is a limit point of $A$ then every neighborhood of $x$ contains infinitely many points of $A$.

    For the other direction, if every neighborhood of $x$ contains infinitely many points of $A$, it definitely has to contain \textit{some} point of $A$, so it is a limit point of $A$.
\end{proof}

\begin{theorem}
    In Hausdorff spaces, sequences converge to at most one point.
\end{theorem}

\begin{proof}
    Let
    \[
        x_n \xrightarrow{n\to\infty} x.
    \]
    Take a point $x' \neq x$.
    Then we can separate $x$ and $x'$ with disjoint neighborhoods $U$ and $U'$ respectively.
    Now, since $x_n \xrightarrow{n\to\infty} x$, there exists $N$ such that $x_n \in U$ for all $n \geq N$.
    However, this means $x_n \neq U'$ for all $n \geq N$.
    Hence it's impossible to find a positive integer $M$ such that $m \geq M$ implies $x_m \in U'$ (for any $M$, pick the first $n$ greater than both $M$ and $N$).
    So it must be that $x_n \centernot{\xrightarrow{n\to\infty}}x$.
\end{proof}

\begin{theorem}[Some facts about Hausdorff spaces]\
    \begin{enumerate}[label=(\alph*)]
        \item Every totally ordered set is Hausdorff in the order topology.
        \item The product of two Hausdorff spaces is again Hausdorff.
        \item Subspaces of Hausdorff spaces are Hausdorff.
    \end{enumerate}
\end{theorem}

\section*{Exercises}

\todo{Section 17 exercises}

\section{Continuous functions}

\subsection{Continuity of a function}

\begin{definition}
    Let $X$ and $Y$ be topological spaces.
    A function $f: X \to Y$ is said to be \textit{continuous} if for all open sets $V$ of $Y$, the set $f^{-1}(V)$ is an open subset of $X$.
\end{definition}

\begin{proposition}
    Let $f: X \to Y$ be a map between topological spaces, and suppose we have a basis for $Y$.
    Then, to prove that $f$ is continuous, it suffices to check that the preimage of every basis element is open.

    Similarly, if we have a subbasis for $Y$, $f$'s continuity can be checked by checking that the preimages of subbasis elements are open.
\end{proposition}

\begin{theorem}
    Let $X$ and $Y$ be topological spaces, and let $f:X \to Y$.
    Then the following are equivalent:
    \begin{enumerate}[label=(\alph*)]
        \item $f$ is continuous.
        \item $f(\overline{A}) \subseteq \overline{f(A)}$ for all $A \subseteq X$.
        \item For every closed subset $B$ of $Y$, $f^{-1}(B)$ is closed in $X$.
        \item For each $x \in X$ and each neighborhood $V$ of $f(x)$, there is a neighborhood of $x$ such that $f(U) \subseteq V$.
    \end{enumerate}
\end{theorem}

\begin{example}[Continuity of real-valued functions]
    In analysis, we have the \textit{epsilon-delta} definition of continuity, which states that, for functions $f: \RR \to \RR$,
    \begin{align*}
        &f\text{ is continuous at }x_0 \\
        &\iff \\
        &\forall \varepsilon > 0, \exists \delta > 0, \forall x \Big(|x - x_0| < \delta \implies |f(x) - f(x_0)| < \varepsilon\Big).
    \end{align*}
    This definition is equivalent to continuity of real-valued functions with our previous definition, taking the standard topology on $\RR$.
\end{example}

\begin{proof}
    Fix $f: \RR \to \RR$.
    Suppose that $f$ is continuous as per the open set definition.
    Then $f^{-1}\big((a,b)\big)$ is open in $\RR$ for all open intervals $(a,b)$.
    Let $\varepsilon > 0$.
    Consider the interval
    \[
        V = \left(f(x_0) - \frac{\varepsilon}{2}, f(x_0) + \frac{\varepsilon}{2}\right).
    \]
    $V$ contains $f(x_0)$, hence $f^{-1}(V)$ must contain $x_0$.
    $V$ is also open, hence $f^{-1}(V)$ is open.
    There must be then some basis element $(a,b)$ such that $x_0\in (a,b) \subseteq f^{-1}(V)$.
    Put $\delta = \min\{|a-f(x_0)|, |b-f(x_0)|\}$.
    Then, if $|x - x_0| < \delta$, we have that $x \in (x_0-\delta, x_0+\delta) \subseteq (a,b)$, so $f(x) \in f((a,b)) \subseteq f(V)$.
    By how we defined $V$, it must be that $|f(x) - f(x_0)| < \varepsilon$, which shows that $f$ is continuous at $x_0$.
    Since $x_0$ was arbitrary, $f$ is continuous per the epsilon-delta definition.
\end{proof}

\subsection{Homeomorphisms}

\begin{definition}
    Let $X$ and $Y$ be topological spaces, and let $f : X \to Y$ be a bijection.
    If $f$ and $f^{-1}$ are both continuous, then $f$ is called a \textit{homeomorphism}.
\end{definition}

\begin{remark}
    Equivalently, $f$ is a homeomorphism if it is bijective, continuous, and an \textit{open map}, i.e $f(U)$ is open in $Y$ if $U$ is open in $X$.
\end{remark}

\begin{definition}
    If $f$ is a continuous injective map from $X$ into $Y$, then it is called a \textit{topological imbedding} or simply an \textit{imbedding} of $X$ in $Y$.

    If we endow $f(X)$ with the subspace topology, then $f$ is a homeomorphism of $X$ and $f(X)$.
\end{definition}

\begin{example}
    $x \mapsto 3x+1$ is a homeomorphism.
\end{example}

\begin{example}
    $x \mapsto x/(1-x^2)$ is a homeomorphism.
\end{example}

\begin{example}
    Let $S^1$ denote \textit{the unit circle},
    \[
        S^1 \coloneq \{(x,y), x^2 + y^2 = 1\}.
    \]
    The map $t \mapsto (\cos 2\pi t, \sin 2 \pi t)$ is a continuous bijective map between $[0,1)$ and $S^1$ that is \textit{not} a homeomorphism.
\end{example}

\begin{theorem}
    Let $X,Y,Z$ be topological spaces.
    \begin{enumerate}[label=(\alph*)]
        \item The constant map $X \to Y$ given by $x \mapsto y_0$ is continuous.
        \item For any subspace $A$ of $x$, the \textit{inclusion map} $\iota_A: A \to X$ is continuous.
        \item If $f: X \to Y$ and $g: Y \to Z$ are continuous, then the map $g \circ f$ is continuous.
        \item If $f: X \to Y$ is continuous and $A$ is a subspace of $X$, then $f|_A: A \to Y$ is continuous.
        \item a
        \item ooga
    \end{enumerate}
\end{theorem}

\begin{theorem}[Pasting lemma]
    Let $X, Y$ be topological spaces.
    Suppose $X$ is the union of two closed sets $A$ and $B$, and suppose that we have two continuous functions $f: A \to Y$ and $g: B \to Y$, then if $f(x) = g(x)$ for all $x \in A \cap B$, the function 
    \[
        x \mapsto \begin{cases}
            f(x) & x \in A \\
            g(x) & x \in B \setminus A
        \end{cases}
    \]
    is continuous.
\end{theorem}

\begin{theorem}
    Let $f: A \to X \times Y$.
\end{theorem}






\end{document}
