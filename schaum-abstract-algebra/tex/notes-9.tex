\section{Groups}

\definition
A non-empty set $\mathcal{G}$ equipped with a binary operation $\circ$ is a \textit{group} if 
\begin{itemize}
    \item[$\mathbf{P}_1$:] $(a \circ b) \circ c = a \circ (b \circ c)$ (Associativity)
    \item[$\mathbf{P}_2$:] There exists an element $1 \in \mathcal{G}$ such that $1 \circ a = a \circ 1 = a$ for all $a \in \mathcal{G}$ (Unit)
    \item[$\mathbf{P}_3$:] For all $a \in \mathcal{G}$, there exists an element $a^{-1}$ such that $a \circ a^{-1} = a^{-1} \circ a = 1$ (Inverse)
\end{itemize}


\section{Simple Properties of Groups}
\theorem (Left Cancellation) Let $a,b,c \in \mathcal{G}$. Then $a \circ b = a \circ c$ implies $b = c$.

\theorem (Latin Square Property) 

