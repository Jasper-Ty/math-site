\section{Rings}

\definition 
A non-empty set $R$ equipped with two operations $+$ and $\cdot$ is a $\textit{ring}$ if
\begin{itemize}
    \item[(a)]
        $(R, +)$ is an abelian group
    \item[(b)]
        $(R, \cdot)$ is a monoid
    \item[(c)]
        $+$ distributes over $\cdot$, i.e
        \[a(b+c) = ab + ac\]
        \[(b+c)a = ba + ca\]
        for all $a, b, c \in R$
\end{itemize}

\example
$\mathbb{Z}$, $\mathbb{Q}$, $\mathbb{R}$, $\mathbb{C}$, are rings.


\example $S = \{x + y\sqrt[3]{3} + \}$


\section{Properties of Rings}

\section{Subrings}
