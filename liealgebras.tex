\documentclass{article}

% See https://github.com/Jasper-Ty/dotfiles
\usepackage[garamond]{jaspercommon}
\usepackage{tikz-cd}

\newcommand*\wc{{\mkern 3mu\_\mkern 3mu}}
\newcommand{\innerproduct}[2]{\ensuremath{\left\langle #1 , #2 \right\rangle}}
\newcommand{\ip}[1]{\ensuremath{\left\langle{#1}\right\rangle}}
\newcommand{\lb}[1]{\ensuremath{\left[{#1}\right]}}

\newcommand{\iverson}[1]{\ensuremath{\left[{#1}\right]^?}}

\DeclareMathOperator{\s}{s}
\DeclareMathOperator{\ogroup}{O}
\DeclareMathOperator{\Dih}{Dih}
\DeclareMathOperator{\End}{End}
\DeclareMathOperator{\Sym}{Sym}
\DeclareMathOperator{\tr}{tr}

\DeclareMathOperator{\im}{im}

% The algebra of derivations
\DeclareMathOperator{\Der}{Der}

% The adjoint representation of a Lie algebra
\DeclareMathOperator{\ad}{ad}

% An anonymous differential
\newcommand{\dd}{\ensuremath{\text{d}}}

\newcommand{\frka}{\ensuremath{\mathfrak{a}}}
\newcommand{\frkg}{\ensuremath{\mathfrak{g}}}
\newcommand{\frkh}{\ensuremath{\mathfrak{h}}}
\newcommand{\frku}{\ensuremath{\mathfrak{u}}}
\newcommand{\frki}{\ensuremath{\mathfrak{i}}}
\newcommand{\frkj}{\ensuremath{\mathfrak{j}}}
\newcommand{\frkU}{\ensuremath{\mathfrak{U}}}

\newcommand{\GL}{\ensuremath{\text{GL}}}
\newcommand{\SL}{\ensuremath{\text{SL}}}

\newcommand{\glalg}{\ensuremath{\mathfrak{gl}}}
\newcommand{\slalg}{\ensuremath{\mathfrak{sl}}}
\newcommand{\spalg}{\ensuremath{\mathfrak{sp}}}
\newcommand{\talg}{\ensuremath{\mathfrak{t}}}
\newcommand{\oalg}{\ensuremath{\mathfrak{o}}}

\title{Lie algebras}
\author{Jasper Ty}
\date{}

\titleauthorhead

\begin{document}

\maketitle

\section*{What is this?}

These are notes based on my reading of Humphreys's ``Introduction to Lie Algebras and Representation Theory''.

\tableofcontents

\newpage

\section{Definitions}

\begin{convention}
    All vector spaces are finite dimensional and no assumptions are made about the field they are over.
\end{convention}

\subsection{Lie algebras}

\begin{definition}
    A \defstyle{Lie algebra} $\frkg$ is a vector space equipped with a product
    \begin{align*}
        \lb{\wc,\wc}: \frkg \times \frkg
        &\to
        \frkg,
        \\
        (x,y)
        &\mapsto
        \lb{xy},
    \end{align*}
    such that
    \begin{enumerate}[label=(L\arabic*)]
        \item 
            \label{ax:LBIsBilinear}
            $\lb{\wc,\wc}$ is bilinear,
        \item 
            \label{ax:LBNilpotent}
            $\lb{xx} = 0$ for all $x \in \frkg$, and
        \item 
            \label{ax:LBJacobiIdentity}
            $\lb{x\lb{yz}} + \lb{y\lb{zx}} + \lb{z\lb{xy}} = 0$.
    \end{enumerate}
    We refer to $\lb{xy}$ as the \defstyle{bracket} or the \defstyle{commutator} of $x$ and $y$.
\end{definition}

\ref{ax:LBJacobiIdentity} is referred to as the \textit{Jacobi identity}.

As an exercise in using this definition, we show the following:

\begin{proposition}
    Brackets are anticommutative, i.e
    \begin{equation}
        [xy]
        =
        -[yx].
        \tag{L2'}
    \end{equation}
    is a relation in any Lie algebra.
\end{proposition}
\begin{proof}
    By \ref{ax:LBNilpotent}, we have that
    \[
        \lb{x+y,x+y}
        =
        0,
    \]
    and by \ref{ax:LBIsBilinear},
    \[
        \lb{xx} + \lb{xy} + \lb{yx} + \lb{yy}
        =
        0.
    \]
    By \ref{ax:LBNilpotent} again,
    \[
        \lb{xy} + \lb{yx}
        =
        0,
    \]
    which completes the proof.
\end{proof}

We will look at our first example of a Lie algebra, which is closely related to the $\GL(V)$.

\begin{definition}[$\glalg$, abstractly]
    Let $V$ be a vector space.
    The \defstyle{general linear algebra} $\glalg(V)$ is defined to be the Lie algebra with underlying vector space $\End V$ and bracket given by
    \[
        \lb{xy}
        =
        xy - yx
    \]
    defined with $\End V$ natural ring structure.
\end{definition}

    Put in a more concrete sense, $\End V$'s aforementioned ring structure is exactly that of $n \times n$ matrices, where $n = \dim V$.
    Then, the following definition makes sense, and is in a sense ``the only'' finite dimensional general linear algebra.

\begin{definition}[$\glalg$, concretely]
    Let $\FF$ be some field and let $n$ be a positive integer.
    The \defstyle{general linear algebra} $\glalg_n(\FF)$ is defined to be the Lie algebra with underlying vector space the set of all $n \times n$ matrices over $\FF$, with bracket given by
    \[
        \lb{xy}
        =
        xy - yx.
    \]
\end{definition}

We can ea

\begin{proposition}
    Let $\{e_{pq}\}_{p,q = 0}^n$ be the standard basis of $\glalg_n(\FF)$.
    Then
    \[
        \lb{e_{pq}e_{rs}}
        =
        \delta_{qr}e_{ps}
        -
        \delta_{sp}e_{rq},
    \]
    where $\delta$ is the Kronecker delta.
\end{proposition}

\begin{proof}
    Using the Iverson bracket,
    \[
        (e_{pq})_{ij}
        =
        \iverson{p=i \wedge q=j}
    \]
    and so
    \begin{align*}
        (e_{pq}e_{rs})_{ij}
        &=
        \sum_{k=1}^n
        (e_{pq})_{ik}
        (e_{rs})_{kj}
        \\
        &=
        \sum_{k=1}^n
        \iverson{p=i \wedge q=k}
        \iverson{r=k \wedge s=j}
        \\
        &=
        \left(
            \sum_{k=1}^n
            \iverson{q=r=k}
        \right)
        \iverson{p=i \wedge s=j}
        \\
        &=
        \delta_{qr}
        (e_{ps})_{ij}.
    \end{align*}
    So $e_{pq}e_{rs} = \delta_{qr}e_{ps}$. 
    Similarly,
    $e_{rs}e_{pq} = \delta_{sp}e_{rq}$.
\end{proof}

Importantly, many Lie algebras, and in fact all the Lie algebras we are concerned with, occur as subalgebras of the general linear algebra--- a \defstyle{subalgebra} of a Lie algebra $\frkg$ is a vector subspace of $\frkg$ that is closed under the bracket.

\begin{definition}
    A \defstyle{linear Lie algebra} is a subalgebra of $\glalg_n(\FF)$ for some $n$.
\end{definition}


\subsection{Derivations, the adjoint representation}

\begin{definition}
    Let $\frkU$ be a $\FF$-algebra.
    A \defstyle{derivation} of $\frkU$ is a linear map $\dd: \frkU \to \frkU$ which satisfies the \textit{Leibniz rule}
    \[
        \dd(ab)
        =
        a(\dd b) + (\dd a) b.
    \]
    The collection of all derivations of $\frkU$ is denoted $\Der \frkU$.
\end{definition}

\begin{definition}
    The \defstyle{adjoint representation} of a Lie algebra $\frkg$ is the mapping
    \[
        x \mapsto \ad_x
    \]
    where $\ad_x \in \Der \frkg$ is defined to be
    \begin{align*}
        \ad_x: 
        \frkg 
        &\to 
        \frkg \\
        y 
        &\mapsto
        \lb{x,y}.
    \end{align*}
\end{definition}

\begin{proposition}
    $\ad_x$ is a derivation.
\end{proposition}
\begin{proof}
    We start with the Jacobi identity \ref{ax:LBJacobiIdentity}
    \[
        \lb{x\lb{yz}} + \lb{y\lb{zx}} + \lb{z\lb{xy}}
        =
        0,
    \]
    which, using the anticommutation relations $\lb{y\lb{zx}} = -\lb{y\lb{xz}}$ and $\lb{z\lb{xy}} = -\lb{\lb{xy}z}$, is equivalent to
    \[
        \lb{x\lb{yz}}
        =
        \lb{y\lb{xz}} + \lb{\lb{xy}{z}}.
    \]
    But this is saying that
    \[
        \ad_x\lb{yz}
        =
        \lb{y, \ad_x z}
        +
        \lb{\ad_xy, z}
    \]
    which is exactly the defining identity for derivations.
\end{proof}


\subsection{Examples}

We have four distinguished families of Lie algebras:
\[
    \sfA_\ell,\qquad
    \sfB_\ell,\qquad
    \sfC_\ell,\qquad
    \sfD_\ell.
\]
These classify all but five of the so-called \defstyle{semisimple Lie algebras}.


\subsubsection{Type \sfA}

\begin{definition}
    Let $V$ be a $\FF$-vector space, and fix a basis $\{v_1,\ldots,v_n\}$ of $V$.
    The \defstyle{trace} of an endomorphism $x \in \End V$ of $V$ is defined to be the sum
    \[
        \sum_{i=1}^n\ip{v_i,x(v_i)}
    \]
    where $\ip{-,-}$ is the canonical pairing.

    In other words, it is the sum of the diagonal entries of the matrix representation of $x$.

    We say that $x$ is \defstyle{traceless} if $\tr x = 0$.
\end{definition}

\begin{definition}
    The \defstyle{special linear algebra} $\slalg(V)$ is defined to be the set of all traceless endomorphisms of $V$.
\end{definition}

\begin{proposition}
    $\slalg(V)$ is a subalgebra of $\glalg(V)$.
\end{proposition}
\begin{proof}
    The trace is a linear operator $\tr: \glalg(n, \FF) \to \FF$.
    Since the kernel of a linear operator is a vector subspace, we conclude that $\slalg(n, \FF)$ is a vector subspace of $\glalg$.

    Finally, the fact that $\tr(xy - yx) = \tr(xy) - \tr(yx) = 0$ for \textit{all} $x,y \in \glalg(n,\FF)$ means that $\glalg(n,\FF)$'s Lie bracket is closed in $\slalg(n,\FF)$.
\end{proof}

\subsubsection{Type \sfB}

\begin{definition}
    The \defstyle{orthogonal algebra} $\oalg(2n+1,\FF)$ is defined to be
\end{definition}

\subsubsection{Type \sfC}

\begin{definition}
    Let $\dim V = 2\ell$.
    We define a skew-symmetric bilinear form $f$ on $V$ via the matrix
    \[
        s
        \coloneq
        \begin{pmatrix}
            0 & I_\ell \\
            -I_\ell & 0 \\
        \end{pmatrix}.
    \]
    Namely,
    \[
        f(u,v)
        \coloneq
        u^Tsv.
    \]
    The \defstyle{symplectic algebra} $\spalg(V)$ is defined to be the set of all $x \in \End V$ such that
    \[
        f\Big(x(u),v\Big)
        =
        -f\Big(u,x(v)\Big).
    \]
\end{definition}

\subsubsection{Type \sfD}

\begin{definition}
    The \defstyle{orthogonal algebra} $\oalg(2n,\FF)$ is defined to be
\end{definition}

\subsection{Abstract Lie algebras}

\section{Ideals and homomorphisms}

\begin{definition}
    A subspace $I$ of a Lie algebra $L$ is called an \defstyle{ideal} of $L$ if $\lb{xy} \in I$ for all $x \in L$ and $y \in I$.
\end{definition}


\begin{definition}
    The \defstyle{quotient of a Lie algebra} $L$ by an ideal $I$, denoted $L/I$, is defined to be the quotient of $L$ as a vector space by $I$ as a subspace, equipped with the product
    \[
        \lb{x+I,y+I}
        \coloneq
        \lb{xy} + I.
    \]
\end{definition}

\begin{proposition}
    $L/I$ is a Lie algebra.
\end{proposition}
\begin{proof}
    These are all easy to check.
    \begin{align*}
        \lb{ax+by+I,z+I}
        &=
        \\
        \Big(\lb{ax+by,z}\Big) + I
        &=
        \Big(
            a\lb{x,z}  
            +
            b\lb{y,z}  
        \Big)
        + I
        \\
        &=
        \Big(
            a\lb{x,z} + I
        \Big)
        +
        \Big(
            b\lb{y,z} + I 
        \Big)
        \\
        &=
        a\lb{x+I,z+I} + b\lb{y+I,z+I}.
    \end{align*}
    \[
        \lb{x+I,x+I}
        =
        \lb{xx} + I
        =
        0 + I
    \]
\end{proof}

\newcommand{\barphi}{\ensuremath{\overline{\phi}}}

\subsection{Homomorphisms}

There is a natural definition of a Lie algebra homomorphism--- it's a map that respects brackets.

\begin{definition}
    Let $\frkg$ and $\frkh$ be two Lie algebras.
    We say that a map $\phi: \frkg \to \frkh$ is a \defstyle{Lie algebra homomorphism} if it is a linear map for which
    \[
        \phi\Big(\lb{xy}\Big)
        =
        \lb{\phi(x)\phi(y)}
    \]
    for all $x,y \in \frkg$. 
    A \defstyle{Lie algebra isomorphism} is a Lie algebra morphism that is also an isomorphism of vector spaces.
\end{definition}


\begin{definition}
    A \defstyle{representation} of a Lie algebra $\frkg$ is a Lie algebra morphism $\frkg \to \glalg(V)$.
\end{definition}


\begin{theorem}[Lie algebra isomorphism theorems]
    Let $\frkg$ and $\frkh$ be Lie algerbas.
    \begin{enumerate}[label=(\alph*)]
        \item 
            If $\phi: \frkg \to \frkh$ is a homomorphism, then $\frkg / \ker \phi \simeq \im \phi$.
            If $\frki \subseteq \ker \phi$ is an ideal of $\frkg$, there exists a unique homomorphism $\barphi: \frkg/\frki \to \frkh$ that makes the following diagram commute:
            \[
                \begin{tikzcd}
                    \frkg \arrow[r, "\phi"] \arrow[d, "\pi"'] & \frkh \\ 
                                                             \frkg/\frki \arrow[ur, "\barphi"']
                \end{tikzcd}
            \]
        \item 
            if $\frki$ and $\frkj$ are ideals of $\frkg$ such that $\frki \subseteq \frkj$, then $\frkj/\frki$ is an ideal of $\frkg/\frki$ and there is a natural isomorphism
            \[
                (\frkg/\frki)/(\frkj/\frki)
                \simeq
                \frkg/\frkj.
            \]
        \item 
            if $\frki, \frkj$ are ideals of $\frkg$, there is a natural isomorphism
            \[
                (\frki + \frkj)(\frkj)
                \simeq
                \frki/(\frki \cap \frkj).
            \]
    \end{enumerate}
\end{theorem}

\begin{proof}
    \begin{enumerate}[label=(\alph*)]
        \item 
            The map
            \begin{align*}
                \barphi:
                \frkg/\ker\phi
                &\to
                \im \phi
                \\
                x + \ker \phi
                &\mapsto
                \phi(x)
            \end{align*}
            is the desired isomorphism $\frkg / \ker \phi \simeq \im \phi$.
            We verify that it is well defined: let $x + \ker \phi = x' + \ker \phi$.
            Then there exists $k, k' \in \ker \phi$ such that $x + k = x' + k'$, and we have that
            \[
                \phi(x)
                =
                \phi(x + k)
                =
                \phi(x + k')
                =
                \phi(x'),
            \]
            so $\barphi$ is a well-defined function on the cosets in $\frkg / \ker \phi$.

            Next, we check that it respects brackets:
            \begin{align*}
                \barphi
                \Big(
                    \lb{x+\ker\phi, y +\ker\phi}
                \Big)
                &=
                \barphi
                \Big(
                    \lb{xy} + \ker\phi
                \Big)
                \\
                &=
                \phi\Big(\lb{xy}\Big)
                \\
                &=
                \lb{\phi(x)\phi(y)}
                \\
                &=
                \lb{
                    \barphi\Big(x + \ker\phi\Big),
                    \barphi\Big(y + \ker\phi\Big)
                }.
            \end{align*}
            Then, it is a homomorphism.
            To show that it is an isomorphism, we note that it has a trival kernel, trivially:
            \[
                \ker\barphi
                =
                \{x + \ker \phi : x + \ker\phi = \ker \phi\}
                =
                \{0 + \ker \phi\}.
            \]
    \end{enumerate}
\end{proof}

\begin{theorem}
    The adjoint representation $\ad: \frkg \to \glalg(\frkg)$ is a representation of $\frkg$.
\end{theorem}

\begin{proof}
    $\ad$ is evidently linear.
    Next, we just check that it is a homomorphism:
    \begin{align*}
        \lb{\ad_x \ad_y}(z)
        &=
        \Big(\ad_x \ad_y - \ad_y \ad_x\Big)(z) 
        \\
        &=
        \Big(\ad_x \ad_y\Big)(z) - \Big(\ad_y \ad_x\Big)(z) 
        \\
        &=
        \ad_x \lb{yz} - \ad_y \lb{xz}
        \\
        &=
        \lb{x\lb{yz}} - \lb{y\lb{xz}}
        \\
        &=
        \lb{x\lb{yz}} + \lb{y\lb{zx}}
        \\
        &=
        \lb{\lb{xy}z}
        \\
        &=
        \ad_{\lb{xy}}(z).
    \end{align*}
\end{proof}

\begin{corollary}
    Any simple Lie algebra is isomorphic to a linear Lie algebra.
\end{corollary}

\begin{proof}
    Let $\frkg$ be a Lie algebra.
    We have that
    \[
        \ker \ad 
        =
        \Big\{
            x \in \frkg: \ad_x = 0
        \Big\}
        =
        \Big\{
            x \in \frkg: \lb{xy} = 0 \text{ for all } y \in \frkg
        \Big\}
        =
        Z(\frkg).
    \]
    Hence, if $\frkg$ is simple, i.e if $Z(\frkg) = 0$, then $\ad$ has a trivial kernel, so it is an isomorphism. 
\end{proof}

\section{Automorphisms}

\begin{definition}
    A \defstyle{automorphism} of a Lie algebra $\frkg$ is an isomorphism $\frkg \to \frkg$.
\end{definition}

\begin{proposition}
    Let $V$ be a vector space and let $g \in \GL(V)$ be an invertible element of $\End V$.
    Then the map
    \[
        x \mapsto gxg^{-1}
    \]
    is an automorphism of $\glalg(V)$.
\end{proposition}
\begin{proof}
    The aforementioned map is a vector space isomorphism, with explicit inverse
    \[
        x \mapsto g^{-1}xg
    \]
    and it is a homomorphism, as
    \begin{align*}
        g\lb{xy}g^{-1}
        &=
        g\Big(xy - yx\Big)g^{-1}
        \\
        &=
        \Big(gxyg^{-1}\Big)
        -
        \Big(gyxg^{-1}\Big)
        \\
        &=
        \Big(gxg^{-1}gyg^{-1}\Big)
        -
        \Big(gyg^{-1}gxg^{-1}\Big)
        \\
        &=
        \lb{gxg^{-1},gyg^{-1}}.
    \end{align*}
\end{proof}

\section{Solvable and nilpotent Lie algebras}

\subsection{The derived series, solvability}

\begin{definition}
    The \defstyle{derived series} of a Lie algebra $\frkg$ is a sequence of ideals $\frkg^{(0)}, \frkg^{(1)}, \ldots$ defined
    \[
        \begin{cases}
            \frkg^{(0)} \coloneq \frkg \\
            \frkg^{(i)} \coloneq \lb{\frkg^{(i-1)}\frkg^{(i-1)}} \\
        \end{cases}.
    \]
\end{definition}

In other words, $\frkg^{(i)}$ is all those elements of $\frkg$ which can be written as linear combinations of $i$ ``binary trees'' of brackets in $\frkg$.

\begin{definition}
    A Lie algebra $\frkg$ is said to be \defstyle{solvable} if $\frkg^{(n)} = 0$ for some $n$.
\end{definition}

For example, abelian Lie algebras are solvable, whereas simple Lie algebras are never solvable.

\begin{proposition}
    The Lie algebra of upper triangular matrices $\talg_n(\FF)$ is solvable.
\end{proposition}
\begin{proof}
\end{proof}

\subsection{The descending central series, nilpotency}

\begin{definition}
    The \defstyle{descending central series} of a Lie algebra $\frkg$ is a sequence of ideals $\frkg^0, \frkg^1, \ldots$ defined
    \[
        \begin{cases}
            \frkg^0 \coloneq \frkg \\
            \frkg^i \coloneq \lb{\frkg\frkg^{i-1}} \\
        \end{cases}.
    \]
\end{definition}

\begin{definition}
    A Lie algebra $\frkg$ is said to be \defstyle{nilpotent} if $\frkg^n = 0$ for some $n$.
\end{definition}

\begin{proposition}
    All nilpotent Lie algebras are solvable.
\end{proposition}

\begin{definition}
    Let $\frkg$ be a Lie algebra.
    We say that $x \in \frkg$ is \defstyle{ad-nilpotent} if $(\ad_x)^n = 0$ for some $n$.
\end{definition}

\subsection{Engel's theorem}

We will prove \defstyle{Engel's theorem}

\begin{theorem}[Engel]
    Let $\frkg$ be a Lie algebra.
    Then the following are equivalent:
    \begin{enumerate}[label=(\roman*)]
        \item 
            $\frkg$ is nilpotent,
        \item 
            All the elments of $\frkg$ are $\ad$-nilpotent.
    \end{enumerate}
\end{theorem}

\begin{proof}[Proof of Engel's theorem]

\end{proof}

\section{Solutions to exercises}

\begin{exercise}[Humphreys 1.1]
    Verify that $\RR^3$ with the bracket given by the \textit{cross product}
    \[
        [xy]
        \coloneq
        x \times y
    \]
    is a Lie algebra, and write down its structure constants relative to the usual basis of $\RR^3$.
\end{exercise}

Let 
\[
    x
    =
    \begin{pmatrix}
        x_1 \\ x_2 \\ x_3
    \end{pmatrix}, \qquad
    y
    =
    \begin{pmatrix}
        y_1 \\ y_2 \\ y_3
    \end{pmatrix}, \qquad
    z
    =
    \begin{pmatrix}
        z_1 \\ z_2 \\ z_3
    \end{pmatrix}, \qquad
    w
    =
    \begin{pmatrix}
        w_1 \\ w_2 \\ w_3
    \end{pmatrix}.
\]
The cross product is defined
\[
    x \times y
    =
    \begin{pmatrix}
        x_2y_3 - x_3y_2 \\
        x_3y_1 - x_1y_3 \\
        x_1y_2 - x_2y_1 \\
    \end{pmatrix}
\]
Then we directly verify the Lie algebra axioms.

For \ref{ax:LBIsBilinear},
\begin{align*}
    \Big( ax + by \Big) \times z
    &=
    \begin{pmatrix}
        (ax_2 + by_2)z_3 - (ax_3 + by_3)z_2 \\
        (ax_3 + by_3)z_1 - (ax_1 + by_1)z_3 \\
        (ax_1 + by_1)z_2 - (ax_2 + by_2)z_1 \\
    \end{pmatrix}
    \\
    &=
    \begin{pmatrix}
        \Big(ax_2z_3 + by_2z_3\Big) - \Big(ax_3z_2 + by_3z_2\Big) \\
        \Big(ax_3z_1 + by_3z_1\Big) - \Big(ax_1z_3 + by_1z_3\Big) \\
        \Big(ax_1z_2 + by_1z_2\Big) - \Big(ax_2z_1 + by_2z_1\Big) \\
    \end{pmatrix}
    \\
    &=
    \begin{pmatrix}
        a\Big(x_2z_3 - x_3z_2\Big) + b\Big(y_2z_3 + y_3z_2\Big) \\
        a\Big(x_3z_1 - x_1z_3\Big) + b\Big(y_3z_1 + y_1z_3\Big) \\
        a\Big(x_1z_2 - x_2z_1\Big) + b\Big(y_1z_2 + y_2z_1\Big) \\
    \end{pmatrix}
    \\
    &=
    a(x \times z) + b (y \times z).
\end{align*}
And, via an almost identical calculation,
\[
    x \times \Big(ay \times bz\Big) = a(x \times y) + b(x \times z).
\]
Next, we verify \ref{ax:LBNilpotent}
\[
    x \times x
    =
    \begin{pmatrix}
        x_2x_3 - x_3x_2 \\
        x_3x_1 - x_1x_3 \\
        x_1x_2 - x_2x_1 \\
    \end{pmatrix}
    =
    0.
\]
And finally, we verify the Jacobi identity \ref{ax:LBJacobiIdentity}
\[
    x \times x
    =
    \begin{pmatrix}
        x_2x_3 - x_3x_2 \\
        x_3x_1 - x_1x_3 \\
        x_1x_2 - x_2x_1 \\
    \end{pmatrix}
    =
    0.
\]

\end{document}
