% !TEX TS-program = xelatex

\documentclass{article}

% See https://github.com/Jasper-Ty/dotfiles
\usepackage[garamond]{jaspercommon}
\usepackage{bm}

\newcommand*\ullam{\underline{\lambda}}

\title{Ordinary differential equations}
\author{Jasper Ty}
\date{}

\titleauthorhead

\begin{document}

\maketitle

\section*{What is this?}

These are notes I am taking for the class MATH-623, \textit{Ordinary Differential Equations}, at Drexel University, taught by Yixin Guo.

Some notation is changed from her notes, and I try to add as many missing details from proofs as possible.

\tableofcontents

\newpage

\section{Basic theory}

\subsection{Definitions}

\begin{definition}
    Let $J \subseteq \RR$, $U \subseteq \RR$, $\Lambda \subseteq \RR^k$ be open sets, and let $\bff: J \times U \times \Lambda \to \RR^n$ is a smooth function.
    An \defstyle{ordinary differential equation} is an equation of the form 
    \begin{equation}
        \label{eqn:ODE}
        \tag{ODE}
        \dot{\bfx}
        =
        \bff(t, \bfx, \ullam)
    \end{equation}
    where the dot denotes differentiation with respect to the independent variable $t$.
\end{definition}

Morally, the individual parts of an ODE have the following meaning:
\begin{itemize}
    \item[$t$:]
        an independent variable $t \in J$, typically time,
    \item [$\bfx$:]
        a dependent variable $\bfx \in U$,
    \item[$\ullam$:]
        a vector of parameters $\ullam \in \Lambda$, 
    \item[$\bff$:]
        a continuously differentiable function that encodes the (time) evolution of $\bfx$.
\end{itemize}

\begin{definition}
A \defstyle{solution} of (\ref{eqn:ODE}) is a function $\bfF: J_0 \to U$, where $J_0 \subseteq J \subseteq R$, such that 
    \[
        \frac{d}{dt}\bfF(t)
        =
        \bff(t, \bfF(t), \ullam),\qquad 
        \forall t \in J_0
    \]
    i.e, a function for which we can put $\bfx = \bfF(t)$ in (\ref{eqn:ODE}) for all $t \in J_0$.

    The \defstyle{orbit} of the solution $\bfF$ is the set
    \[
        \{
            \bfF(t) \in U: t \in J_0
        \}
        \subseteq \RR^n.
    \]
    This is also called the \defstyle{trajectory}, \defstyle{integral curve}, or \defstyle{solution curve}.
\end{definition}

Evidently, an ODE can have many solutions.
For example, if $J = U = \Lambda = R$ and $f(t,x,\lambda) = \lambda x$, then it's well known that $F(t) = Ce^{\lambda t}$ is a solution to this ODE for all $C \in \RR$.

Fortunately, both real-life and abstract experience tells us that in many cases, the following is a natural way to restrict solutions to an ODE.

\begin{definition}
    An \defstyle{initial-value problem} (IVP) is the system of equations
    \begin{equation}
        \label{eqn:IVP}
        \tag{IVP}
        \begin{cases}
            \dot\bfx = \bff(t, \bfx, \ullam), \\
            \bfx(t_0) = \bfx_0.
        \end{cases}
    \end{equation}
    Namely, it is an (\ref{eqn:ODE}) with additional data $(t_0,\bfx_0) \in J \times U$, encoding an \defstyle{initial value} constraint.

    A \defstyle{solution} of (\ref{eqn:IVP}) is, again a function $\bfF: J_0 \to U$ that solves the underlying ODE, but now subject to the condition that $\bfF(t_0) = \bfx_0$.
\end{definition}

\begin{example}
    The \defstyle{forced Van der Pol} equation is defined to be
    \begin{equation}
        \label{eqn:ForcedVanDerPol}
        \tag{ForcedVanDerPol}
        \begin{cases}
            \dot{x_1} = x_2, \\
            \dot{x_2} = b(1-x_1^2)x_2- \omega^2x_1 + a \cos \Omega t.
        \end{cases}
    \end{equation}
\end{example}

\subsection{Existence and uniqueness}

Let $J, U \subseteq \RR$, and consider the IVP
\begin{equation}
    \label{eqn:IVP1d}
    \tag{IVP-$1$D}
    \frac{dy}{dt} = f(t,y),\qquad
    y(t_0)=y_0
\end{equation}
where $f: J \to U$ is continuously differentiable in $t$ on some open interval containing $t_0$.

Supposing we can integrate (\ref{eqn:IVP1d}) from $t_0$ to a given point $t$,
\begin{align*}
    \int_{t_0}^t \frac{dy}{d\tau}d\tau
    &=
    \int_{t_0}^t f(\tau, y)d\tau, \\
    y(t)-y(t_0)
    &=
    \int_{t_0}^t f(\tau, y)d\tau, \\
    \label{eqn:IVP1dIntegralEqn}
    \tag{$\int$IVP-$1$D}
    y(t)
    &=
    y(t_0) + \int_{t_0}^t f(\tau, y)d\tau, \\
    y(t)
    &=
    y_0 + \int_{t_0}^t f(\tau, y)d\tau.
\end{align*}

If $F$ satisfies (\ref{eqn:IVP1dIntegralEqn}), it satisifes (\ref{eqn:IVP1d}) and vice versa, and this is easily seen using the fundamental theorem of calculus.

\begin{definition}
    The \defstyle{Picard iterates} $y_i$, given the data for (\ref{eqn:IVP1d}), are defined recursively as follows:
    \begin{equation}
        \label{eqn:PicardIter}
        \tag{PicardIter}
        \begin{cases}
            y_0(t) \coloneq y_0, \\
            y_{i+1}(t) \coloneq y_0 + \int_{t_0}^tf\big(\tau,y_i(\tau)\big)d\tau.
        \end{cases}
    \end{equation}
\end{definition}

We have a weak but straightforward estimate on the $y_i$.

\begin{lemma}
    For all $a > 0$, $b > 0$, define $R$ to be the rectangle $[t_0, t_0 + a] \times [y_0-b,y_0+b]$.
    Then if we define
    \[
        M
        \coloneq
        \max_{(t,y) \in R}
        |f(t,y)|, \qquad
        \alpha
        \coloneq
        \min\left\{a,\frac{b}{M}\right\}.
    \]
    Then 
    \begin{equation}
        |y_n(t)-y_0| \leq M(t-t_0)
        \text{ for all }
        t_0 \leq t \leq t + \alpha
    \end{equation}
    for all $n$.
\end{lemma}
\begin{proof}
    We prove this by induction.

    We note that if the statement holds for $n$, then for all $t_0 \leq t \leq \alpha$,
    \begin{align*}
        |y_n(t) - y_0| 
        &\leq 
        M(t-t_0) \\
        &\leq 
        M\alpha \\
        &=
        M \cdot \min\left\{a, \frac{b}{M}\right\} \\
        &=
        \min\left\{\frac{a}{M}, b\right\} \\
        &\leq 
        b.
    \end{align*}
    Hence
    \begin{equation}
        \label{eqn:PicardIteratesStayInRectangle}
        |f\big(t, y_n(t)\big)| \leq M.
    \end{equation}

    \begin{itemize}
        \item[$n=0$:]
            We have that $|y_0(t) - y_0| = 0 \leq M(t-t_0)$ trivially for \textit{all} $t \geq t_0$.
        \item[$n > 0$:]
            Suppose that $|y_n(t) - y_0| \leq M(t-t_0)$ for all $t_0 \leq t \leq t_0 + \alpha$.
            Then
            \begin{align*}
                |y_{n+1}(t)-y_0|
                &=
                \left\lvert
                    \int_{t_0}^t
                    f\big(\tau, y_n(\tau)\big)
                    d\tau
                \right\rvert
                \\
                &\leq
                \int_{t_0}^t
                \underbrace{
                    \left\lvert
                        f\big(\tau, y_n(\tau)\big)
                    \right\rvert
                }_{\text{use (\ref{eqn:PicardIteratesStayInRectangle})}}
                d\tau
                \\
                &\leq
                \int_{t_0}^t
                M
                d\tau
                \\
                &=
                M(t-t_0)
            \end{align*}
            for all $t_0 \leq t \leq t_0 + \alpha$.
            This completes the proof.
    \end{itemize}
\end{proof}

Next we show that the the Picard iterates $y_n$ converge to a function $y$ on $[t_0, t_0 + \alpha]$ which satisfies (\ref{eqn:IVP1d}).

\begin{theorem}
    Let $f(t,y)$ be continuously differentiable in both $t$ and $y$.
    Then
\end{theorem}

\begin{proof}

We have that
\begin{align*}
    y_n(t)
    &=
    y_0(t) + [y_1(t)-y_0] + \cdots + [y_n(t)-y_{n-1}(t)] \\
    &=
    y_0(t)
    +
    \sum_{i=1}^n
    y_i(t)-y_{i-1}(t).
\end{align*}

So, $y_n(t)$ converges if and only if $\sum_{i=1}^n y_i(t)-y_{i-1}(t)$ converges.

\end{proof}

\end{document}
