% !TEX TS-program = xelatex

\documentclass{article}

% See https://github.com/Jasper-Ty/dotfiles
\usepackage[garamond]{jaspercommon}
\usepackage{bm}

\newcommand*\bflam{\bm{\lambda}}

\title{Ordinary differential equations}
\author{Jasper Ty}
\date{}

\titleauthorhead

\begin{document}

\maketitle

\section*{What is this?}

These are notes I am taking for the class MATH-623, \textit{Ordinary Differential Equations}, at Drexel University, taught by Yixin Guo.

\tableofcontents

\newpage

\section{Basic theory}

\subsection{Definitions}

\begin{definition}
    Let $J \subseteq \RR$, $U \subseteq \RR$, $\Lambda \subseteq \RR^k$ be open sets, and let $f: J \times U \times \Lambda \to \RR^n$ is a smooth function.
    An \defstyle{ordinary differential equation (ODE)} is an equation of the form 
    \begin{equation}
        \label{eqn:ODE}
        \dot{\bfx}
        =
        f(t, \bfx, \bm{\lambda})
    \end{equation}
    where the dot denotes differentiation with respect to the independent variable $t$.
\end{definition}

\begin{definition}
    A \defstyle{solution} of an ODE (\ref{eqn:ODE}) is a function $\bfF: J_0 \to U$, where $J_0 \subseteq J \subseteq R$, such that 
    \[
        \frac{d}{dt}\bfF(t)
        =
        f(t, \bfF(t), \bflam),\qquad 
        \forall t \in J_0
    \]
    i.e, a function for which we can put $\bfx = \bfF(t)$ in (\ref{eqn:ODE}).
    The \defstyle{orbit} of the solution $\bfF$ is the set
    \[
        \{
            \bfF(t) \in U: t \in J_0
        \}
        \subseteq \RR^n.
    \]
    This is also called the \defstyle{trajectory}, \defstyle{integral curve}, or \defstyle{solution curve}
\end{definition}

\begin{example}
    The \defstyle{forced Van der Pol} equation is defined
    \begin{equation}
        \label{eqn:ForcedVanDerPol}
        \begin{cases}
            \dot{x_1} = x_2, \\
            \dot{x_2} = b(1-x_1^2)x_2- \omega^2x_1 + a \cos \Omega t.
        \end{cases}
    \end{equation}
\end{example}

\end{document}
