% !TEX TS-program = xelatex
\documentclass{article}

% See https://github.com/Jasper-Ty/dotfiles
\usepackage[garamond]{jaspercommon}

% ≃
\newcommand*\isom{\ensuremath{\simeq}}

% automorphisms
\newcommand*\Aut{\ensuremath{\operatorname{Aut}}}

% endomorphisms
\newcommand*\End{\ensuremath{\operatorname{End}}}

\title{Modules}
\author{Jasper Ty}
\date{}

\titleauthorhead

\begin{document}

\maketitle


\section*{What is this?}

These are notes I am taking about the theory of modules, while taking Jonah Blasiak's Abstract Algebra I class at Drexel.

The content is being taken mostly from Nathan Jacobson's \textit{Basic Algebra I}, Chapter 3, although the class is using Dummit and Foote.

\tableofcontents

\section{Modules over a principal ideal domain}

\subsection{The ring of endomorphisms of an abelian group}

We extend the story of groups arising from considering composition of maps.

\begin{definition}
    The \defstyle{ring of endomorphisms} $\End M$ of an abelian group $(M, +)$ is the set of all homomorphisms $M \to M$, with ring structure given by 
    \[
        (\eta\zeta)(x)
        \coloneq
        \eta\big(\zeta(x)\big)
        ,\qquad
        (\eta + \zeta)(x)
        \coloneq
        \eta(x) + \zeta(x)
    \]
    for all $\eta, \zeta \in \End M$ and $x \in M$.
    From this definition, it follows that $1$ is the map $x \mapsto x$ and $0$ is the map $x \mapsto 0$.
\end{definition}

Paperwork ahead!

\begin{theorem}
    \label{thm:EndMIsARing}
    $\End M$ is a ring.
\end{theorem}
\begin{proof}
    First we prove that $M$'s addition is an abelian group.

    Every statement in this proof holds for all $A, B, C \in \End M$:
    \begin{enumerate}
        \item
            \textsc{\color{Crimson} Addition is closed.}

            In other words, the pointwise sum of two abelian group homomorphisms is again an abelian group homomorphism.

            For all $x,y \in M$,
            \begin{align*}
                (A+B)(x+y)
                &=
                A(x+y) + B(x+y) \\
                &=
                \big(A(x)+A(y)\big) + \big(B(x)+B(y)\big) \\
                &=
                A(x) + B(x) + A(y) + B(y) \\
                &=
                (A+B)(x) + (A+B)(y),
            \end{align*}
            so $A+B \in \End M$.
        \item
            \textsc{\color{Crimson} Addition is associative.}

            For all $x \in M$,
            \begin{align*}
                \big(A + (B+C)\big)(x)
                &= 
                A(x) + (B+C)(x) \\
                &=
                A(x) + \big(B(x) + C(x)\big) \\
                &=
                \big(A(x) + B(x)\big) + C(x) \\
                &=
                (A+B)(x) + C(x) \\
                &=
                \big((A+B)+C\big)(x).
            \end{align*}
            So $A+(B+C) = (A+B)+C$.
        \item 
            \textsc{\color{Crimson} The zero map is the unit of addition.}

            Let $\overline0$ denote the map $x \mapsto 0$.
            Then $\overline0$ is an abelian group homomorphism, as
            \[
                \overline0(x + y)
                =
                0
                =
                0 + 0
                =
                \overline0(x) + \overline0(y)
            \]
            for all $x, y \in M$.

            For all $x \in M$,
            \begin{align*}
                (A + \overline0)(x)
                &=
                A(x) + \overline0(x) \\
                &=
                A(x) + 0 \\
                &=
                A(x).
            \end{align*}
            So, $A + \overline0 = A$, and a nearly-identical calculation shows $\overline0 + A = A$
        \item 
            \textsc{\color{Crimson} Every endomorphism has an additive inverse.}

            Define $-A$ to be the map $x \mapsto -A(x)$.
            Then it is a abelian group homomorphism, as
            \begin{align*}
                (-A)(x+y)
                &=
                -\big(A(x+y)\big) \\
                &=
                -\big(A(x) + A(y)\big) \\
                &=
                \big(-A(y)\big) + \big(-A(x)\big) \\
                &=
                \big(-A(x)\big) + \big(-A(y)\big) \\
                &=
                (-A)(x) + (-A)(y).
            \end{align*}
            for all $x, y \in M$.

            For all $x \in M$,
            \begin{align*}
                \big(A + (-A)\big)(x)
                &=
                A(x) + (-A)(x) \\
                &=
                A(x) + \big(-A(x)\big) \\
                &=
                0,
            \end{align*}
            so $A + (-A) = \overline0$.
        \item 
            \textsc{\color{Crimson} Addition commutes.}

            For all $x \in M$,
            \[
                (A+B)(x)
                =
                A(x) + B(x)
                =
                B(x) + A(x)
                =
                (B+A)(x),
            \]
            which shows that $A+B = B+A$.
    \end{enumerate}
    Next, we show that composition is a monoid.
    This was already done earlier in the book, but we repeat it.
    \begin{enumerate}
        \item
            \textsc{\color{Crimson} Multiplication is closed.}

            In other words, the composition of two abelian group homomorphisms is again an abelian group homomorphism.

            For all $x, y \in M$,
            \begin{align*}
                (AB)(x+y)
                &=
                A\big(B(x+y)\big) \\
                &=
                A\big(B(x) + B(y)\big) \\
                &=
                A\big(B(x)\big) + A\big(B(y)\big) \\
                &=
                (AB)(x) + (AB)(y).
            \end{align*}
            So $AB \in \End M$.
        \item 
            \textsc{\color{Crimson} Multiplication is associative.}

            For all $x \in M$,
            \begin{align*}
                \big(A(BC)\big)(x)
                &= 
                A\big((BC)(x)\big) \\
                &=
                A\Big(B\big(C(x)\big)\Big) \\
                &=
                (AB)\big(C(x)\big) \\
                &=
                \big((AB)C\big)(x),
            \end{align*}
            so $A(BC) = (AB)C$.
        \item 
            \textsc{\color{Crimson} The identity map is the unit of multiplication.}

            Let $\overline1$ denote the map $x \mapsto x$.
            Then $\overline1$ is an abelian group homomorphism, as
            \[
                \overline1(x+y)
                =
                x+y
                =
                \overline1(x) + 1_{\End M}(y)
            \]
            for all $x,y \in M$.

            For all $x \in M$,
            \begin{align*}
                (A\overline1)(x)
                &=
                A\big(\overline1(x)\big) \\
                &=
                A(x)
            \end{align*}
            and
            \begin{align*}
                (\overline1A)(x)
                &=
                \overline1\big(A(x)\big) \\
                &=
                A(x),
            \end{align*}
            so $\overline1A = A\overline1 = A$.
    \end{enumerate}

    Finally, we check that the distributive laws hold

    \begin{enumerate}
        \item
            \textsc{\color{Crimson} Right multiplication distributes over addition.}

            For all $x \in M$,
            \begin{align*}
                \big((A+B)C\big)(x)
                &=
                (A+B)\big(C(x)\big) \\
                &=
                A\big(C(x)\big) + B\big(C(x)\big) \\
                &=
                (AC)(x) + (BC)(x) \\
                &=
                (AC + BC)(x),
            \end{align*}
            so $(A+B)C = AC + BC$.
        \item
            \textsc{\color{Crimson} Left multiplication distributes over addition.}

            For all $x \in M$,
            \begin{align*}
                \big(A(B+C)\big)(x)
                &=
                A\big((B+C)(x)\big) \\
                &=
                A\big(B(x) + C(x)\big) \\
                &=
                A\big(B(x)\big) + A\big(C(x)\big) \\
                &=
                (AB)(x) + AC(x) \\
                &=
                (AB + AC)(x),
            \end{align*}
            so $A(B+C) = AB + AC$.

    \end{enumerate}

    This completes the theorem.
\end{proof}

We have some examples.

\begin{example}
    The ring of endomorphisms of the infinite cyclic group $(\ZZ,+,0)$ is isomorphic to $(\ZZ,+,\cdot,1,0)$, so $\End \isom \ZZ$.
\end{example}

\begin{proof}
    Since $\ZZ$ is generated by $1$, it suffices to know the image of $1$ to determine an endomorphism in $\End \ZZ$.
\end{proof}

\begin{example}
    The ring of endomorphisms of $\ZZ \times \ZZ$ is isomorphic to $M_2(\ZZ)$.
\end{example}

\begin{example}
    The ring of endomorpisms of $\ZZ/n\ZZ$ is isomorphic to $\ZZ/n\ZZ$.
\end{example}

We have the following analogue of Cayley's theorem

\begin{theorem}
    Any ring is isomorphic to the ring of endomorphisms of an abelian group.
\end{theorem}

\begin{proof}
    Let $R$ be a ring, and let $R_+$ denote its additive group.
    For all $a \in R$, define the left multiplication map $a_L: x \mapsto ax$.
    By the distributive law, $a_L(x+y) = a(x+y) = ax + ay = a_L(x) + a_L(y)$, so $a_L \in \End R_+$ for all $a \in R$.

    We will show that the map sending $a \mapsto a_L$ is a ring homomorphism $R \hookrightarrow \End R_+$.
    Let $a, b \in R$.
    For all $x \in R_+$,
    \begin{align*}
        (a + b)_L(x) 
        &= 
        (a+b)x \\
        &= 
        ax + bx \\
        &=
        a_L(x) + b_L(x) \\
        &=
        (a_L + b_L)(x).
    \end{align*}
    and
    \begin{align*}
        (ab)_L(x) 
        &= 
        (ab)(x) \\
        &=
        abx \\
        &=
        a_L\big(b_L(x)\big) \\
        &=
        (a_Lb_L)(x),
    \end{align*}
    so $(a+b)_L = a_L + b_L$, $(ab)_L = a_Lb_L$.
    Finally, we note that $1_L$ is the map $x \mapsto 1x = x$, so $1_L = 1$.

    So the map $a \mapsto a_L$ is a ring homomorphism.
    Denote its image by $R_L$.

    We will show that it is a monomorphism, so that $R \isom R_L \subseteq \End R_+$.

    Suppose $a_L = b_L$.
    Then $a = a_L(1) = b_L(1) = b$, so $a = b$.

    This completes the proof--- by the first isomorphism theorem for rings, $R \isom R_L$.
\end{proof}

We can define the right action $R_R$ as well.
\begin{theorem}
    $R_L = Z(R_R)$ and $R_R = Z(R_L)$
\end{theorem}

\subsubsection*{Exercises}

\begin{exercise}
    Let $G$ be a group (written multiplicatively), and let $F = G^G$ be the set of maps of $G$ into $G$.
    If $\eta, \zeta \in F$ define $\eta\zeta$ in the usual way as the composite $\eta$ following $\zeta$.
    Define $1 \coloneq x \mapsto x$, $0 \coloneq x \mapsto 1$.
    Investigate the properties of the structure $(F,+,\cdot,0,1)$.
\end{exercise}

Omitted.

\begin{exercise}
    Let $M$ be an abelian group.
    Observe that $\Aut M$ is the group of units (invertible elements) of $\End M$.
    Use this to show that $\Aut M$ for the cyclic group of order $n$ is isomorphic to the group of cosets $\overline{m} = m + (n)$ in $\ZZ/(n)$ such that $(m,n) = 1$.
\end{exercise}

Omitted.

\begin{exercise}
    Determine $\Aut M$ for $M = (\ZZ \times \ZZ, +, 0)$.
\end{exercise}

Two-by-two invertible integer matrices, namely the matrices
\[
    \begin{pmatrix}
        a & b \\
        c & d
    \end{pmatrix}
\]
such that $ad - bc \in {+1,-1}$.

\begin{exercise}
    Determine $\End (\QQ,+,0)$.
\end{exercise}

We have that $\End (\QQ,+,0) \isom \QQ$.

First, we note that 
\begin{align*}
    a \cdot \frac{p}{q}
    &=
    \overbrace{
        \frac{p}{q} + \cdots + \frac{p}{q}
    }^{a \text{ times}}
    \\
    &=
    \frac{
        \overbrace{
            p + \cdots + p
        }^{a \text{ times}}
    }{q}.
    \\
    &=
    \frac{ap}{q}
\end{align*}
for all $a \in \ZZ$ and $p/q \in \QQ$.

Then, 
\[
    a\frac{p}{q} = \frac{r}{s} \iff \frac{p}{q} = \frac{r}{sa},
\]
as both equalities hold if and only if
\[
    sap = rq.
\]
Finally, we have that
\[
    \phi\left(\frac{p}{q}\right)
    =
    \frac{p}{q} \phi(1)
\]
for any homomorphism $\phi: \QQ \to \QQ$.

\begin{exercise}
    In several cases we have considered, we have $\End(R,+,0) \isom R$ for a ring $R$.
    Does this hold in general? Does it hold if $R$ is a field?
\end{exercise}

I don't think so?

\subsection{Left and right modules}

\begin{definition}
    Let $R$ be a ring.
    A \defstyle{left $R$-module} is an abelian group $M$ together with a scaling map $\cdot : R \times M \mapsto M$ such that
    \begin{enumerate}
        \item
            $a(x+y) = ax + ay$,
        \item 
            $(a+b)x = ax + bx$,
        \item 
            $(ab)x = a(bx)$,
        \item 
            $1x = x$,
    \end{enumerate}
    for all $x, y \in M$ and $a,b \in R$.
\end{definition}

\begin{proposition}
    Any abelian group $M$ is a $\ZZ$-module with the scaling map
    \[
        ax
        \coloneq
        \begin{cases}
            \underbrace{x + \cdots + x}_{a \text{ times}}, & a > 0 \\
            \underbrace{(-x) + \cdots + (-x)}_{-a \text{ times}}, & a < 0 \\
            0. & a = 0
        \end{cases}
    \]
    where $a \in \ZZ$ and $x \in M$.
\end{proposition}

\subsection{Fundamental concepts and results}

\begin{definition}
    Fix a ring $R$.
    A \defstyle{$R$-module homomorphism} between the two $R$-modules $M$ and $N$ is a map $\eta: M \to N$ such that $\eta$ is a group homomorphism of $M$ and $N$'s additive groups, and it ``commutes with scaling'': $\eta(ax) = a\eta(x)$ for all $a \in R$ and $x \in M$.
\end{definition}

\end{document}
