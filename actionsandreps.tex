\documentclass{article}

% See https://github.com/Jasper-Ty/dotfiles
\usepackage[garamond]{jaspercommon}

\DeclareMathOperator{\id}{id}

\title{Group actions and representations}
\author{Jasper Ty}
\date{}

\titleauthorhead

\begin{document}

\maketitle

This are notes I took while going through Chapter 4 of Darij Grinberg's ``Introduction to the symmetric group algebra'' \cite{DarijSGA}.

\tableofcontents

\section{Group actions and \texorpdfstring{$G$}{G}-sets}

\subsection{Introduction}

We start with one of the most famous applications of group actions.

\begin{theorem}[Cayley's theorem]\label{thm:cayley}
    Every group is isomorphic to a subgroup of some symmetric group.
\end{theorem}
\begin{proof}[Proof (sketch)]
    Fix a group $G$. For all group elements $g \in G$, define the map
    \begin{align*}
        L_g: G &\to G \\
        k &\mapsto gk,
    \end{align*}
    which is easily verified to be an element of $S_G$.
    It is also easily shown that $L_{gh} = L_g \circ L_h$, implying that the map $g \mapsto L_g$ is homomorphic. 
    Then $G$ is isomorphic to the image of the aforementioned map, a subgroup of $S_G$.
\end{proof}

This is, at least, the standard proof.
If you're cheating and already more or less know the idea of the proof ahead of time, an even terser proof sketch would go:

\begin{proof}[Proof (sketchier)]
    \textit{$G$ acts on itself via left multiplication}.
\end{proof}

In fact, $G$ can act on \textit{far, far more} than just itself.

\begin{definition}[Left group actions]
    Let $G$ be a group, and let $X$ be a set.
    We call a binary operation
    \[
        \ast : G \times X \to X
    \]
    a \textit{left action of $G$ on $X$}, or a \textit{left $G$-action of $X$} whenever it satisfies
    \begin{align*}
        g \ast (h \ast x) &= (gh) \ast x \\
        1_G \ast x &= x
    \end{align*}
for any $g,h \in G$, $x \in X$.
\end{definition}

\begin{definition}[$G$-sets]
    A \textit{left $G$-set} is a set $X$ equipped with a left $G$-action of it.
\end{definition}

We can define right actions and right $G$-sets analogously.
But we won't.

\begin{convention}
    All actions are left actions unless explicitly noted otherwise.
\end{convention}

Now, we note that in Cayley's theorem, we expressed our group action as a map into $S_G$.
This gives us an equivalent formulation of $G$-actions and $G$-sets.

\begin{definition}[$G$-actions, curried form]
    A left $G$-action on $X$ is a homomorphism
    \begin{align*}
        \tau: G &\to S_X \\
        g &\mapsto \tau_g.
    \end{align*}
\end{definition}

We may go between the two different definitions by the relationships
\[
    g \ast x \coloneq \tau_g(x) \qquad \tau_g \coloneq (x \mapsto g \ast x)
\]
going between both curried and uncurried forms.


\subsection{Examples}

For brevity, I omit quantifiers--- any formula written here is implicitly meant to be quantified \textit{universally} over its free variables.

\begin{example}
    Let $G$ be a group. 
    Then we have the following $G$-actions on $G$
    \begin{enumerate}[label=(\alph*)]
        \item The \textit{left regular $G$-action};
            \[
                g \ast h := gh.
            \]
        \item The \textit{inverse right regular $G$-action};
            \[
                g \ast h := hg^{-1}.
            \]
        \item The \textit{conjugation $G$-action};
            \[
                g \ast h := ghg^{-1}.
            \]
    \end{enumerate}
\end{example}

\begin{example}
    Let $G$ be a group and let $X$ be any set.
    The \textit{trivial $G$-action on $X$} is defined by
    \[
        g \ast x = x.
    \]
\end{example}

\begin{example}
    Let $X$ be any set.
    The \textit{natural} action of $S_X$ on $X$ is defined by
    \[
        g \ast x = g(x).
    \]
\end{example}

\begin{example}\label{ex:cosetgset}
    Let $G$ be a group and let $H \leq G$.
    Define
    \[
        G / H := \{\text{left cosets of }H\text{ in }G\} = \{gH : g \in G\}.
    \]
    This set, as we know, is a group (with the Minkowski product) if and only if $H$ is normal.
    Nevertheless, it can always be given the structure of a $G$-set, by defining
    \[
        g \ast (uH) := (gu)H.
    \]
\end{example}

\subsection{Symmetric group actions}

The following two definitions have a particularly annoying mess of clashing conventions built on top of them, one which my advisor regularly asks me to be careful about.

\begin{definition}[A Tale of Two Actions]
    Let $n \in \NN$.
    Let $X$ be a set.
    Consider the following actions on $X^n$.
    \begin{enumerate}[label=(\alph*)]
        \item We have a $S_X$ action given by
            \[
                g \ast (x_1,\ldots,x_n) := (g(x_1),\ldots,g(x_n)).
            \]
        \item We have a $S_n$ action given by
            \[
                g \ast (x_1,\ldots,x_n) := (x_{g(1)},\ldots,x_{g(n)}).
            \]
    \end{enumerate}
    The former sends \textit{values to values}, and is called the \textit{entrywise action}, or the \textit{action on entries}.
    The latter sends \textit{positions to positions}, and is called the \textit{place permutation action}, or the \textit{action on places}.
\end{definition}

It gets particularly confusing (to me!) when $X$ itself is the set $\{1, \ldots, n\}$, then both the action on entries and places are $S_n$-actions!

\subsection{\texorpdfstring{$G$}{G}-set morphisms}

$G$ sets form a category (a \href{https://en.wikipedia.org/wiki/Groupoid}{groupoid}, in particular), whose morphisms respect $G$-actions in the following way:

\begin{definition}
    Let $X$ and $Y$ be two $G$-sets.
    Then a \textit{$G$-set morphism} or a \textit{$G$-equivariant} map is a function $f: X \to Y$ such that
    \[
        f(g \ast x) = g \ast f(x)
    \]
    for any $g \in G$, $x \in X$.
\end{definition}

\begin{example}
    Let $A$ be any set.
    Equip $A$ with the trivial $S_n$-action, and $A^n$ with the place-permuting $S_n$-action.
    The map 
    \begin{align*}
        f: A &\to A^n \\
        a &\mapsto \left(\underbrace{a,a,...,a}_{n\text{ times}}\right)
    \end{align*}
    is then a $S_n$-set morphism, since if $g \in S_n$,
    \[
        f(g \ast a) = f(a) = \left(\underbrace{a,a,...,a}_{n\text{ times}}\right)
    \]
    and
    \[
        g \ast f(a) = g \ast \left(\underbrace{a,a,...,a}_{n\text{ times}}\right) = \left(\underbrace{a,a,...,a}_{n\text{ times}}\right).
    \]
\end{example}

More generally,

\begin{example}
    Let $X$ and $Y$ be two $G$-sets, equipped with the trivial $G$-actions.
    Then \textit{any} map $f: X \to Y$ is $G$-equivariant.
\end{example}

\begin{proposition}
    If $f$ is a $G$-set morphism and its inverse $f^{-1}$ exists, then $f^{-1}$ is a $G$-set morphism.
\end{proposition}

\begin{proof}
    Let $f: X \to Y$.
    That $f^{-1}$ is an inverse of $f$ means that
    \[
        f^{-1} \circ f = f \circ f^{-1} = \id.
    \]
    That $f$ is a $G$-set morphism means that, given the curried forms $\tau: X \to S_X$ and $\phi: Y \to S_Y$ of $X$ and $Y$'s $G$-actions, for all elements $g \in G$
    \[
        \phi_g \circ f = f \circ \tau_g.
    \]
    Then
    \[
        f^{-1} \circ (\phi_g \circ f) \circ f^{-1} = f^{-1} \circ (f \circ \tau_g) \circ f^{-1}
    \]
    so
    \[
        f^{-1} \circ \phi_g = \tau_g \circ f^{-1}.
    \]
    which shows that $f^{-1}$ is a $G$-set morphism.
\end{proof}

\begin{definition}
    A \textit{$G$-set isomorphism} is a $G$-set morphism that is invertible, i.e, it is a $G$-set morphism $f$ for which there exists another $G$-set morphism $f^{-1}$ such that
    \[
        f \circ f^{-1} = f^{-1} \circ f = \id.
    \]
    If $X$ and $Y$ are two $G$-sets that have a $G$-set isomorphism between them, we say that $X$ and $Y$ are \textit{isomorphic as $G$-sets}, and write $X \simeq Y$.
\end{definition}

\begin{example}
    $G/G$, as defined in \ref{ex:cosetgset} is isomorphic to $\{1\}$.
\end{example}

\begin{example}
    $G/\{1_G\} \sim G$.
\end{example}

\begin{example}
    Consider the $S_n$-action on $\scrP([n])$, and consider the place-permutation action of $S_n$ on $\mathbf{2}^n$.
    Then $\scrP([n]) \simeq \mathbf{2}^n$.
\end{example}
\begin{proof}
    Consider the bijection given by sending each element of $\scrP([n])$ to its corresponding indicator function in $\mathbf{2}^n$.
\end{proof}

\subsection{Combining \texorpdfstring{$G$}{G}-sets}

\begin{definition}
    The \textit{Cartesian product of $G$-sets}
\end{definition}


\section{Linear group actions and representations of \texorpdfstring{$G$}{G}}

\subsection{Linear actions, aka representations}

\begin{definition}
    Let $V$ be a $\mathbf{k}$-module, and let $G$ be a group.
    \begin{enumerate}[label=(\alph*)]
        \item A $G$-action on $V$ is \textit{$\mathbf{k}$-linear} if the map $\tau_g$ is $\mathbf{k}$-linear for all $g \in G$.
    \end{enumerate}
\end{definition}

\section{Modules over rings}
\subsection{Definitions}
\begin{definition}

\end{definition}

\begin{theorem}
    A left $\mathbf{k}[G]$-module $V$
\end{theorem}




\begin{thebibliography}{999999}
    \raggedright\footnotesize

    \bibitem[GrinbergSGA]{DarijSGA}
    Darij Grinberg, \textit{An introduction to the symmetric group algebra}, \url{http://www.cip.ifi.lmu.de/~grinberg/t/24s/sga.pdf}

\end{thebibliography}

\end{document}
