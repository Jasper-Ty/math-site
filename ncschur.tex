% !TEX TS-program = xelatex

\documentclass{article}

% Formal Power Series
\newcommand{\fps}[1]{\left[\kern-1.4pt\left[ {#1} \right]\kern-1.4pt\right]}
% Formal Power Series
\newcommand{\ncfps}[1]{\left\langle\kern-1.4pt\left\langle {#1} \right\rangle\kern-1.4pt\right\rangle}

\newcommand{\ip}[1]{
    \left\langle
        {#1}
    \right\rangle
}

\newcommand*\colword{\operatorname{\mathsf{colword}}}
\newcommand*\frkJ{\mathfrak{J}}
\newcommand*\SSYT{\operatorname{SSYT}}
\newcommand*\sgn{\operatorname{sgn}}
\newcommand*\Des{\operatorname{Des}}

% See https://github.com/Jasper-Ty/dotfiles
\usepackage[garamond]{jaspercommon}

\barenv[][section]{theorem}{Theorem}
\barenv[bartheorem]{proposition}{Proposition}
\barenv[bartheorem]{corollary}{Corollary}
\barenv[bartheorem]{lemma}{Lemma}
\barenv[bartheorem]{definition}{Definition}
\barenv[bartheorem]{convention}{Convention}

\title{Noncommutative Schur functions}
\author{Jasper Ty}
\date{}

\titleauthorhead

\begin{document}

\maketitle

\section*{What is this?}

This is (going to be) an ``infinite napkin''-type set of notes I am taking about the Fomin-Greene theory of noncommutative Schur functions.

Note that this is distinct from the theory of the ring called $\mathrm{NCSym}$, in which there exists structural analogues of monomial, elementary, homogeneous, power, and Schur functions.
I am not currently aware of any connection between these two theories.

\tableofcontents

\newpage

\section{Ideals and words}

Let $\bfu = (u_1,\ldots,u_N)$ be a collection of noncommuting variables.
Let $\langle \bfu \rangle$ be the free semigroup on the generators $\bfu$.
Then, let $\calU = \ZZ\langle\bfu\rangle$ denote the corresponding semigroup ring--- the free associative ring generated by $\bfu$.

We will denote by $\calU^\ast$ the $\ZZ$-module spanned by words in the alphabet $\{1,\ldots,N\}$.

We will have a fundamental pairing $\ip{\wc,\wc}$ given by making noncommutative monomials dual to words.

Now, if $I$ is an ideal of $\calU$, we define $I^\perp$ by
\[
    I^\perp
    \coloneq
    \{
        \gamma \in \calU^\ast
        \mid
        \ip{I, \gamma} = 0
    \}.
\]

Let $\bfx = (x_1,\ldots,x_n)$ be a collection of commuting variables.
We will define $\calU[\bfx] \coloneq \calU \otimes_{\ZZ} \ZZ[\bfx]$, and for all ideals $I$ of $\calU$

\section{
    Noncommutative elementary and homogeneous symmetric functions
}

\begin{definition}
    The \defstyle{noncommutative elementary symmetric function} $e_k(\bfu)$ is defined to be
    \begin{equation}
        \label{eqn:ncElementaryDef}
        e_k(\bfu)
        \coloneq
        \underbrace{
            \sum_{i_1 > i_2 > \cdots > i_k}
        }_{\text{(decreasing!)}}
        u_{i_1} u_{i_2} \cdots u_{i_k}. 
    \end{equation}
    The \defstyle{noncommutative homogeneous symmetric function} $h_k(\bfu)$ is defined to be
    \begin{equation}
        \label{eqn:ncHomogeneousDef}
        h_k(\bfu)
        \coloneq
        \underbrace{
            \sum_{i_1 \leq i_2 \leq \cdots \leq i_k}
        }_{\text{(increasing!)}}
        u_{i_1} u_{i_2} \cdots u_{i_k}. 
    \end{equation}
    We will have the convention that $e_0(\bfu) = h_0(\bfu) = 1$, and that $e_k(\bfu) = h_k(\bfu) = 0$ if $k < 0$.
\end{definition}

In symmetric function theory, the elementary and homogeneous symmetric functions can be seen as generating functions for semistandard Young tableaux whose shape is a single column and a single row respectively.

The same idea works here--- except now one takes the \textit{column word} of a tableau.

\subsection{
    Newton's identities
}

We define noncommutative analogues of standard generating functions for the elementary and homogeneous symmetric functions.

\begin{definition}
    Define the following functions:
    \begin{equation}
        \label{eqn:colGenFunc}
        E(x)
        \coloneq
        \sum_{k=0}^N
        x^k e_k(\bfu)
        =
        \prod_{i=N}^1
        (1+xu_i)
    \end{equation}
    and
    \begin{equation}
        \label{eqn:rowGenFunc}
        H(x)
        \coloneq
        \sum_{k=0}^\infty
        x^k h_k(\bfu)
        =
        \prod_{i=1}^N
        (1-xu_i)^{-1}.
    \end{equation}
    in the ring $\calU[x]$.
\end{definition}

An immediate consequence of this definition is a noncommutative analogue of Newton's identities.

\begin{proposition}
    [Noncommutative Newton-Girard formulas]
    We have
    \begin{equation}
        \label{eqn:ncNewtonGirardFPS}
        E(x)H(-x) = H(x)E(-x) = 1.
    \end{equation}
    In particular,
    \[
        \sum_{k=0}^n (-1)^ke_k(\bfu)h_{n-k}(\bfu) = 0
    \]
    and
    \[
        \sum_{k=0}^n (-1)^kh_k(\bfu)e_{n-k}(\bfu) = 0
    \]
    for all $n$.
\end{proposition}

\begin{proof}
    Putting together (\ref{eqn:colGenFunc}) and (\ref{eqn:rowGenFunc}) immediately gives us (\ref{eqn:ncNewtonGirardFPS})
    \begin{align*}
        E(x)H(-x)
        &=
        \left[
            \prod_{i=N}^1
            (1+xu_i)
        \right]
        \left[
            \prod_{i=1}^N
            (1+xu_i)^{-1}
        \right]
        \\
        &=
        \left[
            \prod_{i=N}^1
            (1+xu_i)
        \right]
        \underbrace{
            \left[
                \prod_{i=N}^1
                (1+xu_i)
            \right]^{-1}
        }_{\text{note reversal of product order!}}
        \\
        &=
        1.
    \end{align*}
    And $H(x)E(-x)=1$ is proved exactly the same way.
\end{proof}


\begin{corollary}
    \label{corr:ECommutesIffHCommutes}
    Let $I$ be an ideal of $\calU$.
    Then $E(x)E(y) \equiv_{I[x,y]} E(y)E(x)$ if and only if $H(x)H(y) \equiv_{I[x,y]} H(y)H(x)$.
\end{corollary}

\begin{proof}
    Suppose $E(x)E(y) = E(y)E(x)$ for all commuting $x,y$.
    Then $H(x)H(y) = E(-x)^{-1}E(-y)^{-1} = E(-y)^{-1}E(-x)^{-1} = H(y)H(x)$.
    The reverse implication is proved identically.
\end{proof}

\subsection{
    When do the elementaries commute?
}

\begin{lemma}
    \label{lem:ElementariesCommuteIffECommutes}
    Let $I$ be an ideal of $\calU$.
    The following are equivalent:
    \begin{enumerate}[label=(\alph*)]
        \item 
            $E(x)E(y) \equiv_{I[x,y]} E(y)E(x)$.
        \item 
            $e_k(\bfu)e_j(\bfu) \equiv_I e_j(\bfu)e_k(\bfu)$ for all $j,k$. 
    \end{enumerate}
\end{lemma}

\begin{proof}
    Expand and compare coefficients.
\end{proof}

\begin{lemma}
    Let $I$ be an ideal of $\calU$.
    The following are equivalent:
    \begin{enumerate}[label=(\alph*)]
        \item 
            $e_k(\bfu)e_j(\bfu) \equiv_I e_j(\bfu)e_k(\bfu)$ for all $j,k$. 
        \item 
            $h_k(\bfu)h_j(\bfu) \equiv_I h_j(\bfu)h_k(\bfu)$ for all $j,k$.
    \end{enumerate}
\end{lemma}

\begin{proof}
    Combine Corollary \ref{corr:ECommutesIffHCommutes} and Lemma \ref{lem:ElementariesCommuteIffECommutes}.
\end{proof}


\begin{definition}
    We define the ideal $I_C$ to be the ideal consisting of exactly the elements
    \begin{align}
        &
        u_b^2u_a + u_au_bu_a - u_bu_au_b - u_bu_a^2 
        &
        (a<b),
        \\
        &
        u_bu_cu_a + u_au_cu_b - u_bu_au_c - u_cu_au_b
        &
        (a<b<c),
        \\
        &
        u_cu_bu_cu_a + u_bu_cu_au_c - u_cu_bu_au_c - u_bu_c^2u_a
        &
        (a<b<c).
    \end{align}
\end{definition}

Compactly, these are the relations
\[
    [u_au_b]u_a \equiv u_b[u_au_b], \quad
    [u_au_c]u_b \equiv u_b[u_au_c], \quad
    [u_bu_c][u_au_c] \equiv 0.
\]
for all $a<b<c$.

We now come to the key theorem about $I_C$--- namely that it is the smallest ideal which allows the noncommutative elementaries to commute.

We will follow A.N Kirillov's proof \cite[Theorem 2.26]{K16}.
Blasiak and Fomin also have a proof carried out in much higher generality \cite{BF18}.

\begin{theorem}
    \label{thm:ICImpliesEsCommute}
    If $I \supseteq I_C$, then $e_k(\bfu)e_j(\bfu) \equiv_I e_j(\bfu)e_k(\bfu)$ for all $j,k$. 
\end{theorem}

\begin{proof}
    First we show that, in $I_C$, the elementaries commute.
    Define $E_n(x) = $
    \begin{align*}
    \end{align*}
\end{proof}

\subsection{
    The map \texorpdfstring{$\Psi_I$}{Psi\_I}
}

\begin{theorem}
    [Fundamental theorem of symmetric functions]
    \label{thm:FundThmSymFuncs}
    Let $\Lambda(\bfx)$ denote the ring of symmetric polynomials in the commuting variables $\bfx = (x_1,\ldots,x_n)$.
    Then the elementary symmetric functions in $\bfx$ are algebraically independent, and moreover
    \[
        \Lambda(\bfx)
        \simeq
        \QQ[e_1(\bfx), e_2(\bfx), \ldots, e_n(\bfx)].
    \]
\end{theorem}

\begin{proof}
    See Theorem 7.4.4 in \cite{EC2}.
    One can prove this via the \textit{Gale-Ryser} theorem.
\end{proof}

\begin{corollary}
    If $I$ contains $I_C$, then the map
    \begin{align*}
        \Psi_I:
        \Lambda_n(\bfx)
        &\to
        \calU/I
        \\
        e_k(\bfx)
        &\mapsto
        e_k(\bfu)
    \end{align*}
    extends to a ring homomorphism.
\end{corollary}



\begin{proof}
    Combine Theorems \ref{thm:FundThmSymFuncs} and \ref{thm:ICImpliesEsCommute}.
\end{proof}



\section{
    Noncommutative Schur functions
}

\begin{definition}
    Let $I \supseteq I_C$.
    The \defstyle{noncommutative Schur function} $\frkJ(\bfu) \in \calU/I$ is defined to be
    \[
        \frkJ_\lambda(\bfu)
        \coloneq
        \sum_{\pi \in S_{t}}
        \sgn(\pi)
        e_{\lambda^\top_1+\pi(1)-1}(\bfu)
        e_{\lambda^\top_2+\pi(2)-2}(\bfu)
        \cdots
        e_{\lambda^\top_{t}+\pi(t)-t}(\bfu),
    \]
    where $t = \lambda_1$ is the number of parts of $\lambda^\top$.
    Alternatively, since the $h$'s commute whenever the $e$'s do,
    \[
        \frkJ_\lambda(\bfu)
        \coloneq
        \sum_{\pi \in S_{t}}
        \sgn(\pi)
        h_{\lambda_1+\pi(1)-1}(\bfu)
        h_{\lambda_2+\pi(2)-2}(\bfu)
        \cdots
        h_{\lambda_t+\pi(t)-t}(\bfu).
    \]
\end{definition}

The first definition is based on the \defstyle{Kostka-Naegelsbach identity}
\[
    s_\lambda(\bfx)
    =
    \det\big(e_{\lambda^\top_i+j-i}(\bfx)\big)_{i,j=1}^n,
\]
and the second is based on the \defstyle{Jacobi-Trudi identity}
\[
    s_\lambda(\bfx)
    =
    \det\big(h_{\lambda_i+j-i}(\bfx)\big)_{i,j=1}^n.
\]
Since these are purely polynomials of elementary symmetric and complete homogeneous polynomials, one sees the following
\begin{definition}
    If $I \supseteq I_C$, then
    \[
        \Psi_I\big(s_\lambda(\bfx)\big)
        \equiv
        \frkJ_\lambda(\bfu)
        \mod I.
    \]
\end{definition}
\begin{proof}
    \begin{align*}
        \Psi_I\big(s_\lambda(\bfx)\big)
        &=
        \Psi_I \left(
            \det\big(e_{\lambda^\top_i+j-i}(\bfx)\big)_{i,j=1}^n
        \right)
        \\
        &=
        \Psi_I \left(
            \sum_{\pi \in S_n}
            \sgn(\pi)
            h_{\pi_1+\pi(1)-1}(\bfx)
            \cdots
            h_{\pi_n+\pi(n)-n}(\bfx)
        \right)
        \\
        &\equiv
        \sum_{\pi \in S_n}
        \sgn(\pi)
        h_{\pi_1+\pi(1)-1}(\bfu)
        \cdots
        h_{\pi_n+\pi(n)-n}(\bfu) \mod I
        \\
        &\equiv
        \frkJ_\lambda(\bfu) \mod I.
    \end{align*}
\end{proof}



\begin{theorem}
    If $I$ contains $I_C$, then for all $\gamma \in I_C^\perp$,
    \[
        \ip{
            \prod_{i=1}^m
            \prod_{j=n}^1
            (1 + x_iu_j),
            \gamma
        }
        =
        \sum_\lambda
        s_\lambda(\bfx)
        \ip{
        \frkJ_{\lambda^\top}(\bfu),\gamma
        }.
    \]
\end{theorem}

\begin{proof}
\end{proof}

\begin{theorem}
    [\cite{FG98}, \cite{BF16}]
    In the ideal $I_\varnothing$,
    \[
        \frkJ_\lambda(\bfu)
        \coloneq
        \sum_{T \in \SSYT(\lambda;N)}
        \bfu^{\colword T}.
    \]
\end{theorem}

\subsection{Cauchy kernel}

\begin{definition}
    Let $\bfx = (x_1,x_2\ldots)$ be a countable collection of commuting variables.
\end{definition}

\section{Applications}

\subsection{Recovering known results in the plactic algebra}

\begin{theorem}
    [Littlewood-Richardson rule]
\end{theorem}

\subsection{Stanley symmetric functions via the nilCoxeter algebra}

The connection between Schubert polynomials and the nilCoxeter ideal was first explored by Richard Stanley and 

\begin{definition}
    The \defstyle{nilCoxeter} ideal
\end{definition}

\subsection{LLT polynomials via the algebra of Ribbon Schur operators}

\section{Linear programming}

Consider the positive cones $\calU_{\geq 0}$ and $\calU^\ast_{\geq 0}$.


\section{Algebras of operators}

\begin{definition}
    A \defstyle{combinatorial representation} of $\calU/I$ is
\end{definition}

\section{Switchboards}

\section{Appendix}

\subsection{Formal power series}

\begin{corollary}
\end{corollary}

\begin{proposition}

\end{proposition}

\subsection{Gessel's fundamental quasisymmetric function}

\begin{definition}
    Let ${\sf w} \in \calU^\ast$ be a word.
    We define the \defstyle{fundamental quasisymmetric function} $Q_{\Des(\sf w)}$ by
    \[
        Q_{\Des(\sf w)}
        \coloneq
        \sum_{
            \substack{
                1 \leq i_1 \leq \cdots \leq i_n \\
                j \in \Des({\sf w}) \implies i_j < i_{j+1}
            }
        }
        x_{i_1} \cdots x_{i_n}.
    \]
\end{definition}

\subsection{The Edelman-Greene correspondence}

\begin{thebibliography}{999999}
    \raggedright\footnotesize

    \bibitem[EC2]{EC2}
    Richard P. Stanley, 
    \textit{Enumerative Combinatorics. Volume 2}, 
    Cambridge University Press 2023.

    \bibitem[FG98]{FG98}
    Sergey Fomin and Curtis Greene, 
    \textit{Noncommutative Schur functions and their applications}, 
    Discrete Math. \textbf{193} (1998), 179-200.

    \bibitem[BF16]{BF16}
    Jonah Blasiak and Sergey Fomin, 
    \textit{Noncommutative Schur functions, switchboards, and Schur positivity},
    Sel. Math. \textbf{23} (2017), 727-766.

    Also available as \arxiv{1510.00657}.

    \bibitem[BF18]{BF18}
    Jonah Blasiak and Sergey Fomin, 
    \textit{Rules of Three for commutation relations},
    J. Algebra. \textbf{500} (2018), 193-220.

    Also available as \arxiv{1608.05042}.

    \bibitem[A15]{A15}
    Sami Assaf,
    \textit{Dual equivalence graphs I: A new paradigm for Schur positivity},
    Forum. Math. Sigma \textbf{3} (2015), e12.

    Also available as \arxiv{1506.03798}.

    \bibitem[K16]{K16}
    Alexandre Kirillov,
    \textit{Notes on Schubert, Grothendieck, and Key polynomials},
    SIGMA \textbf{12} (2016)

    Also available as \arxiv{1501.07337}.

    \bibitem[L04]{L04}
    Thomas Lam,
    \textit{Ribbon Schur operators},
    European J. Combin. \textbf{29} (2008), 343-359.

    Also available as \arxiv{math/0409463}.

    \bibitem[FS91]{FS91}
    Sergey Fomin and Richard P. Stanley, 
    \textit{Schubert Polynomials and the NilCoxeter Algebra},
    Adv. Math. \textbf{103} (1994), 196-207.

    \bibitem[M91]{M91}
    Ian G. Macdonald, 
    \textit{Notes on Schubert Polynomials}, 
    LACIM, 1991.

\end{thebibliography}

\end{document}
