% !TEX TS-program = xelatex

\documentclass{article}

% Formal Power Series
\newcommand{\fps}[1]{\left[\kern-1.4pt\left[ {#1} \right]\kern-1.4pt\right]}
% Formal Power Series
\newcommand{\ncfps}[1]{\left\langle\kern-1.4pt\left\langle {#1} \right\rangle\kern-1.4pt\right\rangle}

\newcommand{\ip}[1]{
    \left\langle
        {#1}
    \right\rangle
}

\newcommand*\WW{\mathbb{W}}
\newcommand*\colword{\operatorname{\mathsf{colword}}}
\newcommand*\frkJ{\mathfrak{J}}
\newcommand*\SSYT{\operatorname{SSYT}}
\newcommand*\sgn{\operatorname{sgn}}
\newcommand*\Des{\operatorname{Des}}

% See https://github.com/Jasper-Ty/dotfiles
\usepackage[garamond]{jaspercommon}

\usepackage{lscape}

\barenv[][section]{theorem}{Theorem}
\barenv[bartheorem]{proposition}{Proposition}
\barenv[bartheorem]{corollary}{Corollary}
\barenv[bartheorem]{lemma}{Lemma}
\barenv[bartheorem]{definition}{Definition}
\barenv[bartheorem]{convention}{Convention}

\title{Noncommutative Schur functions}
\author{Jasper Ty}
\date{}

\titleauthorhead

\begin{document}

\maketitle

\section*{What is this?}

This is (going to be) an ``infinite napkin''-type set of notes I am taking about the Fomin-Greene theory of noncommutative Schur functions.

Note that this is distinct from the theory of the ring called $\mathrm{NCSym}$, in which there exists structural analogues of monomial, elementary, homogeneous, power, and Schur functions.
I am not currently aware of any connection between these two theories.

\tableofcontents

\newpage

\section{Some formalism}

\begin{convention}
    Rings are unital.
    Ideals are two-sided.
\end{convention}

We will set up a minimal framework for stating theorems of the form
\begin{quote}
    Suppose the relations \rule{1cm}{0.15mm} hold among the variables \rule{1cm}{0.15mm}, then \rule{1cm}{0.15mm}.
\end{quote}
where we will work with two obviously isomorphic objects:
\begin{itemize}
    \item
        noncommutative monomials, and
    \item
        words,
\end{itemize}
which sacrifices only a little bit of generality for plenty of conceptual clarity.

\subsection{Noncommutative monomials}

We begin our construction from the ``variables'' side.

\begin{definition}
    \label{def:nc}
    Let $\bfu = (u_1,\ldots,u_N)$ be some list of elements, which we will call our \defstyle{noncommuting variables}.
    \begin{enumerate}[label=(\alph*)]
        \item 
            \label{def:ncMonomials}
            The \defstyle{monoid of noncommutative monomials} $\langle \bfu \rangle$ is the free monoid on the generating set $\bfu$.
        \item 
            \label{def:FreeAssociativeRing}
            The \defstyle{free associative ring} $\calU$ is the monoid ring $\ZZ\langle\bfu\rangle$. 
    \end{enumerate}
\end{definition}

As one would expect, $\calU$ is simply the associative unital $\ZZ$-algebra whose basis consists of elements
\[
    u_{i_1} \cdots u_{i_k} \in \langle \bfu \rangle
\]
for some list of integers $(i_1,\ldots,i_k)$, and whose multiplication on basis elements is given by
\[
    (u_{i_1} \cdots u_{i_k}) \cdot (u_{i_1} \cdots u_{j_\ell})
    =
    u_{i_1} \cdots u_{i_k} u_{i_1} \cdots u_{j_\ell}.
\]
\subsection{Words}

Next, continue from the ``words'' side.

\begin{definition}
    Let $[N] = \{1,\ldots,N\}$ be set of first $N$ positive integers, which we will call our collection of \defstyle{letters}.
    \begin{enumerate}[label=(\alph*)]
        \item 
            The \defstyle{monoid of words} $\mathbb{W}$ is the free monoid on the generating set $[N]$.
        \item 
            We define $\calU^\ast$ to be the monoid ring $\ZZ\mathbb{W}$.
    \end{enumerate}
\end{definition}

$\calU^\ast$ is the associative unital $\ZZ$-algebra whose basis consists of words $\sfw = \sfw_1\cdots\sfw_k \in \WW$ and whose multiplication is given by concatenation.

Of course, as a tuple of integers, they also index noncommutative monomials--- as a shorthand, if $\sfw$ is a word, then we will denote by $\bfu_\sfw$ the monomial
\[
    \bfu_\sfw
    \coloneq
    u_{\sfw_1} \cdots u_{\sfw_k}.
\]

This definition becomes an actual structure on $\calU$ and $\calU^\ast$ via pairing--- we have simply set $\langle \bfu \rangle$ and $\WW$ to be orthonormal bases.

\begin{definition}
    We define a scalar product $\ip{\wc,\wc}: \calU \times \calU^\ast \to \ZZ$ by
    \[
        \ip{\bfu_\sfv, \sfw} 
        = 
        \delta_{\sfv\sfw}.
    \]
    for all $\bfu_\sfv \in \calU$ and $\sfw \in \calU^\ast$.
\end{definition}

\subsection{Ideals}

\begin{definition}
    For any commutative ring $R$, let $\calU_R$ denote the extension of scalars $\calU \otimes_\ZZ R$.
    We define $\calU^\ast_R$ similarly.
\end{definition}

\begin{definition}
    Let $I$ be an ideal of $\calU$. 
    The \defstyle{orthogonal complement} $I^\perp$ of $I$ is defined
    \[
        I^\perp
        \coloneq
        \{
            \gamma \in \calU^\ast
            \mid
            \ip{I, \gamma} = 0
        \}.
    \]
\end{definition}

$I^\perp$ really provides a parameterization of coset representatives for $\calU/I$--- this is precise when we are working over, say, $\calU_\QQ$ or $\calU_\RR$.

\subsection{Polynomial and formal power series rings over $\calU$}

\begin{definition}
    If $I$ is a two-sided ideal of $\calU$, then we let $I\lBrack x \rBrack$ be the ideal of $\calU \lBrack x \rBrack$ generated by $I$.
\end{definition}

\section{
    Noncommutative elementary and homogeneous symmetric functions
}

\begin{definition}
    The \defstyle{noncommutative elementary symmetric function} $e_k(\bfu)$ is defined to be
    \begin{equation}
        \label{eqn:ncElementaryDef}
        e_k(\bfu)
        \coloneq
        \underbrace{
            \sum_{i_1 > i_2 > \cdots > i_k}
        }_{\text{(decreasing!)}}
        u_{i_1} u_{i_2} \cdots u_{i_k}. 
    \end{equation}
    The \defstyle{noncommutative homogeneous symmetric function} $h_k(\bfu)$ is defined to be
    \begin{equation}
        \label{eqn:ncHomogeneousDef}
        h_k(\bfu)
        \coloneq
        \underbrace{
            \sum_{i_1 \leq i_2 \leq \cdots \leq i_k}
        }_{\text{(weakly increasing!)}}
        u_{i_1} u_{i_2} \cdots u_{i_k}. 
    \end{equation}
    We will have the convention that $e_0(\bfu) = h_0(\bfu) = 1$, and that $e_k(\bfu) = h_k(\bfu) = 0$ if $k < 0$.
\end{definition}

In other words, the elementary symmetric functions are generating functions for columns, and the homogeneous symmetric functions are generating functions for rows.

\subsection{
    Newton's identities
}

We define noncommutative analogues of standard generating functions for the elementary and homogeneous symmetric functions.

\begin{definition}
    \label{def:rowcolGenFuncs}
    We define the generating functions for $e_k(\bfu)$'s and $h_k(\bfu)$'s
    \begin{align}
        \label{eqn:colGenFunc}
        E(x)
        \coloneq
        \sum_{k=0}^N
        x^k e_k(\bfu)
        &=
        \prod_{i=N}^1
        (1+xu_i),
        \\
        \label{eqn:rowGenFunc}
        H(x)
        \coloneq
        \sum_{k=0}^\infty
        x^k h_k(\bfu)
        &=
        \prod_{i=1}^N
        (1-xu_i)^{-1}.
    \end{align}
    in $\calU\lBrack x \rBrack$.
\end{definition}

An immediate consequence of this definition is a noncommutative analogue of Newton's identities.

\begin{proposition}
    [Noncommutative Newton-Girard formulas]
    We have
    \begin{equation}
        \label{eqn:ncNewtonGirardFPS}
        E(x)H(-x) = H(x)E(-x) = 1.
    \end{equation}
    In particular,
    \begin{align}
        \label{eqn:ncNewtonGirard1}
        \sum_{k=0}^n (-1)^ke_k(\bfu)h_{n-k}(\bfu) 
        &= 
        0, \\
        \label{eqn:ncNewtonGirard2}
        \sum_{k=0}^n (-1)^kh_k(\bfu)e_{n-k}(\bfu) 
        &= 
        0
    \end{align}
    for all $n > 1$.
\end{proposition}

\begin{proof}
    Putting together (\ref{eqn:colGenFunc}) and (\ref{eqn:rowGenFunc}) immediately gives us (\ref{eqn:ncNewtonGirardFPS}).
    One obtains (\ref{eqn:ncNewtonGirard1}) and (\ref{eqn:ncNewtonGirard2}) by comparing coefficients in (\ref{eqn:ncNewtonGirardFPS}).
\end{proof}

\begin{corollary}
    \label{corr:ECommutesIffHCommutes}
    Let $I$ be an ideal of $\calU$.
    The following are equivalent:
    \begin{enumerate}[label=(\alph*)]
        \item 
            $E(x)E(y) \overset{I[x,y]}{\equiv} E(y)E(x)$. 
        \item 
            $H(x)H(y) \overset{I[x,y]}{\equiv} H(y)H(x)$.
    \end{enumerate}
\end{corollary}

\begin{proof}
    Suppose $E(x)E(y) = E(y)E(x)$ for all commuting $x,y$.
    Then $H(x)H(y) = E(-x)^{-1}E(-y)^{-1} = E(-y)^{-1}E(-x)^{-1} = H(y)H(x)$.
    The reverse implication is proved identically.
\end{proof}

\subsection{
    The ideal $I_C$
}

Evidently, in the quotient
\[
    \calU / [\calU\calU],
\]
the elementaries commute, since all monomials commute:
\[
    e_k(\bfu)e_j(\bfu) 
    \overset{[\calU\calU]}{\equiv} 
    e_j(\bfu)e_k(\bfu).
\]
This ideal is pretty big--- it turns the entirety of $\calU$ into a commutative algebra.
It turns out that, not only can we find smaller ideals, but we can find the \textit{smallest ideal} in which the elementaries commute.

\begin{definition}
    We define the ideal $I_C$ to be the ideal consisting of exactly the elements
    \begin{align}
        &
        u_b^2u_a + u_au_bu_a - u_bu_au_b - u_bu_a^2 
        &
        (a<b),
        \\
        &
        u_bu_cu_a + u_au_cu_b - u_bu_au_c - u_cu_au_b
        &
        (a<b<c),
        \\
        &
        u_cu_bu_cu_a + u_bu_cu_au_c - u_cu_bu_au_c - u_bu_c^2u_a
        &
        (a<b<c).
    \end{align}
    Equivalently, $I_C$ is the ideal generated by the relations
    \begin{align}
        \label{eqn:I_C1}
        [u_au_b]u_a 
        &\equiv 
        u_b[u_au_b]
        &
        (a<b),
        \\
        \label{eqn:I_C2}
        [u_au_c]u_b 
        &\equiv 
        u_b[u_au_c]
        &
        (a<b<c),
        \\
        \label{eqn:I_C3}
        [u_bu_c][u_au_c] 
        &\equiv 
        0
        &
        (a<b<c).
    \end{align}
\end{definition}

We now have the following deep, difficult calculation--- the proof which is deferred to the Appendix, \S\ref{subsec:I_CImpliesEsCommuteProof}.

\begin{theorem}
    \label{thm:ICImpliesEsCommute}
    If $I \supseteq I_C$, then $e_k(\bfu)e_j(\bfu) \overset{I}{\equiv} e_j(\bfu)e_k(\bfu)$ for all $j,k$. 
\end{theorem}

With this, we now precisely know in which quotients of $\calU$ the $e_k(\bfu)'s$'s generate a commutative subalgebra.

Moreover, these are the exact same quotients in which the $h_k(\bfu)$'s generate a commutative subalgebra.

\begin{lemma}
    Let $I$ be an ideal of $\calU$.
    The following are equivalent:
    \begin{enumerate}[label=(\alph*)]
        \item 
            $e_k(\bfu)e_j(\bfu) \overset{I}{\equiv} e_j(\bfu)e_k(\bfu)$ for all $j,k$. 
        \item 
            $h_k(\bfu)h_j(\bfu) \overset{I}{\equiv} h_j(\bfu)h_k(\bfu)$ for all $j,k$.
    \end{enumerate}
\end{lemma}


\subsection{
    The map $\Psi$
}

\begin{theorem}
    [Fundamental theorem of symmetric functions]
    \label{thm:FundThmSymFuncs}
    Let $\Lambda(\bfx)$ denote the ring of symmetric polynomials in the commuting variables $\bfx = (x_1,\ldots,x_N)$.
    Then the elementary symmetric functions in $\bfx$ are algebraically independent, and moreover generate $\Lambda(\bfx)$ as an algebra:
    \[
        \Lambda(\bfx)
        =
        \QQ[e_1(\bfx), e_2(\bfx), \ldots, e_n(\bfx)].
    \]
\end{theorem}

\begin{proof}
    One can prove this via the \textit{Gale-Ryser} theorem, to show that the transition matrix from the elementary symmetric function basis to the monomial symmetric function basis is invertible.
    This proof is carried out in \cite[Theorem 7.4.4]{EC2}.
\end{proof}

\begin{corollary}
    If $I$ contains $I_C$, then the map
    \begin{align*}
        \Psi:
        \Lambda(\bfx)
        &\to
        \calU
        \\
        e_k(\bfx)
        &\mapsto
        e_k(\bfu)
    \end{align*}
    induces a ring homomorphism $\Lambda(\bfx) \to \calU/I$.
\end{corollary}

\begin{proof}
    Combine Theorems \ref{thm:FundThmSymFuncs} and \ref{thm:ICImpliesEsCommute}.
\end{proof}

\begin{proposition}
    If $I$ contains $I_C$, any identity in $\Lambda(\bfx)$ that can be purely expressed in terms of polynomials in $h_k(\bfx)$'s and $e_k(\bfx)$'s holds in $\Lambda(\bfu)/I$.
\end{proposition}

\begin{proof}
\end{proof}

\section{
    Noncommutative Schur functions
}


\begin{definition}
    The \defstyle{noncommutative Schur function} $\frkJ(\bfu)$ is defined 
    \[
        \frkJ_\lambda(\bfu)
        \coloneq
        \sum_{\pi \in S_{t}}
        \sgn(\pi)
        e_{\lambda^\top_1+\pi(1)-1}(\bfu)
        e_{\lambda^\top_2+\pi(2)-2}(\bfu)
        \cdots
        e_{\lambda^\top_{t}+\pi(t)-t}(\bfu),
    \]
    where $t = \lambda_1$ is the number of parts of $\lambda^\top$.
    Alternatively, we also define
    \[
        \frkJ_\lambda(\bfu)
        \coloneq
        \sum_{\pi \in S_{t}}
        \sgn(\pi)
        h_{\lambda_1+\pi(1)-1}(\bfu)
        h_{\lambda_2+\pi(2)-2}(\bfu)
        \cdots
        h_{\lambda_t+\pi(t)-t}(\bfu),
    \]
    and it's easy to see that they are congruent modulo any ideal containing $I_C$.
\end{definition}

The first definition is based on the \defstyle{Kostka-Naegelsbach identity}
\[
    s_\lambda(\bfx)
    =
    \det\big(e_{\lambda^\top_i+j-i}(\bfx)\big)_{i,j=1}^n,
\]
and the second is based on the \defstyle{Jacobi-Trudi identity}
\[
    s_\lambda(\bfx)
    =
    \det\big(h_{\lambda_i+j-i}(\bfx)\big)_{i,j=1}^n.
\]
Since these are purely polynomials of elementary symmetric and complete homogeneous polynomials, one sees the following
\begin{definition}
    If $I \supseteq I_C$, then
    \[
        \Psi\big(s_\lambda(\bfx)\big)
        \overset{I}{\equiv}
        \frkJ_\lambda(\bfu).
    \]
\end{definition}
\begin{proof}
    \begin{align*}
        \Psi\big(s_\lambda(\bfx)\big)
        &=
        \Psi \left(
            \det\big(e_{\lambda^\top_i+j-i}(\bfx)\big)_{i,j=1}^n
        \right)
        \\
        &=
        \Psi \left(
            \sum_{\pi \in S_n}
            \sgn(\pi)
            h_{\pi_1+\pi(1)-1}(\bfx)
            \cdots
            h_{\pi_n+\pi(n)-n}(\bfx)
        \right)
        \\
        &\overset{I}{\equiv}
        \sum_{\pi \in S_n}
        \sgn(\pi)
        h_{\pi_1+\pi(1)-1}(\bfu)
        \cdots
        h_{\pi_n+\pi(n)-n}(\bfu)
        \\
        &=
        \frkJ_\lambda(\bfu).
    \end{align*}
\end{proof}

\begin{theorem}
    [\cite{FG98}, \cite{BF16}]
    In the ideal $I_\varnothing$,
    \[
        \frkJ_\lambda(\bfu)
        \coloneq
        \sum_{T \in \SSYT(\lambda;N)}
        \bfu^{\colword T}.
    \]
\end{theorem}

\subsection{Cauchy kernel}

\begin{theorem}
    If $I$ contains $I_C$, then for all $\gamma \in I_C^\perp$,
    \begin{align}
        \label{eq:CauchyNC}
        \ip{
            \prod_{i=1}^N
            \prod_{j=1}^N
            (1 - x_iu_j)^{-1},
            \gamma
        }
        &=
        \sum_\lambda
        s_\lambda(\bfx)
        \ip{
        \frkJ_\lambda(\bfu),\gamma
        },
        \\
        \label{eq:DualCauchyNC}
        \ip{
            \prod_{i=1}^N
            \prod_{j=N}^1
            (1 + x_iu_j),
            \gamma
        }
        &=
        \sum_\lambda
        s_\lambda(\bfx)
        \ip{
        \frkJ_{\lambda^\top}(\bfu),\gamma
        }.
    \end{align}
\end{theorem}

\begin{proof}
    We will prove (\ref{eq:CauchyNC}) directly:
    \begin{align*}
        \ip{
            \prod_{i=1}^N
            \prod_{j=1}^N
            (1 - x_iu_j)^{-1},
            \gamma
        }
        &=
        \ip{
            \prod_{i=1}^N
            \sum_{k=0}^\infty
            x^kh_k(\bfu),
            \gamma
        }
        \\
        &=
        \ip{
            \Psi\left(
                \prod_{i=1}^N
                \sum_{k=0}^\infty
                x^kh_k(\bfy)
            \right),
            \gamma
        }
        \\
        &=
        \ip{
            \Psi\left(
                \sum_\lambda
                s_\lambda(\bfx)
                s_\lambda(\bfy)
            \right),
            \gamma
        }
        \\
        &=
        \ip{
            \sum_\lambda
            s_\lambda(\bfx)
            \frkJ_\lambda(\bfu),
            \gamma
        }
        \\
        &=
        \sum_\lambda
        s_\lambda(\bfx)
        \ip{\frkJ_\lambda(\bfu),\gamma}.
    \end{align*}
    (\ref{eq:DualCauchyNC}) is proven the exact same way.
\end{proof}


\section{
    The symmetric function $F_\gamma$
}

We will now give the definition of the symmetric function associated to a vector in $I^\perp$, first defined in \cite{FG98}.

\begin{definition}
    Fix an ideal $I$ containing $I_C$, and let $\gamma \in \calU^\ast$.
    We define $F_\gamma$ to be 
    \[
        F_\gamma(\bfx)
        \coloneq
        \ip{\Omega(\bfx,\bfu),\gamma}.
    \]
    for all $\gamma \in I^\perp$.
\end{definition}

\begin{definition}
\end{definition}

\begin{theorem}
    If $\frkJ_\lambda(\bfu)$ has a monomial-positive representative in $\calU$ for all $\lambda$, then $F_\gamma(\bfx)$ is Schur positive.

    If, moreover, one can has a procedure for computing this representative, one automatically has a procedure for computing the coefficients of $F_\gamma(\bfx)$'s Schur expansion.
\end{theorem}

\section{Applications}

\subsection{Recovering known results in the plactic algebra}

\subsection{Stanley symmetric functions via the nilCoxeter algebra}

The connection between Schubert polynomials and the nilCoxeter ideal was first explored by Richard Stanley and 

\begin{definition}
    The \defstyle{nilCoxeter} ideal
\end{definition}

\subsection{LLT polynomials}




\section{Linear programming}

Consider the positive cones $\calU_{\geq 0}$ and $\calU^\ast_{\geq 0}$.


\section{Algebras of operators}

\begin{definition}
    A \defstyle{combinatorial representation} of $\calU/I$ is
\end{definition}

\section{Switchboards}

\begin{definition}
    The \defstyle{switchboard ideal} $I_S$ is
\end{definition}

\begin{definition}
    Let $\sfw = \sfw_1\cdots\sfw_n \in \calU^\ast$.
    We define the \defstyle{fundamental quasisymmetric function} $Q_{\Des(\sf w)}(\bfx)$ by
    \[
        Q_{\Des(\sf w)}(\bfx)
        \coloneq
        \sum_{
            \substack{
                i_1 \leq \cdots \leq i_n \\
                j \in \Des({\sf w}) \implies i_j < i_{j+1}
            }
        }
        x_{i_1} \cdots x_{i_n}.
    \]
    In general
    \[
        Q_{\Des(\sf w)}(\bfx)
        \coloneq
        \sum_{
            \substack{
                i_1 \leq \cdots \leq i_n \\
                j \in \Des({\sf w}) \implies i_j < i_{j+1}
            }
        }
        x_{i_1} \cdots x_{i_n}.
    \]
\end{definition}


\section{Appendix}

\subsection{Proof of Theorem \ref{thm:ICImpliesEsCommute}}
\label{subsec:I_CImpliesEsCommuteProof}

We will follow A.N Kirillov's proof \cite[Theorem 2.26]{K16}.
It has a few errors, but it's recoverable.
Blasiak and Fomin prove the assertion in the context of a much more general framework in \cite{BF18}.

First, we recast the problem in terms of generating functions.
Recall that $E(x) \coloneq (1+xu_N) \cdots (1+xu_1)$ is the generating function for the $e_k(\bfu)'s$, that is,
\[
    E(x) 
    = 
    \sum_{k=1}^N
    x^k
    e_k(\bfu).
\]
Then, we can recast our problem by working with the $E$'s.
\begin{lemma}
    \label{lem:ElementariesCommuteIffECommutes}
    Let $I$ be an ideal of $\calU$.
    The following are equivalent:
    \begin{enumerate}[label=(\alph*)]
        \item 
            $E(x)E(y) \overset{I[x,y]}{\equiv} E(y)E(x)$.
        \item 
            $e_k(\bfu)e_j(\bfu) \overset{I}{\equiv} e_j(\bfu)e_k(\bfu)$ for all $j,k$. 
    \end{enumerate}
\end{lemma}

\begin{proof}
    Expand and compare coefficients.
\end{proof}

To do this inductively, we put $E_{ji}(x) = (1+xu_j) \cdots (1+xu_i)$ for all $i < j$.

\begin{proof}
    [Proof of Theorem \ref{thm:ICImpliesEsCommute}]

    We will first show this inductively--- we claim that
    \[
        [E_{n,1}(x)E_{n,1}(y)]
        \overset{I_C[x,y]}{\equiv}
        0,
    \]
    for all $n \geq 0$.
    The case $n=1$ is easily verified.
    Now, suppose $[E_n(x)E_n(y)] \equiv 0$.
    Then
    \begin{landscape}
        \begin{align*}
            &[E_{n+1,1}(x),E_{n+1,1}(y)]
            \\
            &=
            \left[
                (1+xu_{n+1})E_{n,1}(x),
                (1+yu_{n+1})E_{n,1}(y)
            \right]
            \\
            &=
            (1+xu_{n+1})E_{n,1}(x)(1+yu_{n+1})E_{n,1}(y)
            -(1+yu_{n+1})E_{n,1}(y)(1+xu_{n+1})E_{n,1}(x)
            \\
            &=
            (1+xu_{n+1})
            \Big(
                \big[E_{n,1}(x),(1+yu_{n+1})\big]
                + (1+yu_{n+1})E_{n,1}(x)
            \Big)
            E_{n,1}(y)
            \\
            &\qquad-
            (1+yu_{n+1})
            \Big(
                \big[E_{n,1}(y),(1+xu_{n+1})\big]
                + (1+xu_{n+1})E_{n,1}(y)
            \Big)
            E_{n,1}(x)
            \\
            &=
            (1+xu_{n+1})\big[E_{n,1}(x),(1+yu_{n+1})\big]E_{n,1}(y)
            + (1+xu_{n+1})(1+yu_{n+1})E_{n,1}(x)E_{n,1}(y)
            \\
            &\qquad-
            (1+yu_{n+1})\big[E_{n,1}(y),(1+xu_{n+1})\big]E_{n,1}(x)
            - (1+yu_{n+1})(1+xu_{n+1})E_{n,1}(y)E_{n,1}(x)
            \\
            &\equiv
            (1+xu_{n+1})\big[E_{n,1}(x),(1+yu_{n+1})\big]E_{n,1}(y)
            - (1+yu_{n+1})\big[E_{n,1}(y),(1+xu_{n+1})\big]E_{n,1}(x)
            \\
            &=
            (1+xu_{n+1})
            \left[
                (1+xu_n)\cdots(1+xu_1),(1+yu_{n+1})
            \right]
            E_{n,1}(y)
            \\
            &\qquad-
            (1+yu_{n+1})
            \left[
                (1+yu_n)\cdots(1+yu_1),(1+xu_{n+1})
            \right]
            E_{n,1}(x)
            \\
            &=
            (1+xu_{n+1})
            \left(
                \sum_{i=1}^n
                E_{n,i+1}(x)
                \big[
                    (1+xu_i),(1+yu_{n+1})
                \big]
                E_{i-1,1}(x)
            \right)
            E_n(y)
            \\
            &\qquad-
            (1+yu_{n+1})
            \left(
                \sum_{i=1}^n
                E_{n,i+1}(y)
                \big[
                    (1+yu_i),(1+xu_{n+1})
                \big]
                E_{i-1,1}(y)
            \right)
            E_n(x)
            \\
            &=
            (1+xu_{n+1})
            \left(
                xy
                \sum_{i=1}^n
                E_{n,i+1}(x)
                \big[
                    u_i,u_{n+1}
                \big]
                E_{i-1,1}(x)
            \right)
            E_n(y)
            \\
            &\qquad-
            (1+yu_{n+1})
            \left(
                xy
                \sum_{i=1}^n
                E_{n,i+1}(y)
                \big[
                    u_i,u_{n+1}
                \big]
                E_{i-1,1}(y)
            \right)
            E_n(x)
            \\
            &=
            xy
            \sum_{i=1}^n
            \Bigg(
            (1+xu_{n+1})
            E_{n,i+1}(x)
            \big[
                u_i,u_{n+1}
            \big]
            E_{i-1,1}(x)E_n(y)
            \\
            &\hspace{6em}-
            (1+yu_{n+1})
            E_{n,i+1}(y)
            \big[
                u_i,u_{n+1}
            \big]
            E_{i-1,1}(y)E_n(x)
            \Bigg)
            \\
            &\equiv
            xy
            \sum_{i=1}^n
            \Bigg(
            (1+xu_{n+1})
            [u_i,u_{n+1}]
            E_{n,i+1}(x)E_{i-1,1}(x)E_n(y)
            \\
            &\hspace{6em}-
            (1+yu_{n+1})
            [u_i,u_{n+1}]
            E_{n,i+1}(y)E_{i-1,1}(y)E_n(x)
            \Bigg)
            \\
            &\equiv
            xy
            \sum_{i=1}^n
            \Bigg(
            [u_i,u_{n+1}]
            (1+xu_i)E_{n,i+1}(x)E_{i-1,1}(x)E_n(y)
            \\
            &\hspace{6em}-
            [u_i,u_{n+1}]
            (1+yu_i)E_{n,i+1}(y)E_{i-1,1}(y)E_n(x)
            \Bigg)
            \\
            &=
            xy
            \sum_{i=1}^n
            [u_i,u_{n+1}]
            \Big(
                (1+xu_i)E_{n,i+1}(x)E_{i-1,1}(x)E_n(y)
                -(1+yu_i)E_{n,i+1}(y)E_{i-1,1}(y)E_n(x)
            \Big)
            \\
            &\equiv
            xy
            \sum_{i=1}^n
            [u_i,u_{n+1}]
            \Big(
                E_{n,1}(x)E_{n,1}(y)
                - E_{n,1}(y)E_{n,1}(x)
            \Big)
            \\
            &=
            xy
            \sum_{i=1}^n
            [u_i,u_{n+1}]
            [E_{n,1}(x),E_{n,1}(y)]
            \\
            &=
            0.
        \end{align*}
    \end{landscape}
\end{proof}

\begin{thebibliography}{999999}
    \raggedright\footnotesize

    \bibitem[FG98]{FG98}
    Sergey Fomin and Curtis Greene, 
    \textit{Noncommutative Schur functions and their applications}, 
    Discrete Math. \textbf{193} (1998), 179-200.

    \bibitem[BF16]{BF16}
    Jonah Blasiak and Sergey Fomin, 
    \textit{Noncommutative Schur functions, switchboards, and Schur positivity},
    Sel. Math. \textbf{23} (2017), 727-766.

    Also available as \arxiv{1510.00657}.

    \bibitem[BF18]{BF18}
    Jonah Blasiak and Sergey Fomin, 
    \textit{Rules of Three for commutation relations},
    J. Algebra. \textbf{500} (2018), 193-220.

    Also available as \arxiv{1608.05042}.

    \bibitem[A15]{A15}
    Sami Assaf,
    \textit{Dual equivalence graphs I: A new paradigm for Schur positivity},
    Forum. Math. Sigma \textbf{3} (2015), e12.

    Also available as \arxiv{1506.03798}.

    \bibitem[K16]{K16}
    Alexandre Kirillov,
    \textit{Notes on Schubert, Grothendieck, and Key polynomials},
    SIGMA \textbf{12} (2016)

    Also available as \arxiv{1501.07337}.

    \bibitem[L04]{L04}
    Thomas Lam,
    \textit{Ribbon Schur operators},
    European J. Combin. \textbf{29} (2008), 343-359.

    Also available as \arxiv{math/0409463}.

    \bibitem[B16]{B16}
    Jonah Blasiak, 
    \textit{Haglund's conjecture on 3-column Macdonald polynomials},
    Math. Z. \textbf{283} (2016), 601-628.

    Also available as \arxiv{1411.3646}.

    \bibitem[FS91]{FS91}
    Sergey Fomin and Richard P. Stanley, 
    \textit{Schubert Polynomials and the NilCoxeter Algebra},
    Adv. Math. \textbf{103} (1994), 196-207.

    \bibitem[M91]{M91}
    Ian G. Macdonald, 
    \textit{Notes on Schubert Polynomials}, 
    LACIM, 1991.

    \bibitem[EC2]{EC2}
    Richard P. Stanley, 
    \textit{Enumerative Combinatorics. Volume 2}, 
    Cambridge University Press 2023.

\end{thebibliography}

\end{document}
