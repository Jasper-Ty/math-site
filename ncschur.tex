% !TEX TS-program = xelatex

\documentclass{article}

\newcommand{\ip}[1]{
    \left\langle
        {#1}
    \right\rangle
}

\newcommand*\colword{\operatorname{\mathsf{colword}}}
\newcommand*\frkJ{\mathfrak{J}}
\newcommand*\SSYT{\operatorname{SSYT}}
\newcommand*\sgn{\operatorname{sgn}}
\newcommand*\Des{\operatorname{Des}}

% See https://github.com/Jasper-Ty/dotfiles
\usepackage[garamond]{jaspercommon}

\barenv[][section]{theorem}{Theorem}
\barenv[bartheorem]{proposition}{Proposition}
\barenv[bartheorem]{corollary}{Corollary}
\barenv[bartheorem]{lemma}{Lemma}
\barenv[bartheorem]{definition}{Definition}
\barenv[bartheorem]{convention}{Convention}

\title{Noncommutative Schur functions}
\author{Jasper Ty}
\date{}

\titleauthorhead

\begin{document}

\maketitle

\section*{What is this?}

This is (going to be) an ``infinite napkin'' set of notes I am taking about the Fomin-Greene theory of noncommutative Schur functions.

\tableofcontents

\newpage

\section{Ideals and words}

Let $\bfu = (u_1,\ldots,u_N)$ be a collection of variables.
Let $\langle \bfu \rangle$ be the free semigroup on the generators $\bfu$.
Then, let $\calU = \ZZ\langle\bfu\rangle$ denote the corresponding semigroup ring--- the free associative ring generated by $\bfu$.

We will denote by $\calU^\ast$ the $\ZZ$-module spanned by words in the alphabet $\{1,\ldots,N\}$.

We will have a fundamental pairing $\ip{\wc,\wc}$ given by making noncommutative monomials dual to words.

Now, if $I$ is an ideal of $\calU$, we define $I^\perp$ by
\[
    I^\perp
    \coloneq
    \{
        \gamma \in \calU^\ast
        \mid
        \ip{I, \gamma} = 0
    \}.
\]

\section{
    Noncommutative 
    \texorpdfstring{$e$}{e}'s
    and
    \texorpdfstring{$h$}{h}'s
}

\begin{definition}
    The \defstyle{noncommutative elementary symmetric function} $e_k(\bfu)$ is defined to be
    \[
        e_k(\bfu)
        \coloneq
        \sum_{i_1 > i_2 > \cdots > i_k}
        u_{i_1} u_{i_2} \cdots u_{i_k}. 
    \]
    The \defstyle{noncommutative complete homogeneous symmetric function} $h_k(\bfu)$ is defined to be
    \[
        h_k(\bfu)
        \coloneq
        \sum_{i_1 \geq i_2 \geq \cdots \geq i_k}
        u_{i_1} u_{i_2} \cdots u_{i_k}. 
    \]
\end{definition}




\subsection{
    The ideal \texorpdfstring{$I_C$}{I\_C}
}

\begin{lemma}
    Let $I$ be an ideal of $\calU$.
    The following are equivalent:
    \begin{enumerate}[label=(\alph*)]
        \item 
            $e_k(\bfu)e_j(\bfu) \equiv e_j(\bfu)e_k(\bfu) \mod I$ for all $j,k$. 
        \item 
            $h_k(\bfu)h_j(\bfu) \equiv h_j(\bfu)h_k(\bfu) \mod I$ for all $j,k$.
    \end{enumerate}
\end{lemma}



\begin{definition}
    We define the ideal $I_C$ to be the ideal consisting of exactly the elements
    \begin{align}
        &
        u_b^2u_a + u_au_bu_a - u_bu_au_b - u_bu_a^2 
        &
        (a<b),
        \\
        &
        u_bu_cu_a + u_au_cu_b - u_bu_au_c - u_cu_au_b
        &
        (a<b<c),
        \\
        &
        u_cu_bu_cu_a + u_bu_cu_au_c - u_cu_bu_au_c - u_bu_c^2u_a
        &
        (a<b<c).
    \end{align}
\end{definition}

Compactly, these are the relations
\[
    [u_au_b]u_a \equiv u_b[u_au_b], \quad
    [u_au_c]u_b \equiv u_b[u_au_c], \quad
    [u_cu_b]u_cu_a \equiv [u_cu_b]u_au_c
\]
for all $a<b<c$.

\begin{theorem}
    \label{thm:ICImpliesEsCommute}
    $I_C$ is the smallest ideal in which the elementary symmetric functions $e_k(\bfu_S)$ and $e_\ell(\bfu_S)$ commute for any $k,\ell,S$.
\end{theorem}

\subsection{
    The map \texorpdfstring{$\Psi_I$}{Psi\_I}
}

\begin{theorem}
    [Fundamental theorem of symmetric functions]
    \label{thm:FundThmSymFuncs}
    Let $\Lambda(\bfx)$ denote the ring of symmetric polynomials in the commuting variables $\bfx = (x_1,\ldots,x_n)$.
    Then
    \[
        \Lambda(\bfx)
        \simeq
        \QQ[e_1(\bfx), e_2(\bfx), \ldots, e_n(\bfx)].
    \]
\end{theorem}

\begin{proof}
    See Theorem 7.4.4 in \cite{EC2}.
    One checks that products of the form.
    One can prove this via the \textit{Gale-Ryser} theorem.
\end{proof}

\begin{corollary}
    If $I$ contains $I_C$, then the map
    \begin{align*}
        \Psi_I:
        \Lambda_n(\bfx)
        &\to
        \calU/I
        \\
        e_k(\bfx)
        &\mapsto
        e_k(\bfu)
    \end{align*}
    extends to a ring homomorphism.
\end{corollary}



\begin{proof}
    Combine Theorems \ref{thm:FundThmSymFuncs} and \ref{thm:ICImpliesEsCommute}.
\end{proof}



\section{
    Noncommutative Schur functions
}

\begin{definition}
    Let $I \supseteq I_C$.
    The \defstyle{noncommutative Schur function} $\frkJ(\bfu) \in \calU/I$ is defined to be
    \[
        \frkJ_\lambda(\bfu)
        =
        \sum_{\pi \in S_{t}}
        \sgn(\pi)
        e_{\lambda^\top_1+\pi(1)-1}(\bfu)
        e_{\lambda^\top_2+\pi(2)-2}(\bfu)
        \cdots
        e_{\lambda^\top_{t}+\pi(t)-t}(\bfu),
    \]
    where $t = \lambda_1$ is the number of parts of $\lambda^\top$.
    Alternatively, since the $h$'s commute whenever the $e$'s do,
    \[
        \frkJ_\lambda(\bfu)
        =
        \sum_{\pi \in S_{t}}
        \sgn(\pi)
        h_{\lambda_1+\pi(1)-1}(\bfu)
        h_{\lambda_2+\pi(2)-2}(\bfu)
        \cdots
        h_{\lambda_t+\pi(t)-t}(\bfu).
    \]
\end{definition}

The first definition is based on the \defstyle{Kostka-Naegelsbach identity}
\[
    s_\lambda(\bfx)
    =
    \det\big(e_{\lambda^\top_i+j-i}(\bfx)\big)_{i,j=1}^n,
\]
and the second is based on the \defstyle{Jacobi-Trudi identity}
\[
    s_\lambda(\bfx)
    =
    \det\big(h_{\lambda_i+j-i}(\bfx)\big)_{i,j=1}^n.
\]
Since these are purely polynomials of elementary symmetric and complete homogeneous polynomials, one sees the following
\begin{definition}
    If $I \supseteq I_C$, then
    \[
        \Psi_I\big(s_\lambda(\bfx)\big)
        \equiv
        \frkJ_\lambda(\bfu)
        \mod I.
    \]
\end{definition}
\begin{proof}
    \begin{align*}
        \Psi_I\big(s_\lambda(\bfx)\big)
        &=
        \Psi_I \left(
            \det\big(e_{\lambda^\top_i+j-i}(\bfx)\big)_{i,j=1}^n
        \right)
        \\
        &=
        \Psi_I \left(
            \sum_{\pi \in S_n}
            \sgn(\pi)
            h_{\pi_1+\pi(1)-1}(\bfx)
            \cdots
            h_{\pi_n+\pi(n)-n}(\bfx)
        \right)
        \\
        &\equiv
        \sum_{\pi \in S_n}
        \sgn(\pi)
        h_{\pi_1+\pi(1)-1}(\bfu)
        \cdots
        h_{\pi_n+\pi(n)-n}(\bfu) \mod I
        \\
        &\equiv
        \frkJ_\lambda(\bfu) \mod I.
    \end{align*}
\end{proof}



\begin{theorem}
    If $I$ contains $I_C$, then for all $\gamma \in I_C^\perp$,
    \[
        \ip{
            \prod_{i=1}^m
            \prod_{j=n}^1
            (1 + x_iu_j),
            \gamma
        }
        =
        \sum_\lambda
        s_\lambda(\bfx)
        \ip{
        \frkJ_{\lambda^\top}(\bfu),\gamma
        }.
    \]
    \begin{proof}
    \end{proof}
\end{theorem}

\begin{theorem}
    [\cite{FG98}, \cite{BF16}]
    In the ideal $I_\varnothing$,
    \[
        \frkJ_\lambda(\bfu)
        \coloneq
        \sum_{T \in \SSYT(\lambda;N)}
        \bfu^{\colword T}.
    \]
\end{theorem}

\subsection{Cauchy kernel}

\begin{definition}
    Let $\bfx = (x_1,x_2\ldots)$ be a countable collection of commuting variables.
\end{definition}

\section{Applications}

\section{Linear programming}

Consider the positive cones $\calU_{\geq 0}$ and $\calU^\ast_{\geq 0}$.


\section{Algebras of operators}

\begin{definition}
    A \defstyle{combinatorial representation} of $\calU/I$ is
\end{definition}

\begin{definition}
\end{definition}


\section{Appendix}

\subsection{Gessel's fundamental quasisymmetric function}

\begin{definition}
    Let $\sf{w}$ be a word.
    We define the \defstyle{fundamental quasisymmetric function} $Q_{\Des(\sf w)}$ by
    \[
        Q_{\Des(\sf w)}
        \coloneq
        \sum_{
            \substack{
                1 \leq i_1 \leq \cdots \leq i_n \\
                j \in \Des(\sf w) \implies i_j < i_{j+1}
            }
        }
        x_{i_1} \cdots x_{i_n}.
    \]
\end{definition}

\subsection{The Edelman-Greene correspondence}

\begin{thebibliography}{999999}
    \raggedright\footnotesize

    \bibitem[EC2]{EC2}
    Richard P. Stanley, 
    \textit{Enumerative Combinatorics. Volume 2}, 
    Cambridge University Press 2023.

    \bibitem[FG98]{FG98}
    Sergey Fomin and Curtis Greene, 
    \textit{Noncommutative Schur functions and their applications}, 
    Discrete Math. \textbf{193} (1998), 179-200.

    \bibitem[BF16]{BF16}
    Jonah Blasiak and Sergey Fomin, 
    \textit{Noncommutative Schur functions, switchboards, and Schur positivity},
    Sel. Math. \textbf{23} (2017), 727-766.

    Also available as \arxiv{1510.00657}.

    \bibitem[A15]{A15}
    Sami Assaf,
    \textit{Dual equivalence graphs I: A new paradigm for Schur positivity},
    Forum. Math. Sigma \textbf{3} (2015), e12.

    Also available as \arxiv{1506.03798}.

    \bibitem[L04]{L04}
    Thomas Lam,
    \textit{Ribbon Schur operators},
    European J. Combin. \textbf{29} (2008), 343-359.

    Also available as \arxiv{math/0409463}.



\end{thebibliography}

\end{document}
