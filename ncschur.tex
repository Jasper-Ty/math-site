% !TEX TS-program = xelatex

\documentclass{article}

\newcommand{\ip}[1]{
    \left\langle
        {#1}
    \right\rangle
}

\newcommand*\colword{\operatorname{\mathsf{colword}}}
\newcommand*\frkJ{\mathfrak{J}}
\newcommand*\SSYT{\operatorname{SSYT}}
\newcommand*\sgn{\operatorname{sgn}}

% See https://github.com/Jasper-Ty/dotfiles
\usepackage[garamond]{jaspercommon}

\barenv[][section]{theorem}{Theorem}
\barenv[bartheorem]{proposition}{Proposition}
\barenv[bartheorem]{corollary}{Corollary}
\barenv[bartheorem]{lemma}{Lemma}
\barenv[bartheorem]{definition}{Definition}
\barenv[bartheorem]{convention}{Convention}

\title{Noncommutative Schur functions}
\author{Jasper Ty}
\date{}

\titleauthorhead

\begin{document}

\maketitle

\section*{What is this?}

This is (going to be) an ``infinite napkin'' set of notes I am taking about noncommutative Schur functions.

\tableofcontents

\newpage

\section{Ideals and words}

Let $\bfu = (u_1,\ldots,u_N)$ be a collection of variables.
Let $\langle \bfu \rangle$ be the free semigroup on the generators $\bfu$.
Then, let $\calU = \ZZ\langle\bfu\rangle$ denote the corresponding semigroup ring--- the free associative ring generated by $\bfu$.

We will denote by $\calU^\ast$ the $\ZZ$-module spanned by words in the alphabet $\{1,\ldots,N\}$.

We will have a fundamental pairing $\ip{\wc,\wc}$ given by making noncommutative monomials dual to words.

Now, if $I$ is an ideal of $\calU$, we define $I^\perp$ by
\[
    I^\perp
    \coloneq
    \{
        \gamma \in \calU^\ast
        \mid
        \ip{I, \gamma} = 0
    \}.
\]
So that the elements of $\calU/I$ are in correspondence with words in $I^\perp$.

\section{
    Noncommutative 
    \texorpdfstring{$e$}{e}'s
    and
    \texorpdfstring{$h$}{h}'s
}

\begin{definition}
    The \defstyle{noncommutative elementary symmetric function} $e_k(\bfu)$ is defined to be
    \[
        e_k(\bfu)
        \coloneq
        \sum_{i_1 > i_2 > \cdots > i_k}
        u_{i_1} u_{i_2} \cdots u_{i_k}. 
    \]
    The \defstyle{noncommutative complete homogeneous symmetric function} $h_k(\bfu)$ is defined to be
    \[
        h_k(\bfu)
        \coloneq
        \sum_{i_1 \geq i_2 \geq \cdots \geq i_k}
        u_{i_1} u_{i_2} \cdots u_{i_k}. 
    \]
\end{definition}




\subsection{Commutation relations}

\begin{lemma}
    Let $I$ be an ideal.
    If the $e$'s commute modulo $I$, then the $h$'s commute modulo $I$ as well, and vice versa.
\end{lemma}

\begin{definition}
    We define the ideal $I_C$ to be the ideal consisting of exactly the elements
    \begin{align}
        &
        u_b^2u_a + u_au_bu_a - u_bu_au_b - u_bu_a^2 
        &
        (a<b),
        \\
        &
        u_bu_cu_a + u_au_cu_b - u_bu_au_c - u_cu_au_b
        &
        (a<b<c),
        \\
        &
        u_cu_bu_cu_a + u_bu_cu_au_c - u_cu_bu_au_c - u_bu_c^2u_a
        &
        (a<b<c).
    \end{align}
\end{definition}

\begin{theorem}
    $I_C$ is the smallest ideal in which the elementary symmetric functions $e_k(\bfu_S)$ and $e_\ell(\bfu_S)$ commute for any $k,\ell,S$.
\end{theorem}

\section{
    Noncommutative Schur functions
}

\begin{definition}
    The \defstyle{noncommutative Schur function} $\frkJ(\bfu)$ is defined to be
    \[
        \frkJ_\lambda(\bfu)
        \coloneq
        \sum_{T \in \SSYT(\lambda;N)}
        \bfu^{\colword T}.
    \]
\end{definition}

\begin{theorem}
    [\cite{FG98}, \cite{BF16}]
    In the ideal $I_\varnothing$,
    \[
        \frkJ_\lambda(\bfu)
        =
        \sum_{\pi \in S_{t}}
        \sgn(\pi)
        e_{\lambda'_1+\pi(1)-1}(\bfu)
        e_{\lambda'_2+\pi(2)-2}(\bfu)
        \cdots
        e_{\lambda'_{t}+\pi(t)-t}(\bfu),
    \]
    where $t = \lambda_1$.
\end{theorem}

\begin{corollary}
    [\cite{FG98}]
    If $I$ contains the ideal $I_\varnothing$, the map 
    \begin{align*}
        \Lambda_n
        &\to
        \calU/I
        \\
        s_\lambda
        &\mapsto
        \frkJ_\lambda
    \end{align*}
    defines an algebra homomorphism.
\end{corollary}

\subsection{Cauchy kernel}

\begin{definition}
    Let $\bfx = (x_1,x_2\ldots)$ be a countable collection of commuting variables.
\end{definition}

\section{Stuff}

\begin{definition}
    A \defstyle{combinatorial representation} of $\calU/I$ is
\end{definition}

\begin{definition}
\end{definition}


\section{Appendix}

\subsection{Gessel's fundamental quasisymmetric function}

\subsection{The Edelman-Greene correspondence}

\begin{thebibliography}{999999}
    \raggedright\footnotesize

    \bibitem[FG98]{FG98}
    Sergey Fomin and Curtis Greene, 
    \textit{Noncommutative Schur functions and their applications}, 
    Discrete Math. \textbf{193} (1998), 179-200.

    \bibitem[BF16]{BF16}
    Jonah Blasiak and Sergey Fomin, 
    \textit{Noncommutative Schur functions, switchboards, and Schur positivity},
    Sel. Math. \textbf{23} (2017), 727-766.

    Also available as \arxiv{1510.00657}.

    \bibitem[A15]{A15}
    Sami Assaf,
    \textit{Dual equivalence graphs I: A new paradigm for Schur positivity},
    Forum. Math. Sigma \textbf{3} (2015), e12.

    Also available as \arxiv{1506.03798}.

    \bibitem[L04]{L04}
    Thomas Lam,
    \textit{Ribbon Schur operators},
    European J. Combin. \textbf{29} (2008), 343-359.

    Also available as \arxiv{math/0409463}.



\end{thebibliography}

\end{document}
