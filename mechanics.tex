\documentclass{article}

\usepackage[garamond]{jaspercommon}
\usepackage{physics}

\DeclareMathOperator{\SO}{SO}

\title{Mechanics}
\author{Jasper Ty}
\date{}

\titleauthorhead
\setcounter{MaxMatrixCols}{20}

\begin{document}

\maketitle

These are notes written as I self-studied Mechanics by Landau and Lifshitz.

I'm a math student, but I didn't bother to ``upgrade'' anything to be more rigorous.

\tableofcontents

\section{Equations of Motion}

\subsection{Generalized coordinates}
\subsection{Least action}

\begin{definition}
    The path taken in going from $\mathbf{q}(t_1)$ to $\mathbf{q}(t_2)$ must satisfy
    \[
        \delta S = \delta \int_{t_1}^{t_2} L(\mathbf{q},\dot{\mathbf{q}},t)\dd x = 0.
    \]
    This is the \textit{principle of least action}, or \textit{Hamilton's principle}.
\end{definition}

\begin{theorem}
    Let $\mathbf{q}(t) = \big(q_1(t), \ldots, q_s(t)\big)$.
    Then equations $\mathbf{q}$ must satisfy so that $\delta S = 0$ are
    \[
        \dv{t} \left(\pdv{L}{\dot{q_i}}\right) - \pdv{L}{q_i} = 0, \qquad 1 \leq i \leq s.
    \]
    This is called the \textit{Euler-Lagrange equation}.
\end{theorem}

\begin{proof}
    We note that
    \[
        L(q + \delta q, \dot{q} + \delta \dot{q}, t) \approx L(q,\dot{q},t) + \underbrace{\pdv{L}{q}\delta q + \pdv{L}{\dot{q}} \delta \dot{q}}_{\text{First order terms}}.
    \]
    Then
    \begin{align*}
        \delta S &= \delta \int_{t_1}^{t_2} L(\mathbf{q},\dot{\mathbf{q}},t) \dd t \\
                 &= \int_{t_1}^{t_2} \left(\pdv{L}{\mathbf{q}}\delta\mathbf{q} + \pdv{L}{\dot{\mathbf{q}}}\delta\dot{\mathbf{q}}\right)\dd t \\
                 &= \int_{t_1}^{t_2} \pdv{L}{\mathbf{q}}\delta\mathbf{q} \dd t + \underbrace{\int_{t_1}^{t_2} \pdv{L}{\dot{\mathbf{q}}}\delta\dot{\mathbf{q}}\dd t}_{\text{integrate by parts}} \\
                 &= \int_{t_1}^{t_2} \pdv{L}{\mathbf{q}}\delta\mathbf{q} \dd t + \left(\underbrace{\left[\pdv{L}{\dot{\mathbf{q}}}\delta \mathbf{q}\right]_{t_1}^{t_2}}_{=\delta\mathbf{q}(t_2)-\delta\mathbf{q}(t_1)=0} - \int_{t_1}^{t_2} \left(\dv{t}\pdv{L}{\dot{\mathbf{q}}}\right)\delta\mathbf{q}\dd t\right) \\
                 &= \int_{t_1}^{t_2} \pdv{L}{\mathbf{q}}\delta\mathbf{q} \dd t  - \int_{t_1}^{t_2} \left(\dv{t}\pdv{L}{\dot{\mathbf{q}}}\right)\delta\mathbf{q}\dd t \\
                 &= \int_{t_1}^{t_2} \left(\pdv{L}{\mathbf{q}} - \dv{t}\pdv{L}{\dot{\mathbf{q}}}\right)\delta\mathbf{q}\dd t,
    \end{align*}
    and this integral must be zero for \textit{any} variation $\delta q$ effected.
    So it must be that
    \[
        \dv{t} \left(\pdv{L}{\dot{\mathbf{q}}}\right) - \pdv{L}{\mathbf{q}} = 0.
    \]
\end{proof}

\subsection{Galilean transformations}

\begin{definition}
    The \textit{Galilean group} is the group of transformations of $\RR^3 \times \RR$ generated by the following families of transformations:
    \begin{enumerate}[label=(\alph*)]
        \item 
            Space translations:
            \[
                (\mathbf{x}, t) \mapsto (\mathbf{x}+\mathbf{a}, t), \quad \mathbf{a} \in \RR^3
            \]
        \item
            Time translations:
            \[
                (\mathbf{x}, t) \mapsto (\mathbf{x}, t+s), \quad s \in \RR
            \]
        \item
            Uniform motion:
            \[
                (\mathbf{x}, t) \mapsto (\mathbf{x} + t\mathbf{v}, t), \quad \mathbf{v} \in \RR^3
            \]
        \item
            Rotations:
            \[
                (\mathbf{x}, t) \mapsto (R\mathbf{x}, t), \quad R \in \SO(3)
            \]
    \end{enumerate}
\end{definition}
    

\end{document}
