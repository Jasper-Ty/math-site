\documentclass{article}

\usepackage[garamond, tableau]{jaspercommon}

\newcommand*\Sym{\mathrm{Sym}}
\newcommand*\RCF{\mathsf{RCF}}
\newcommand*\OrdField{\mathsf{OrdField}}

\barenv{theorem}{Theorem}
\barenv[bartheorem]{definition}{Definition}
\barenv[bartheorem]{proposition}{Proposition}
\barenv[bartheorem]{lemma}{Lemma}

\title{Hillar-Nie 2006}
\author{Jasper Ty}
\date{}

\titleauthorhead

\begin{document}

\maketitle

\section*{What is this?}

This are notes I took while reading ``An elementary and constructive solution to Hilbert's 17th Problem for matrices'' by Christopher J. Hillar and Jiawang Nie \cite{HN06}.

This was for the class ``Positive Polynomials and Sums of Squares'' I took Winter 2024.

\tableofcontents

\section*{Notation}

Let $\bfx = (x_1,\ldots,x_m)$ be a collection of indeterminates, and let $F[\bfx]$ and $F(\bfx)$ denote the polynomial ring and the ring of rational functions with coefficients in the field $F$ respectively.

For any subset $S$ of a commutative ring $R$, let $\Sigma S^2$ denote the \defstyle{sums of squares} of $S$, i.e
\[
    \Sigma S^2
    \coloneq
    \left\{
        \sum_{i=1}^k r_i^2: r_1,\ldots,r_k \in S
    \right\}.
\]
Similarly, let $R^2$ denote the \defstyle{squares} of $R$.

Let $R^{d \times d}$ denote the set of $d \times d$ matrices in the ring $R$. And, let $\Sym_d(R)$ denote the subset of $R^{d \times d}$ consisting of symmetric matrices.

If $A$ is a matrix and $J \subseteq \{1,\ldots,n\}$, let $A[J]$ denote the principal submatrix with indices picked out by $J$.

\section{Introduction}

We seek to exposit the proof given in \cite{HN06} of the following result:

\begin{theorem}[Procesi-Schacher, Gondard-Ribenboim]
    \label{thm:1} 
    Let $A \in \RR^{d \times d}[\bfx]$ be symmetric.
    Let $A(\bfx_0)$ denote $A$ with all entries evaluated at $\bfx_0 \in \RR^d$.
    If $A(\bfx_0)$ is positive semidefinite for all choices of $\bfx_0 \in \RR^m$, then $A$ is a sum of squares of 
\end{theorem}

This generalizes Artin's celebrated, classical result on nonnegative polynomials with real coefficients. 

\begin{theorem}[Artin's solution to Hilbert's 17th Problem]
    \label{thm:2}
    Let $f \in \RR[\bfx]$.
    The following are equivalent:
    \begin{enumerate}[label=(\roman*)]
        \item 
            $f(\bfx) \geq 0$ for all $\bfx$.
        \item 
            $f \in \Sigma \RR(\bfx)^2$.
    \end{enumerate}
\end{theorem}

We will prove the more general statement, which proves Theorem \ref{thm:1} with the help of Theorem \ref{thm:2}.

\begin{theorem}
    \label{thm:3}
    Let $F$ be a real field, and let $A \in \Sym_d(F)$ such that $\det A[J] \in \Sigma F^2$ for all $J \subseteq \{1,\ldots,n\}$.
Then $A \in \Sigma \big[\Sym_d(F)\big]^2$.
\end{theorem}

\begin{proof}[Proof that Theorem \ref{thm:3} implies Theorem \ref{thm:1}]
    Let $A \in \Sym_d(\RR[\bfx])$.

    We will first show that all principal minors of $A$ are in fact non-negative polynomials.
    We note that for all matrices $H \in \RR[\bfx]^{d \times d}$, $(\det H)(\bfx_0)=\det\big(H(\bfx_0)\big)$ and $H[J](\bfx_0) = H(\bfx_0)[J]$ for all $J \subseteq [n]$.
    In other words, \textit{taking determinants and taking submatrices commutes with evaluation}.
    So, if $J \subseteq [n]$,
    \[
        \big(\det A[J]\big)(\bfx_0)
        = \det \big(A[J](\bfx_0)\big)
        = \det \big(A(\bfx_0)[J]\big)
        \geq 0,
    \]
    for all $\bfx_0 \in \RR^d$ supposing that $A(\bfx_0)$ is positive semidefinite for all $\bfx_0 \in \RR^d$.
    Then we may apply Theorem \ref{thm:2} to $\det A[J] \in \RR[\bfx]$, to conclude that $\det A[J] \in \Sigma \RR(\bfx)^2$.

    Now, take $A$ to live in $\Sym_d(\RR(\bfx))$, where we are simply extending the inclusion of $\RR[\bfx]$ into $\RR(\bfx)$, then we can apply Theorem \ref{thm:3}, with $F = \RR(\bfx)$, to say that $A \in \Sigma\big[\Sym_d(\RR(\bfx))\big]^2$.
\end{proof}

\section{Review of real algebra}

We will recover basic results in the theory of real symmetric matrices in the more general context of real closed fields.

First, a small digression about ordering.
The data involving the order in an ordered field can be encoded as a set that names all the positive elements.

\begin{definition}
    Let $P$ be a subset of $F$.
    We say that $P$ is a \defstyle{ordering} of $F$ if
    \begin{enumerate}[label=(\alph*)]
        \item 
            $P + P \subseteq P$,
        \item 
            $P \cdot P \subseteq P$,
        \item 
            $F^2 \subseteq P$,
        \item
            $-1 \notin P$, and
        \item 
            $P \cup -P = F$.
    \end{enumerate}
\end{definition}

If one has an ordered field $F$, then one has an ordering $P$ by considering all the elements $p \in F$ such that $p \geq 0$.
Conversely, if one has a field and an ordering $P$ and a field $F$, one can make $F$ an ordered field by putting $p \geq 0$ for all $p \in P$.

Now, we discuss real closed fields.

\begin{definition}
The \defstyle{first order language of ordered fields} $\OrdField$ consists of well-formed sentences involving the usual logical symbols and connectives, as well as the non-logical symbols $+$, $\cdot$, $0$, $1$, $\ ^{-1}$, $\leq$.

    A \defstyle{real closed field} is an ordered field for which a sentence $\psi$ in $\OrdField$ is true if and only if it is true over $\RR$.
\end{definition}

This is not the usual definition of a real closed field.
We will discuss a few important, equivalent definitions.

\begin{theorem}
    [Artin-Schreier 1926]
    \label{thm:ArtinSchreier1926}
    Let $F$ be a field.
    The following are equivalent:
    \begin{enumerate}[label=(\roman*)]
        \item 
            $-1 \notin \Sigma F^2$, and $-1 \in \Sigma G^2$ for any nontrivial algebraic extension $G$ of $F$.
        \item 
            $F^2$ is an ordering of $F$, and every odd degree polynomial with coefficients in $F$ has a root in $F$.
        \item 
            $F \neq F[\sqrt{-1}]$, and $F[\sqrt{-1}]$ is algebraically closed.
    \end{enumerate}
\end{theorem}

\begin{proof}
    See Theorem 1.2.9 in \cite{NetzerRAG}
\end{proof}


\begin{theorem}
    [Tarski 19??]
    Let $F$ be a field.
    The following are equivalent:
    \begin{enumerate}[label=(\roman*)]
        \item 
            $F$ is real closed.
        \item 
            $F$ satisfies any of the statements in Theorem \ref{thm:ArtinSchreier1926}.
    \end{enumerate}
\end{theorem}

\begin{proof}
    We will define $\RCF$ to be the \defstyle{theory of real closed fields}, to be the field axioms adjoined with (the correct encoding of) statement (ii) in Theorem \ref{thm:ArtinSchreier1926}.

    One can prove \defstyle{quantifier elimination} is possible in $\RCF$, and moreover algorithmically possible, hence $\RCF$ is a decidable theory.
    Moreover, one can show that $\RCF$ can prove or disprove any quantifier free statement in $\OrdField$, hence $\RCF$ is complete.
    Lastly, $\RR \models \RCF$, so by basic model theory, if $R \models \RCF$, $R$ and $\RR$ are elementarily equivalent, i.e, they agree on all sentences in $\OrdField$.
\end{proof}

With the logic out of the way, we can begin to glean some properties of real closed fields.

\begin{proposition}
    [The ordering on $\RCF$s]
    \label{thm:RCFOrder}
    In a real closed field $R$, the set $R^2$ identifies all the positive elements.
\end{proposition}

\begin{proof}
    Consider the $\OrdField$ sentences
    \[
        \forall y (y^2 \geq 0)
    \]
    and
    \[
        \forall x \Big(x \geq 0 \iff \exists y (x = y^2)\Big),
    \]
    which are evidently true in $\RR$.
\end{proof}

\begin{proposition}
    [Characterizations of PSD matrices over an RCF]
    \label{prop:RCFpsd}
    Let $R$ be a real-closed field and let $A \in \Sym_d(R)$.
    The following are equivalent
    \begin{enumerate}[label=(\roman*)]
        \item 
            All the principal minors of $A$ are nonnegative.
        \item 
            $\bfx^T A \bfx \geq 0$ for all $\bfx \in R^d$.
        \item 
            $A$ is diagonalizable with nonnegative eigenvalues.
    \end{enumerate}
\end{proposition}

\begin{proof}
    If we fix $d$, we may completely encode the statement (i) $\implies$ (ii) in $\OrdField$, hence its truth in $R$ coincides with its truth in $\RR$.

    As an example, put $d = 2$.
    Then our statement in the first order language of ordered fields is
    \begin{align*}
        &\forall a,b,c,d \\
        &\Bigg[
            \underbrace{
                \Big(
                    a \geq 0 
                    \wedge d \geq 0
                    \wedge ad - bc \geq 0
                \Big)
            }_{\text{nonnegative principal minors}}
            \implies
            \underbrace{
                \forall x,y
                \left(
                    ax^2 + (b+c)xy + dy^2 \geq 0
                \right)
            }_{\text{positive-semidefiniteness}}
        \Bigg].
    \end{align*}
    Similarly, we may do (ii) $\implies$ (iii).

    The statement ``$\begin{pmatrix}a & b \\ c & d\end{pmatrix}$ is diagonalizable with nonnegative eigenvalues'', in the $d=2$ case, is \footnote{Trust me}
    \begin{align*}
        &\exists e,f,g,h \\
        &\Bigg[
        eh - fg = 1
        \wedge e^2b + efd - fea + f^2c = 0
        \wedge g^2b + ghd - hga + h^2c = 0
        \\
        &\wedge hea + hfc - geb - gfd \geq 0
        \wedge egb + ehd - fga - fhc \geq 0
        \Bigg]
    \end{align*}
    The point is, we can encode the whole theorem for a fixed $d$ entirely as a sentence in $\OrdField$.
    Then, we use the fact that the theorem is true for real symmetric matrices.
\end{proof}

Next, we discuss weaker objects than real closed fields, which we will need.

\begin{definition}
    A \defstyle{real field} is a field $F$ in which $-1 \notin \Sigma F^2$.
\end{definition}

\begin{proposition}
    All real fields $F$ have at least one ordering $\leq$ such that $(F, \leq)$ is an ordered field.
    Moreover, when equipped with such an order, there exists an ordered field $R$ such that $R$ is real closed, $R$ is algebraic extension of $K$, and the order on $R$ extends the order on $F$.
    We call $R$ a \defstyle{real closure} of $F$.
\end{proposition}

\begin{proof}
    Theorem 1.4.2 in \cite{NetzerRAG}.
\end{proof}

\section{Proof of the theorem}

We will need the following lemma.

\begin{lemma}
    \label{lem:4}
    Let $F$ be a real field and suppose $A$ satisfies the hypotheses of Theorem 3; $A \in \Sym_d(F)$ such that $\det A[J] \in \Sigma F^2$ for all $J \subseteq \{1,\ldots,n\}$.

    Then the minimal polynomial $m(t) \in F[t]$ of $A$ is of the form:
    \[
        m(t)
        =
        \sum_{i=0}^k
        (-1)^{k-i}a_it^i
        =
        t^k - a_{k-1}t^{k-1} + \cdots + (-1)^ka_0.
    \]
    where $a_i \in \Sigma F^2$ for all $i$.
    Moreover, $a_1 \neq 0$.
\end{lemma}

\begin{proof}
    This proof happens fairly quickly in \cite{HN06}.
    We will spend some more detail on this.
    \begin{itemize}
        \item[\textbf{Step 1}]
            \textsc{\color{Crimson} Characterize sums of squares in terms of nonnegativity in real closures}

            Sums of squares play a special role in real fields $K$.
            We have that
            \[
                \tag{\cite{NetzerRAG} Theorem 1.1.16}
                \Sigma K^2
                =
                \bigcap_{\substack{\text{$P$ is an}\\\text{ordering of $K$}}} P.
            \]
            One can interpret this as saying that they are the elements that will \textit{always} be positive regardless of the order one realizes on $K$.
            So, if $x \in \Sigma F^2$, that means that $x \geq 0$ in \textit{any} ordering of $F$. 
            In fact, if $x \geq 0$ in any real closure, then this means that $x \in \Sigma K^2$, as $x \geq 0$ in a real closure $R$ means that $x \in P$ in some ordering $P$ of $F$ which $R$ extends.
            We conclude:
            \begin{center}
                \textit{If $x \geq 0$ in all real closures of $F$, then $x \in \Sigma F^2$, and conversely.}
            \end{center}
            Then the path ahead is clear: \textit{we want to show that $a_i \geq 0$ in all real closures $R$ of $F$}.

        \item[\textbf{Step 2}]
            \textsc{\color{Crimson} Show that $A$ has nonnegative eigenvalues in every real closure}

            If $R$ is a real closure of $F$, all the principal minors $\det A[J]$ of $A$ are nonnegative in $R$, as, by the hypothesis, they are sums of squares in $F$, hence they are sums of squares in $R$, and the nonnegative elements of $R$ are precisely the squares (Theorem \ref{thm:RCFOrder}), so $\det A[J]$ is a sum of nonnegative elements of $R$.

            Then, we have the following:
            \begin{center}
                \textit{In any real closure of $F$, all the principal minors of $A$ are nonnegative}.
            \end{center}
            Now, combined with \ref{prop:RCFpsd}, this statement reads
            \begin{center}
                \textit{In any real closure of $F$, $A$ is diagonalizable with nonnegative eigenvalues}.
            \end{center}

        \item[\textbf{Step 3}]
            \textsc{\color{Crimson} Prove the lemma}

            Each $a_i$ is a sum of products of eigenvalues of $A$.
            (Specifically, it is an elementary symmetric polynomial in the distinct eigenvalues of $A$, since $A$ is diagonalizable).

            Then $a_i$ is nonnegative in every real closure $R$ of $A$, as we have shown that its eigenvalues in $R$ are nonnegative.
            But, as we have noted, this means that $a_i$ is a sum of squares in $F$!
            This completes the proof of the first statement.
    \end{itemize}

    Finally, we complete the theorem by proving the second statement.

    Since $A$ is diagonalizable, $m(t)$ has no repeated roots, hence $0$ can only appear at most once. 
    This means that there is exactly $1$ term in $a_1$, the $k-1$th elementary symmetric polynomial in the roots of $m(t)$, that avoids this zero and is hence positive, hence $a_1 \neq 0$.
\end{proof}

There is a formula in \cite{HornJohnson} that expresses the characteristic polynomial directly in terms of principal minors, and I'm sure it simplifies this proof, but I haven't had the time to try it.

We are now ready to prove the main theorem.

\begin{proof}
    [Proof of Theorem \ref{thm:3}]
    Let $F$ be a real field and let $A \in \Sym_d(F)$ be a matrix whose principal minors are all nonnegative.

    Let $m(t) = t^k - a_{k-1}t^{k-1} + \cdots + (-1)^k a_0$ be the minimal polynomial of $A$.

    Then, by Cayley-Hamilton, $m(A) = 0$, so by splitting the even and odd degree terms, 
    \[
        (A^{k-1} + a_{k-2}A^{k-3} + \cdots + a_1I)A
        =
        a_{m-1}A^{k-1} + a_{m-3}A^{m-3} + \cdots + a_0I.
    \]
    Now put $B = A^{k-1} + a_{k-2}A^{k-3} + \cdots + a_1I \in \Sym_d(F)$.
    $B$ is invertible, since $a_1 = 0$, hence it does not have $0$ as an eigenvalue.
    Moreover, $B$'s inverse is also symmetric, i.e $B^{-1} \in \Sym_d(F)$.

    Then, $B^{-1} = B \cdot B^{-2} = B \cdot (B^{-1})^2$, so
    \[
        A
        =
        B\Big(
            a_{k-1}B^{-2}A^{k-1}
            + a_{k-3}B^{-2}A^{k-3}
            + \cdots
            + a_0 B^{-2}
        \Big).
    \]
    Everything ``in sight'' is a sum of squares.
    \begin{itemize}
        \item
            All coefficients $a_i \in F$ appearing are sums of squares; $a_i \in \Sigma F^2$.
        \item
            Each $A^{k-2l}$ term is a square, as $k$ is odd; $A^{k-2l} \in \big[\Sym_d(F)\big]^2$. 
        \item 
            $B$ itself is a sum of squares, as $B = A^{k-1} + a_{k-2}A^{k-3} + \cdots + a_1I$, and $k$ is odd; $B \in \Sigma \big[\Sym_d(F)\big]^2$
        \item 
            And finally, $B^{-2} = (B^{-1})^2 \in \big[\Sym_d(F)\big]^2$.
    \end{itemize}
    So, in all, $A \in \Sigma \big[\Sym_d(F)\big]^2$.
    The $k$ even case is similarly argued.
\end{proof}

\begin{thebibliography}{999999}
    \raggedright\footnotesize

    \bibitem[H\&J]{HornJohnson}
    Roger Horn, Charles Johnson, 
    \textit{Matrix Analysis}.

    \bibitem[N]{NetzerRAG}
    Tim Netzer, 
    \textit{Real Algebra and Geometry}, 
    Universit\"at Innsbruck, 
    \url{https://www.uibk.ac.at/mathematik/algebra/media/teaching/ragen.pdf}.

    \bibitem[S]{SchweighoferRAG}
    Markus Schweighofer, 
    \textit{Real Algebraic Geometry, Positivity and Convexity}, 
    Universit\"at Konstanz, 
    \url{https://cloud.uni-konstanz.de/index.php/s/3bcgdnT2eXR9EKF}.

    Also available as \arxiv{2205.04211}.

    \bibitem[HN06]{HN06}
    Christopher Hillar and Jiewang Nie,
    \textit{An elementary and constructive solution to Hilbert's 17th Problem for Matrices},
    Proc. Amer. Math. Soc. \textbf{136} (2008), 73-76.

    \url{https://doi.org/10.1090/S0002-9939-07-09068-5}. 

    Also available as \arxiv{math/0610388v3}.
\end{thebibliography}

\end{document}
