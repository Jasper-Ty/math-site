\documentclass{article}

\usepackage{mathrsfs}
\usepackage{amsmath}
\usepackage{amssymb}
\usepackage[margin=0.5in]{geometry}

\newenvironment{myboxed}{\noindent\begin{tabular}{|p{.975\linewidth}|}\hline \\}{\\\\\hline\end{tabular}}
\newenvironment{mytitle}{\noindent\large\begin{flushright}}{\end{flushright}\normalsize}


\def\theChapter{1}

\newcounter{Answer}
\newenvironment{Answer} {\refstepcounter{Answer}\par\noindent\textbf{A\theChapter.\theAnswer:}} {\bigskip}

\begin{document}

\begin{mytitle}
    Calculus \\
    Chapter 1.1 \\
    \normalsize Jasper Ty
\end{mytitle}
\begin{Answer}
    \begin{itemize}
        \item[(a)]
            $f(-1) = -2$
        \item[(b)]
            $f(2) \approx 1.8$
        \item[(c)]
            $f(x) = 2$ at $x = 1$ and $x = -3$
        \item[(d)]
            $f(x) = 0$ at $x \approx -2.5$ and $x \approx 0.4$
        \item[(e)]
            The domain of $f$ is $[-3, 3]$ and the range of $f$ is $[-2, 3]$
        \item[(f)]
            $f$ is increasing on $[-1, 3]$
    \end{itemize}
\end{Answer}

\begin{Answer}
    \begin{itemize}
        \item[(a)] $f(-4) = -2$ and $g(3) = 4$
        \item[(b)]
            $f(x) = g(x)$ at $x = -2$ and $x = 2$
        \item[(c)]
            $f(x) = -1$ at $x = -3$ and $x = 4$
        \item[(d)]
            $f$ is decreasing on $[0, 4]$
        \item[(e)]
            The domain of $f$ is $[-4, 4]$ and the range of $f$ is $[-2, 3]$
        \item[(f)]
            The domain of $g$ is $[-4, 3]$ and the range of $g$ is $[0.5, 4]$
    \end{itemize}
\end{Answer}

\begin{Answer}
    The range of the vertical ground acceleration function is about $[-70, 120]$ 

    The range of the north-south ground acceleration function is about $[-300, 420]$ 

    The range of the east-west ground acceleration function is about $[-200, 200]$ 
\end{Answer}

\begin{Answer}
    Skip
\end{Answer}

\begin{Answer}
    Is not a function. Fails the vertical line test. 
\end{Answer}

\begin{Answer}
    Is a function. The domain is $[-2, 2]$ and the range is $[-1, 2]$
\end{Answer}

\begin{Answer}
    Is a function. The domain is $[-3, 2]$ and the range is $[-3, -2) \cup [-1, 3]$
\end{Answer}

\begin{Answer}
    Is not a function. Fails the vertical line test.
\end{Answer}

\begin{Answer}
    Skip
\end{Answer}

\begin{Answer}
    Skip
\end{Answer}

\begin{Answer}
    Skip
\end{Answer}

\begin{Answer}
    Skip
\end{Answer}

\begin{Answer}
    Skip
\end{Answer}

\begin{Answer}
    Skip
\end{Answer}

\begin{Answer}
    Skip
\end{Answer}

\begin{Answer}
    Skip
\end{Answer}

\begin{Answer}
    Skip
\end{Answer}

\begin{Answer}
    Skip
\end{Answer}

\begin{Answer}
    \begin{itemize}
        \item $f(2) = 3(2)^2 - 2 + 2 = 3\cdot 4 = 12$
        \item $f(-2) = 3(-2)^2 - (-2) + 2 = 3\cdot 4 + 4 = 16$
        \item $f(a) = 3a^2 - a + 2$
        \item $f(-a) = 3(-a)^2 - (-a) + 2 = 3a^2 + a + 2$
        \item $f(a+1) = 3(a+1)^2 - (a+1) + 2 = 3(a^2 + 2a + 1) - a + 3 = 3a^2 + 5a + 6$
        \item $2f(a) = 2(3a^2 - a + 2) = 6a^2 - 2a + 4$
        \item $f(2a) = 3(2a)^2 - 2a + 2 = 3\cdot 4a^2 - 2a + 2 = 12a^2 - 2a + 2$
        \item $f(a^2) = 3(a^2)^2 - 2a^2 + 2 = 3a^4 - 2a^2 + 2$
        \item $[f(a)]^2 = [3a^2 - a + 2]^2 = 9a^4 + a^2 + 4 - 3a^3 + 6a^2 - 2a = 9a^4 + 7a^2 - 3a^3 - 2a + 4$
        \item $f(a+h) = 3(a+h)^2 - (a+h) + 2 = 3(a^2 + 2ah + h^2) - a - h + 2 = 3a^2 - 2ah + h^2 - a - h + 2$
    \end{itemize}
\end{Answer}

\begin{Answer}
    Let $f(r) := V(r+1) - V(r)$ 
    \begin{align*}
        f(r) &= \frac{4}{3} \pi (r+1)^3 - \frac{4}{3} \pi r^3 \\
             &= \frac{4}{3}\pi \bigg( (r+1)^3 - r^3\bigg) \\
             &= \frac{4}{3}\pi \bigg( r^3 + 3r^2 + 3r + 1 - r^3\bigg) \\
             &= \frac{4}{3}\pi \bigg(3r^2 + 3r + 1\bigg) \\
    \end{align*}
\end{Answer}

\begin{Answer}
    \begin{align*}
        f(2+h) &= (2+h) - (2+h)^2 \\
               &= 2 + h - 4 + 4h + h^2 \\
               &= h^2 + 5h - 2
    \end{align*}
    \begin{align*}
        f(x+h) &= (x+h) - (x+h)^2 \\
               &= x + h - x^2 + 2hx + h^2 \\
               &= h^2 + h + 2hx - x^2 + x
    \end{align*}
    \begin{align*}
        \frac{f(x+h)-f(x)}{h} &= \frac{(h^2 + h + 2hx - x^2 + x) - (x - x^2)}{h} \\
        &= \frac{h^2 + h + 2hx}{h} \\
        &= h + 1 + 2x
    \end{align*}
\end{Answer}

\begin{Answer}
    \begin{align*}
        f(2+h) &= \frac{2+h}{2+h+1} \\
               &= \frac{2+h}{3+h}
    \end{align*}
    \begin{align*}
        f(x+h) &= \frac{x+h}{x+h+1}
    \end{align*}
    \begin{align*}
        \frac{f(x+h) - f(x)}{h} &= \frac{\frac{x+h}{x+h+1} - \frac{x}{x+1}}{h} \\
                                &= h^{-1} \left( \frac{(x+h)(x+1)}{(x+h+1)(x+1)} - \frac{x(x+h+1)}{(x+h+1)(x+1)}\right) \\
                                &= h^{-1} \left(\frac{x^2 + hx + x + h - x^2 - hx - x}{x^2 + hx + x + x + h + 1}\right) \\
                                &= h^{-1} \left(\frac{h}{x^2 + hx + 2x + h + 1}\right) \\
                                &= \frac{1}{x^2 + 2x + hx + h + 1}
    \end{align*}
\end{Answer}

\begin{Answer}
    The domain of $f$ is $\mathbb{R} - \{1/3\}$
\end{Answer}

\begin{Answer}
    The domain of $f$ is $\mathbb{R} - \{-2, -1\}$
\end{Answer}

\begin{Answer}
    The domain of $f$ is $[0, \infty)$
\end{Answer}

\begin{Answer}
    The domain of $g$ is $[0, 4]$
\end{Answer}

\begin{Answer}
    The domain of $h$ is all $x$ for which $x^2 - 5x > 0$, so either $x < 0$ or $x > 5$. The domain is $(-\infty, 0) \cup (5, \infty)$
\end{Answer}

\begin{Answer}
    The domain of this function is $[-2, 2]$, the range of this function is $[0, 2]$. I can't sketch it, but the graph is the upper half of a circle of radius $2$ centered at the origin.
\end{Answer}

\begin{Answer}
    Domain: $\mathbb{R}$
\end{Answer}

\begin{Answer}
    Domain: $\mathbb{R}$
\end{Answer}

\begin{Answer}
    Domain: $\mathbb{R}$
\end{Answer}

\begin{Answer}
    Domain: $\mathbb{R} - \{2\}$
\end{Answer}

\begin{Answer}
    Domain: $[5, \infty)$
\end{Answer}

\begin{Answer}
    Domain: $\mathbb{R}$
\end{Answer}

\begin{Answer}
    Domain: $\mathbb{R} - \{0\}$
\end{Answer}

\begin{Answer}
    Domain: $\mathbb{R} - \{0\}$
\end{Answer}

\begin{Answer}
    Domain: $\mathbb{R}$
\end{Answer}

\begin{Answer}
    Domain: $\mathbb{R}$
\end{Answer}

\begin{Answer}
    Domain: $\mathbb{R}$
\end{Answer}

\begin{Answer}
    Domain: $\mathbb{R}$
\end{Answer}

\begin{Answer}
    The slope is $(-6 -1) / 4 - (-2) = -7/6$. The domain is $[-2, 4]$. So 
    \[f(x) = -7/6(x+2) + 1 \qquad -2 \leq x \leq 4\]
\end{Answer}

\begin{Answer}
    \[f(x) = -5/9(x+3) + 1 \qquad -3 \leq x \leq 6\]
\end{Answer}

\begin{Answer}
    \[f(x) = -\sqrt{-x} + 1 \qquad x \leq 0\]
\end{Answer}

\begin{Answer}
    \[f(x) = +\sqrt{(1 - (x-1)^2}\]
\end{Answer}

\begin{Answer}
    \[f(x) = \begin{cases}
        x + 1 & -1 \leq x < 2 \\
        -\frac{3}{2}(x-2) & 2 \leq x \leq 4\\
    \end{cases}\]
\end{Answer}

\begin{Answer}
    \[f(x) = \begin{cases}
        -2x + 2 & 0 \leq x < 1 \\
        x - 1 & 1 \leq x \\
    \end{cases}\]
\end{Answer}

\begin{Answer}
    Let $l$ denote length and $w$ denote width. Then a perimeter of $20$ means
    \[20 = 2l + 2w\]
    The area of a rectangle is $A = lw$. We eliminate $w$ by solving for $w$ in the constraint above. We have that $w = (20 - 2l)/2 = 10 - l$. So our function is
    \[f(l) = l(10 - l)\]
\end{Answer}

\begin{Answer}
    Same as above, but solve for $w$ using the area constraint: $w = 16/l$. Then
    \[f(l) = 2l + 2\frac{16}{l} = 2l + \frac{32}{l}\]
\end{Answer}

\begin{Answer}

\end{Answer}

\end{document}
