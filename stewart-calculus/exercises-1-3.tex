\documentclass{article}

\usepackage{mathrsfs}
\usepackage{amsmath}
\usepackage{amssymb}
\usepackage[margin=0.5in]{geometry}

\newenvironment{myboxed}{\noindent\begin{tabular}{|p{.975\linewidth}|}\hline \\}{\\\\\hline\end{tabular}}
\newenvironment{mytitle}{\noindent\large\begin{flushright}}{\end{flushright}\normalsize}

\def\theChapter{1.3}

\newcounter{Answer}
\newenvironment{Answer} {\refstepcounter{Answer}\par\noindent\textbf{A\theChapter.\theAnswer:}} {\bigskip}

\begin{document}

\begin{mytitle}
    Calculus \\
    Chapter 1.3 \\
    \normalsize Jasper Ty
\end{mytitle}

\begin{Answer}
    \begin{itemize}
        \item[(a)]
            $f(x) + 3$
        \item[(b)]
            $f(x) - 3$
        \item[(c)]
            $f(x-3)$
        \item[(d)]
            $f(x+3)$
        \item[(e)]
            $-f(x)$
        \item[(f)]
            $f(-x)$
        \item[(g)]
            $3f(x)$
        \item[(h)]
            $f(x)/3$
    \end{itemize}
\end{Answer}

\begin{Answer}
    \begin{itemize}
        \item[(a)]
            Vertical stretch by a factor of $5$
        \item[(b)]
            Shift right by $5$
        \item[(c)]
            Flip along $x$ axis
        \item[(d)]
            Flip along $x$ axis and stretch by a factor of $5$
        \item[(e)]
            Horizontal shrink by a factor of $5$
        \item[(f)]
            Vertical stretch by a factor of $5$ then shift down $3$
    \end{itemize}
\end{Answer}

\begin{Answer}
    \begin{itemize}
        \item[(a)]
            3
        \item[(b)]
            1
        \item[(c)]
            4
        \item[(d)]
            5
        \item[(e)]
            2
    \end{itemize}
\end{Answer}

\begin{Answer}
    Skip
\end{Answer}

\begin{Answer}
    Skip
\end{Answer}

\begin{Answer}
    $y = 2\sqrt{3(x-2) - (x-2)^2}$
\end{Answer}

\begin{Answer}
    $y = -\sqrt{3(-x-1) - (-x-1)^2} - 1$
\end{Answer}

\begin{Answer}
    \begin{itemize}
        \item[(a)]
            Vertical stretch by factor of two
        \item[(b)]
            Shift up by one
    \end{itemize}
\end{Answer}

\begin{Answer}
    Skip
\end{Answer}

\begin{Answer}
    Skip
\end{Answer}

\begin{Answer}
    Skip
\end{Answer}

\begin{Answer}
    Skip
\end{Answer}

\begin{Answer}
    Skip
\end{Answer}

\begin{Answer}
    Skip
\end{Answer}

\begin{Answer}
    Skip
\end{Answer}

\begin{Answer}
    Skip
\end{Answer}

\begin{Answer}
    Skip
\end{Answer}

\begin{Answer}
    Skip
\end{Answer}

\begin{Answer}
    Skip
\end{Answer}

\begin{Answer}
    Skip
\end{Answer}

\begin{Answer}
    Skip
\end{Answer}

\begin{Answer}
    Skip
\end{Answer}

\begin{Answer}
    Skip
\end{Answer}

\begin{Answer}
    Skip
\end{Answer}

\begin{Answer}
    Skip
\end{Answer}

\begin{Answer}
    Skip
\end{Answer}

\begin{Answer}
    \begin{itemize}
        \item[(a)]
            The graph of $f\big(|x|\big)$ takes the part of $f$'s graph on the positive half of the plane and mirrors it onto the negative half of the plane.
        \item[(b)]
        \item[(c)]
    \end{itemize}
\end{Answer}

\begin{Answer}
    Skip
\end{Answer}

\begin{Answer}
    Skip
\end{Answer}

\begin{Answer}
    Skip 
\end{Answer}

\begin{Answer}
    Domain of $f+g$: $\mathbb{R}$
    \[f+g = x^3 + 2x^2 + 3x^2 - 1 = x^3 + 5x^2 - 1\]
    Domain of $f-g$: $\mathbb{R}$
    \[f - g = x^3 - x^2 + 1\]
    Domain of $fg$: $\mathbb{R}$
    \[fg = 3x^5 + 6x^4 - x^3 - 2x^2\]
    Domain of $f/g$: $\mathbb{R} - \{+\sqrt{1/3} , -\sqrt{1/3}\}$
    \[f/g = \frac{x^3-2x^2}{3x^2-1}\]
\end{Answer}

\begin{Answer}
    Domain of $f+g$: $[-1, 1]$
    \[f+g = \sqrt{1+x} + \sqrt{1-x}\]
    Domain of $f-g$: $[-1, 1]$
    \[f-g = \sqrt{1+x} - \sqrt{1-x}\]
    Domain of $fg$: $[-1, 1]$
    \[fg = \sqrt{1-x^2}\]
    Domain of $f/g$: $[-1, 1)$
    \[f/g = \frac{\sqrt{1+x}}{\sqrt{1-x}}\]
\end{Answer}

\begin{Answer}
    Skip
\end{Answer}

\begin{Answer}
    Skip
\end{Answer}

\begin{Answer}
    The domain of $f$ is $\mathbb{R}$, the range of $f$ is $[-1/8, \infty)$. The domain of $g$ is $\mathbb{R}$, and the range of $g$ is $\mathbb{R}$
    
    Firstly, the domain of $f \circ g$ is $\mathbb{R}$, but the range of $f \circ g$ is $[-1/8, \infty)$
    \begin{align*}
        (f \circ g) (x) &= 2[g(x)]^2 - g(x) \\
                        &= 2(3x+2)^2 - (3x+2) \\
                        &= 2(9x^2 + 6x + 4) - 3x - 2 \\
                        &= 18x^2 + 12x - 3x + 8 - 2 \\
                        &= 18x^2 + 9x - 6
    \end{align*}

    The domain of $g \circ f$ is $\mathbb{R}$, and the range of $g \circ f$ is $[13/8, \infty)$
    \begin{align*}
        (g \circ f)(x) &= 3f(x) + 2 \\
                       &= 3(2x^2 - x) + 2 \\
                       &= 6x^2 - 2x + 2
    \end{align*}

    The domain of $f \circ f$ is $\mathbb{R}$, and its range is again $[-1/8, \infty)$
    \begin{align*}
        (f \circ f)(x) &= 2[f(x)]^2 - f(x) \\
                       &= 2(2x^2-x)^2 - (2x^2-x) \\
                       &= 2(4x^4 - 4x^3 + x^2) - 2x^2 + x \\
                       &= 8x^4 - 4x^3 + 2x^2 - 2x^2 + x \\
                       &= 8x^4 - 4x^3 + x
    \end{align*}

    The domain of $g \circ g$ is $\mathbb{R}$, and the range of $g \circ g$ is $\mathbb{R}$
    \begin{align*}
        (g \circ g)(x) &= 3[g(x)] + 2 \\
                       &= 3(3x+2) + 2 \\
                       &= 9x + 6 + 2 \\
                       &= 9x + 8
    \end{align*}
\end{Answer}


\begin{Answer}
    The domain of $f$ is $\mathbb{R}$ and the range of $f$ is $\mathbb{R}$. The domain of $g$ is $\mathbb{R} - \{0\}$ and the range of $g$ is $\mathbb{R} - \{0\}$.

    The domain of $f \circ g$ is $\mathbb{R} - \{0\}$, and the range of $f \circ g$ is $\mathbb{R} - \{1\}$
    \begin{align*}
        (f \circ g)(x) &= 1 - [g(x)]^3 \\
                       &= 1 - \left(\frac{1}{x}\right)^3 \\
                       &= 1 - \frac{1}{x^3}
    \end{align*}

    The domain of $g \circ f$ is $\mathbb{R} - \{1\}$. The range of $g \circ f$ is $\mathbb{R} - \{0\}$.
    \begin{align*}
        (g \circ f)(x) &= \frac{1}{f(x)} \\
                       &= \frac{1}{1 - x^3} \\
    \end{align*}

    The domain of $f \circ f$ is $\mathbb{R}$ and the range of $f \circ f$ is $\mathbb{R}$
    \begin{align*}
        (f \circ f)(x) &= 1 - [f(x)]^3 \\
                       &= 1 - (1 - x^3) \\
                       &= x^3
    \end{align*}

    The domain of $g \circ g$ is $\mathbb{R} - \{0\}$. The range of $g \circ g$ is $\mathbb{R} - \{0\}$
    \begin{align*}
        (g \circ g)(x) &= \frac{1}{g(x)} \\
                       &= \frac{1}{\frac{1}{x}} \\
                       &= x
    \end{align*}
\end{Answer}

\begin{Answer}
    $f: \mathbb{R} \to [-1, 1]$, and $g: [0, \infty) \to (-\infty, 1]$

    $f \circ g: [0, \infty) \rightarrow [-1, 1]$
    \begin{align*}
        (f \circ g)(x) &= \sin (g(x)) \\
                       &= \sin (1 - \sqrt{x}) 
    \end{align*}


    $g \circ f: \{x \in R \: : \: 2k\pi < x < (2k+1)\pi\} \rightarrow [0, 1]$
    \begin{align*}
        (g \circ f)(x) &= 1 - \sqrt{f(x)} \\
                       &= 1 - \sqrt{\sin x}
    \end{align*}

    $f \circ f: \mathbb{R} \to [\sin(-1), \sin(1)]$
    \begin{align*}
        (f \circ f)(x) &= \sin (f(x)) \\
                       &= \sin \sin x
    \end{align*}

    $g \circ g: [0, \infty) \to [0, 1]$
    \begin{align*}
        (g \circ g)(x) &= 1 - \sqrt{g(x)} \\
                       &= 1 - \sqrt{1 - \sqrt{x}}
    \end{align*}
\end{Answer}

\begin{Answer}
    $f: \mathbb{R} \rightarrow \mathbb{R}$, and $g: \mathbb{R} \to [31/20, \infty)$

    $f \circ g: \mathbb{R} \to (-\infty, -73/20]$
    \begin{align*}
        (f \circ g)(x) &= 1 - 3g(x) \\
                       &= 1 - 3(5x^2 + 3x + 2) \\
                       &= 1 - 15x^2 - 9x - 6 \\
                       &= -15x^2 - 9x - 5
    \end{align*}

    $g \circ f: \mathbb{R} \rightarrow [31/20, \infty)$
    \begin{align*}
        (g \circ f)(x) &= 5[f(x)]^2 + 3f(x) + 2 \\
                       &= 5(1-3x)^2 + 3(1-3x) + 2 \\
                       &= 5(1 - 6x + 9x^2) + 3 - 9x + 2 \\
                       &= 5 - 30x + 45x^2 + 3 - 9x + 2 \\
                       &= 45x^2 - 39x + 10 
    \end{align*}

    $f \circ f: \mathbb{R} \rightarrow \mathbb{R}$
    \begin{align*}
        (f \circ f)(x) &= 1 - 3f(x) \\
                       &= 1 - 3(1-3x) \\
                       &= 1- (3 - 9x) \\
                       &= 1 - 3 + 9x \\
                       &= 9x - 2
    \end{align*}

    $g \circ g: \mathbb{R} \rightarrow [1493/80, \infty)$
    \begin{align*}
        (g \circ g)(x) &= 5[g(x)]^2 + 3g(x) + 2 \\
                       &= 5(5x^2+3x+2)^2 + 3(5x^2+3x+2) + 2 \\
                       &= 5(25x^4 + 30x^3 + 29x^2 + 12x + 4) + 15x^2 + 9x + 6 + 2 \\
                       &= 125x^4 + 150x^3 + 145x^2 + 60x + 20 + 15x^2 + 9x + 6 + 2 \\
                       &= 125x^4 + 150x^3 + 160x^2 + 69x + 28
    \end{align*}
\end{Answer}

\begin{Answer}
    $f: \mathbb{R} - \{0\} \to (-\infty, -2] \cup [2, \infty)$, $g: \mathbb{R} -\{-2\} \rightarrow \mathbb{R} - \{1\}$

    $f \circ g: \mathbb{R} - \{0\} \rightarrow \{2\}$
    \begin{align*}
        (f \circ g)(x) &= g(x) + \frac{1}{g(x)} \\
                       &= \frac{x+1}{x+2} + \frac{x+2}{x+1} \\
                       &= \frac{(x+1)(x+2) + (x+2)(x+1)}{(x+1)(x+2)} \\
                       &= 2
    \end{align*}

    $g \circ f: \mathbb{R} - \{0\} \to [3/4, \infty)$
    \begin{align*}
        (g \circ f)(x) &= \frac{f(x) + 1}{f(x) + 2} \\
                       &= \frac{x + \frac{1}{x} + 1}{x + \frac{1}{x} + 2} \\
                       &= \frac{\frac{x^2}{x} + \frac{1}{x} + \frac{x}{x}}{\frac{x^2}{x} + \frac{1}{x} + \frac{2x}{x}} \\
                       &= \frac{\frac{x^2+x+1}{x}}{\frac{x^2+2x+1}{x}} \\
                       &= \frac{x^2+x+1}{x^2+2x+1}
    \end{align*}

    $f \circ f: \mathbb{R} - \{0\} \to$
    \begin{align*}
        (f \circ f) &= f(x) + \frac{1}{f(x)} \\
                    &= \left(x + \frac{1}{x}\right) + \frac{1}{x + \frac{1}{x}} \\
                    &= \frac{x^2+1}{x} +\frac{x}{x^2+1} \\
                    &= \frac{(x^2+1)^2 + x^2}{x(x^2+1)} \\
                    &= \frac{x^4 + 2x^2 + 1 + x^2}{x^3 + x} \\
                    &= \frac{x^4 + 3x^2 + 1}{x^3 + x}
    \end{align*}

    $g \circ g: \mathbb{R} \to$
    \begin{align*}
        (g \circ g) &= \frac{g(x) + 1}{g(x) + 2} \\
                    &= \frac{\frac{x+1}{x+2} + 1}{\frac{x+1}{x+2} +2} \\
                    &= \frac{\frac{x+1 + x+2}{x+2}}{\frac{x+1 + 2x+4}{x+2}} \\
                    &= \frac{2x+3}{3x+5}
    \end{align*}
\end{Answer}

\begin{Answer}

    Domain: $\mathbb{R}$
    \begin{align*}
        (f \circ g) &= \sqrt{2g(x) + 3} \\
                    &= \sqrt{2(x^2 +1) + 3} \\
                    &= \sqrt{2x^2 + 2 + 3} \\
                    &= \sqrt{2x^2 + 5}
    \end{align*}

    Domain: $[-3/2, \infty)$
    \begin{align*}
        (g \circ f) &= [f(x)]^2 + 1 \\
                    &= (\sqrt{2x+3})^2 + 1 \\
                    &= 2x+3 + 1 \\
                    &= 2x + 4
    \end{align*}

    Domain: $[-3/2, \infty)$
    \begin{align*}
        (f \circ f) &= \sqrt{2f(x) + 3} \\
                    &= \sqrt{2\sqrt{2x+3} + 3}
    \end{align*}

    Domain: $\mathbb{R}$
    \begin{align*}
        (g \circ g) &= [g(x)]^2 + 1 \\
                    &= (x^2+1)^2 + 1 \\
                    &= (x^4 + 2x^2 + 1) + 1 \\
                    &= x^4 + 2x^2 + 2
    \end{align*}
\end{Answer}

\begin{Answer}
    \begin{align*}
        (f \circ g \circ h)(x) &= (g \circ h)(x) + 1 \\
                               &= 2h(x) + 1 \\
                               &= 2(x-1) + 1 \\
                               &= 2x - 1
    \end{align*}
\end{Answer}

\begin{Answer}
    \begin{align*}
        (f \circ g \circ h)(x) &= 2(g \circ h)(x) - 1 \\
                               &= 2[h(x)]^2 + 1 \\
                               &= 2(1-x)^2 + 1 \\
                               &= 2(x^2 - 2x + 1) + 1 \\
                               &= 2x^2 - 2x + 2
    \end{align*}
\end{Answer}

\begin{Answer}
    \begin{align*}
        (f \circ g \circ h)(x) &= \sqrt{(g \circ h)(x) - 1} \\
                               &= \sqrt{[h(x)]^2 + 2 - 1} \\
                               &= \sqrt{(x+3)^2 + 1} \\
                               &= \sqrt{x^2 + 6x + 10}
    \end{align*}
\end{Answer}

\begin{Answer}
    \begin{align*}
        (f \circ g \circ h)(x) &= \frac{2}{(g \circ h)(x) + 1} \\
                               &= \frac{2}{\cos h(x) + 1} \\
                               &= \frac{2}{\cos \sqrt{x+3} + 1}
    \end{align*}
\end{Answer}

\begin{Answer}
    Let $f(x) := x^{10}$, $g(x) := x^2+1$
\end{Answer}

\begin{Answer}
    Let $f(x) := \sin x$, $g(x) := \sqrt{x}$
\end{Answer}

\begin{Answer}
    Let $f(x) := x/(x+4)$, $g(x) := x^2$
\end{Answer}

\begin{Answer}
    Let $f(x) := 1/x$, $g(x) := x + 3$
\end{Answer}

\begin{Answer}
    Let $f(t) := \sqrt{t}$, $g(t) := \cos t$
\end{Answer}

\begin{Answer}
    Let $f(t) := t/(1+t)$, $g(t) := \tan t$
\end{Answer}

\begin{Answer}
    Let $f(x) := 1-x$, $g(x) := 3^x$, $h(x) := x^2$
\end{Answer}

\begin{Answer}
    Let $f(x) := \sqrt{x}$, $g(x) := x-1$, $h(x) := \sqrt{x}$
\end{Answer}

\begin{Answer}
    Let $f(x) := x^4$, $g(x) := \sec x$, $h(x) := \sqrt{x}$
\end{Answer}

\begin{Answer}
    \begin{itemize}
        \item[(a)]
            $f(g(1)) = f(6) = 5$
        \item[(b)]
            $g(f(1)) = g(3) = 2$
        \item[(c)]
            $f(f(1)) = f(3) = 4$
        \item[(d)]
            $g(g(1)) = g(6) = 3$
        \item[(e)]
            $(g \circ f)(3) = g(f(3)) = g(4) = 1$
        \item[(f)]
            $(f \circ g)(6) = f(g(6)) = f(3) = 4$
    \end{itemize}
\end{Answer}

\begin{Answer}
    \begin{itemize}
        \item[(a)]
            $f(g(2)) = f(5) = 4$
        \item[(b)]
            $g(f(0)) = g(0) = 3$
        \item[(c)]
            $(f \circ g)(0) = f(g(0)) = f(3) = 0$
        \item[(d)]
            $(g \circ f)(6) = g(f(6)) = g(6)$ which does not exist
        \item[(e)]
            $(g \circ g)(-2) = g(g(-2)) = g(1) = 4$
        \item[(f)]
            $(f \circ f)(4) = f(f(4)) = f(2) = -2$
    \end{itemize}
\end{Answer}

\begin{Answer}
    Skip
\end{Answer}

\begin{Answer}
    \begin{itemize}
        \item[(a)]
            $r(t) = 60t$
        \item[(b)]
            $(A \circ r)(t) = \pi \cdot (60t)^2 = 3600\pi t^2$
    \end{itemize}
\end{Answer}

\begin{Answer}
    \begin{itemize}
        \item[(a)]
            $d(t) = 350t$
        \item[(b)]
            $s(d) = \sqrt{1 + d^2}$
        \item[(c)]
            $s(d(t)) = \sqrt{1 + 350^2 t^2}$
    \end{itemize}
\end{Answer}

\begin{Answer}
    Skip
\end{Answer}

\begin{Answer}
    Skip
\end{Answer}

\begin{Answer}
    \begin{itemize}
        \item[(a)]
            Let $f(x) := x^2 + 6$. Then $(f \circ g)(x) = (2x+1)^2 + 6 = 4x^2 + 4x + 1 + 6 = 4x^2 + 4x + 7 = h(x)$
        \item[(b)]
            Let $g(x) := x^2 + x + 1$. Then $(f \circ g)(x) = 3(x^2 + x + 1) + 2 = 3x^2 + 3x + 3 + 2 = 3x^2 + 3x + 5 = h(x)$
    \end{itemize}
\end{Answer}

\begin{Answer}
    Let $g(x) := 4x - 17$. Then $(g \circ f)(x) = 4(x+4) - 17 = 4x + 16 - 17 = 4x - 1 = h(x)$
\end{Answer}

\begin{Answer}
    Yes. $h(x) = f(g(x)) = f(g(-x)) = h(-x)$
\end{Answer}

\begin{Answer}
    No. Cheap proof: let $f$ be a nonzero constant. If $f$ is odd, then $h$ is odd, since $h(-x) = f(g(-x)) = f(-g(x)) = -f(g(x)) = -h(x)$. If $f$ is even, then $h$ is actually even, since $h(-x) = f(g(-x)) = f(-g(x)) = f(g(x)) = h(x)$
\end{Answer}
\end{document}
