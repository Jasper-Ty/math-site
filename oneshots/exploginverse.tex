\documentclass{article}

\usepackage{jaspercommon}

\DeclareMathOperator{\sign}{sign}
\DeclareMathOperator{\len}{len}

\title{exp is the inverse of log}
\author{Jasper Ty}
\date{}

\barenv[][section]{theorem}{Theorem}
\barenv[bartheorem]{definition}{Definition}
\barenv[bartheorem]{proposition}{Proposition}

\titleauthorhead

\begin{document}

\maketitle

\tableofcontents

\section{Introduction}

\begin{definition}
    The power series $\exp$ and $\log$ are defined
    \begin{align}
        \exp(x)
        &\coloneq
        \sum_{n=0}^\infty
        \frac{x^n}{n!},
        \\
        \log(x)
        &\coloneq
        \sum_{n=1}^\infty
        (-1)^{n+1}
        \frac{(x-1)^n}{n}.
    \end{align}
\end{definition}

We wish to show the following:

\begin{theorem}
    \label{thm}
    \begin{equation}
        (\exp \circ \log)(x)
        =
        (\log \circ \exp)(x)
        =
        x.
    \end{equation}
\end{theorem}

You can show this using Lagrange inversion.
However, we will do this with our bare hands.

Here is a small amount of some ``boilerplate'' we'll need:

\begin{definition}
    Let $a_1,\ldots,a_k \geq 0$, and let $n = a_1 + \cdots + a_k$.
    The corresponding \textit{multinomial coefficient} for this tuple is
    \[
        \binom{n}{a_1,\ldots,a_k}
        \coloneq
        \frac{n!}{a_1! \cdots a_k!}.
    \]
\end{definition}

\begin{proposition}
    The multinomial coefficient 
    \[
        \binom{n}{a_1,\ldots,a_k}
        \coloneq
        \frac{n!}{a_1! \cdots a_k!}
    \]
    counts the number of partitions of the set $\{1, \ldots, k\}$ into parts $A_1, \ldots, A_k$, such that $|A_1| = a_1, \ldots, |A_k| = a_k$.
\end{proposition}

\begin{proposition}[``Power rule'']
    \begin{equation}
        \label{eq:powerrule}
        \left(
            \sum_{m=1}^\infty
            c_mx^m
        \right)^n
        =
        \sum_{m=0}^\infty
        \left[
            \sum_{
                \substack{
                    a_1,\ldots,a_n \geq 1 \\
                    a_1 + \cdots + a_n = m
                }
            }
            c_{a_1} \cdots c_{a_n}
        \right]
        x^m
    \end{equation}
\end{proposition}

\section{Proof}

We first grind out a raw expression for the coefficients of $\log \circ \exp$, \textit{as an exponential generating function}, then demonstrate that each coefficient is a signed sum of \textit{cyclically ordered set partitions}--- we then demonstrate cancellation.

\begin{proof}
    [Proof of Theorem \ref{thm}]
    \begin{align*}
        &
        \underbrace{
            \log
        }_{\text{expand}}
        \Big(\exp(x)\Big)
        \\
        &=
        \sum_{n=1}^\infty
        (-1)^{n+1}
        \frac{
            \Big(
                \overbrace{
                    \exp(x)
                }^{\text{expand}}
                -1
            \Big)^n
        }{n}
        \\
        &=
        \sum_{n=1}^\infty
        (-1)^{n+1}
        \frac{
            \overbrace{
                \left(
                    \sum_{m=1}^\infty
                    \frac{x^m}{m!}
                \right)^n
            }^{\text{Use ``power rule'' (\ref{eq:powerrule})}}
        }{n}
        \\
        &=
        \sum_{n=1}^\infty
        (-1)^{n+1}
        \frac{
            \left[
                \sum_{m=0}^\infty
                \sum_{
                    \substack{
                        a_1,\ldots,a_n \geq 1 \\
                        a_1 + \cdots + a_n = m
                    }
                }
                \frac{1}{a_1! \cdots a_n!}
                x^m
            \right]
        }{
            \underbrace{n}_{\text{break up fraction}}
        }
        \\
        &=
        \sum_{n=1}^\infty
        \underbrace{
            \frac{(-1)^{n+1}}{n}
        }_{\text{move into innermost sum}}
        \sum_{m=0}^\infty
        \sum_{
            \substack{
                a_1,\ldots,a_n \geq 1 \\
                a_1 + \cdots + a_n = m
            }
        }
        \frac{1}{a_1! \cdots a_n!}
        x^m
        \\
        &=
        \underbrace{
            \sum_{n=1}^\infty
            \sum_{m=0}^\infty
        }_{\text{interchange sums}}
        \sum_{
            \substack{
                a_1,\ldots,a_n \geq 1 \\
                a_1 + \cdots + a_n = m
            }
        }
        \frac{(-1)^{n+1}}{n}
        \frac{1}{a_1! \cdots a_n!}
        x^m
        \\
        &=
        \sum_{m=0}^\infty
        \underbrace{
            \sum_{n=1}^\infty
            \sum_{
                \substack{
                    a_1,\ldots,a_n \geq 1 \\
                    a_1 + \cdots + a_n = m
                }
            }
        }_{\text{this quantifies over all \textit{strong compositions of} $m$}}
        \frac{(-1)^{n+1}}{n}
        \frac{1}{a_1! \cdots a_n!}
        x^m
        \\
        &=
        \sum_{m=0}^\infty
        \sum_{
            \substack{
                a_1,\ldots,a_n \text{ is a} \\
                \text{strong composition of } m
            }
        }
        \frac{(-1)^{n+1}}{n}
        \underbrace{
            \frac{1}{a_1! \cdots a_n!}
            x^m
        }_{\text{multiply by }1 = m!/m!}
        \\
        &=
        \sum_{m=0}^\infty
        \sum_{
            \substack{
                a_1,\ldots,a_n \text{ is a} \\
                \text{strong composition of } m
            }
        }
        \frac{(-1)^{n+1}}{n}
        \underbrace{
            \frac{m!}{a_1! \cdots a_n!}
        }_{
            =\binom{m}{a_1,\ldots,a_n}
        }
        \frac{x^m}{m!}
        \\
        &=
        \sum_{m=0}^\infty
        \sum_{
            \substack{
                a_1,\ldots,a_n \text{ is a} \\
                \text{strong composition of } m
            }
        }
        \frac{(-1)^{n+1}}{n}
        \underbrace{
            \binom{m}{a_1,\ldots,a_n}
        }_{\text{convert multiplicity into summation}}
        \frac{x^m}{m!}
        \\
        &=
        \sum_{m=0}^\infty
        \underbrace{
            \sum_{
                \substack{
                    a_1,\ldots,a_n \text{ is a} \\
                    \text{strong composition of } m
                }
            }
            \sum_{
                \substack{
                    A_1,\ldots,A_n \text{ is a partition} \\
                    \text{of }\{1,\ldots,m\} \\
                    |A_1| = a_1, \ldots, |A_n| = a_n
                }
            }
        }_{\text{this quantifies over all \textit{ordered set partitions} of} \{1,\ldots,m\}}
        \frac{(-1)^{n+1}}{n}
        \frac{x^m}{m!}
        \\
        &=
        \sum_{m=0}^\infty
        \sum_{
            \substack{
                A_1,\ldots,A_n \text{ is a partition} \\
                \text{of }\{1,\ldots,m\}
            }
        }
        \underbrace{
            \frac{(-1)^{n+1}}{n}
        }_{
            \substack{
                \text{this adds a cyclic symmetry,} \\
                \text{which we will now prove }
            }
        }
        \frac{x^m}{m!}
        \\
        &=
        \sum_{m=0}^\infty
        \underbrace{
            \sum_{
                \substack{
                    A_1,\ldots,A_n \text{ is a partition} \\
                    \text{of }\{1,\ldots,m\}
                }
            }
        }_{\text{break apart parts and order}}
        \frac{(n-1)!}{n!}
        (-1)^{n+1}
        \frac{x^m}{m!}
        \\
        &=
        \sum_{m=0}^\infty
        \sum_{
            \substack{
                \bfA \text{ is an unordered} \\
                \text{partition of }\{1,\ldots,m\}
            }
        }
        \underbrace{
            \sum_{
                \substack{
                    A_1,\ldots,A_n \text{ is an ordering} \\
                    \text{of $\bfA$'s parts}
                }
            }
        }_{\text{there are $n!$ such orders}}
        \frac{(n-1)!}{n!}
        (-1)^{n+1}
        \frac{x^m}{m!}
        \\
        &=
        \sum_{m=0}^\infty
        \sum_{
            \substack{
                \bfA \text{ is an unordered} \\
                \text{partition of }\{1,\ldots,m\}
            }
        }
        n!
        \frac{(n-1)!}{n!}
        (-1)^{n+1}
        \frac{x^m}{m!}
        \\
        &=
        \sum_{m=0}^\infty
        \sum_{
            \substack{
                \bfA \text{ is an unordered} \\
                \text{partition of }\{1,\ldots,m\}
            }
        }
        \underbrace{
            (n-1)!
        }_{
            \substack{
                \text{convert multiplicity into sum--- this is} \\
                \text{the \# of cyclic orders of length $n$}
            }
        }
        (-1)^{n+1}
        \frac{x^m}{m!}
        \\
        &=
        \sum_{m=0}^\infty
        \sum_{
            \substack{
                \bfA \text{ is an unordered} \\
                \text{partition of }\{1,\ldots,m\}
            }
        }
        \sum_{
            \substack{
                A_{\underline{1}},\ldots,A_{\underline{n}}\text{ is a cyclic} \\
                \text{ordering of $\bfA$'s parts}
            }
        }
        (-1)^{n+1}
        \frac{x^m}{m!}
        \\
        &=
        \sum_{m=0}^\infty
        \underbrace{
            \left[
                \sum_{
                    \substack{
                        A_{\underline{1}},\ldots,A_{\underline{n}} \text{ is a cyclically ordered} \\
                        \text{partition of }\{1,\ldots,m\}
                    }
                }
                (-1)^{n+1}
            \right]
        }_{\text{voil\`a!}}
        \frac{x^m}{m!}
    \end{align*}
    Now for this series to be equal to $x$, it must be that
    \[
        \left[
            \sum_{
                \substack{
                    A_{\underline{1}},\ldots,A_{\underline{n}} \text{ is a cyclically ordered} \\
                    \text{partition of }\{1,\ldots,m\}
                }
            }
            (-1)^{n+1}
        \right]
        =
        \begin{cases}
            1, & \text{if } m=1; \\
            0, & \text{otherwise}.
        \end{cases}
    \]
    We will prove this using a \defstyle{sign-reversing involution}--- a function which records exactly how terms cancel out in a sum.
    \begin{quote}
        \textbf{Cancellation principle.}
        Let $X$ be a finite set and let $\sign: X \to \RR$.
        If $f: X \to X$ is a bijection such that $\sign f(x) = -\sign f(x)$ for all $x \in X$, then $\sum_{x \in X} \sign x = 0$.
    \end{quote}

    Now, if we define $\sign \bfA = (-1)^{\len \bfA + 1}$, we can prove that
    \[
        \left[
            \sum_{
                \substack{
                    \bfA \text{ is a cyclically ordered} \\
                    \text{partition of }\{1,\ldots,m\}
                }
            }
            \sign(\bfA)
        \right]
        =
        0
    \]
    for all $m > 1$ if we can find such a function.

    Let $f$ be the function defined by
    \[
        f(\bfA)
        =
        \begin{cases}
            A_{\underline{k}} \cup A_{\underline{k+1}}, \ldots, A_{\underline{n+k}}
            & \text{if } A_{\underline{k}} = \{1\} \\
            \{1\}, A_{\underline{k}} \setminus \{1\}, \ldots, A_{\underline{n+k}}
            & \text{if } A_{\underline{k}} \neq \{1\} \\
        \end{cases}
    \]
    where $A_{\underline{k}}$ is the part of $\bfA$ containing $1$. 

    Then, $\sign f(\bfA) = -\sign f(\bfA)$.
    \begin{itemize}
        \item
            In the first case, we are merging two parts, $A_{\underline{k}}$ and $A_{\underline{k+1}}$, decreasing the number of parts by one.
        \item 
            In the second case, we are splitting apart $A_{\underline{k}}$ into two parts--- one that contains $\{1\}$ and one that doesn't.
    \end{itemize}

    Moreover $f$ is an involution, therefore it is a bijection.
    \begin{itemize}
        \item
            If $A_{\underline{k}} = \{1\}$, then on applying $f$ it will be merged with the next nonempty part.
            Upon applying $f$ again, it will be split apart from this part, reversing what happened.
        \item 
            If $A_{\underline{k}} \neq \{1\}$, then on applying $f$ we get rid of all elements that aren't $1$ and put them in a new part.
            Upon applying $f$ again, these elements will be merged into our original part again, reversing what happened.
    \end{itemize}

    Moreover, this map is actually only defined whenever $m > 1$, as in the $m = 1$ case, it's impossible to do any splitting or merging of parts.

    This proves that
    \[
        \left[
            \sum_{
                \substack{
                    \bfA \text{ is a cyclically ordered} \\
                    \text{partition of }\{1,\ldots,m\}
                }
            }
            \sign(\bfA)
        \right]
        =
        \begin{cases}
            1, & \text{if } m=1; \\
            0, & \text{otherwise}.
        \end{cases}
    \]
    Hence
    \[
        (\log \circ \exp)(x)
        =
        \sum_{m=0}^\infty
        \left[
            \sum_{
                \substack{
                    A_{\underline{1}},\ldots,A_{\underline{n}} \text{ is a cyclically ordered} \\
                    \text{partition of }\{1,\ldots,m\}
                }
            }
            (-1)^{n+1}
        \right]
        \frac{x^m}{m!}
        =
        x,
    \]
    which completes the proof.
\end{proof}

\end{document}
