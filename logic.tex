\documentclass{article}

% See https://github.com/Jasper-Ty/dotfiles
\usepackage[garamond]{jaspercommon}

\newcommand{\powerset}[1]{\ensuremath{\mathscr{P}\left(#1\right)}}

\title{Mathematical logic notes}
\author{Jasper Ty}
\date{}

\titleauthorhead

\begin{document}

\maketitle

\section*{What is this?}

These are notes based on my reading of Mileti's ``Modern Mathematical Logic'', accompanied by sitting in Henry Towsner's \textsc{MATH 5700} class at Penn.

I mix in some other notation along with those in the book.
For example, I take $\PP$ to be the set of positive integers, and I prefer to notate sequences as symbols with subscripts rather than as functions.

The biggest differences are that I use $\_$ as a wildcard variable, and I like to use the $\mapsto$ notation to specify functions.

I have an intense dislike for \texttt{mathcal}, and actively try to replace uses of it whenver I can.

\tableofcontents

\setcounter{section}{1}
\section{Induction and recursion}

We put the notions of induction and recursion in a more general framework.

\subsection{Induction and recursion on \texorpdfstring{\NN}{N}}

First, we recall well-known avatars of induction and recursion.

\begin{definition}
    We define the \textit{successor function} $S$ to be
    \begin{align*}
        S: \NN &\to \NN \\
        n &\mapsto n+1.
    \end{align*}
\end{definition}

\begin{theorem}[Induction on $\NN$ -- steps]
    Let $X \subseteq N$ such that $0 \in X$ and $S(n) \in X$ whenever $n \in X$.
    Then it must be that $X = \NN$.
\end{theorem}

\begin{theorem}[Recursion on $\NN$ -- steps]
    Let $X$ be a set.
    If $y \in X$, and $g: \NN \times X \to X$, there exists a unique function $f: \NN \to X$ such that
    \begin{enumerate}[label=(\roman*)]
        \item $f(0) = y$, and
        \item $f(S(n)) = g(n,f(n))$ for all $n \in \NN$
    \end{enumerate}
\end{theorem}

\begin{theorem}[Induction theorem -- order]
    Let $X \subseteq N$ such that $0 \in X$ and for all $n \in X$, $m \in X$ whenever $m < n$.
    Then $X = \NN$.
\end{theorem}

The utility in the previous formalization of the recursion theorem is apparent in its order form.

\begin{theorem}[Recursion theorem -- order]
    \label{thm:NNRecursionByOrder}
    Let $X$ be a set.
    If $g: X^\ast \to X$, then there is a unique function $f$ such that
    \[
        f(n) = g(f \upharpoonright [n]).
    \]
\end{theorem}

\begin{example}[Fibonacci numbers]
    Let $X = \NN$, and define
    \begin{align*}
        g: \NN^\ast 
        &\to
        \NN \\
        \{a_i\}_{1\leq i \leq n}
        &\mapsto
        \begin{cases}
            0 & n = 0 \\
            1 & n = 1 \\
            a_{n-2} + a_{n-1} & n \geq 2 \\
        \end{cases}
    \end{align*}
    Then by Theorem \ref{thm:NNRecursionByOrder}, there is a unique function $f$ such that $f(n) = g(f \upharpoonright [n])$.
    We call this function the \textit{Fibonacci sequence}.
\end{example}

\subsection{Generation}

\begin{definition}
    Let $A$ be a set, and fix an \textit{arity} $k \in \PP$.
    We define \textit{$k$-ary functions on $A$} to be those functions of the form
    \[
        f: A^k \to A.
    \]
    Common shorthands are \textit{unary}, \textit{binary}, and \textit{ternary} functions for $1$-ary, $2$-ary, and $3$-ary functions respectively.
\end{definition}

\begin{definition}
    Let $A$ be a set, $B \subseteq A$, and let $\mathcal{H}$ be a collection such that each $h \in \mathcal{H}$ is a $\_$-ary function on $A$.

    We call $(A, B, \mathcal{H})$ a \textit{simple generating system}.

    To be able to pick out all the functions $h \in \mathcal{H}$ of a \textit{specific arity} $k \in \PP$, we denote the set of all such functions $\mathcal{H}_k$.
\end{definition}

The interpretation is that $A$ is our background set, and $B$ is the set which we wish to generate some larger set using all the operations of $\mathcal{H}$.

\begin{example}[Subgroups generated by a subset]
    Let $A$ be a group, and let $B \subset A$ be a subset that contain's the identity of $A$.

    We are interested in the \textit{subgroup generated by $A$ generated by $B$}.
    In this case, $\mathcal{H} = \{h_1,h_2\}$, where
    \begin{align*}
        h_1: A^2 &\to A \\
        (x,y) &\mapsto x \cdot y
    \end{align*}
    and 
    \begin{align*}
        h_2: A &\to A \\
        x &\mapsto x^{-1}.
    \end{align*}
\end{example}

\begin{example}[Vector subspaces generated by a subset]
    Now if $V$ is a vector space over an infinite field $\FF$, and $B \subseteq V$ is a subset that contains the zero vector, the correct $\mathcal{H}$ that identifies the \textit{subspace generated by $B$} is now an \textit{infinite family}.

    It contains vector addition, namely the map
    \begin{align*}
        h_+: V^2 &\to V \\
        (u,v) &\mapsto u + v
    \end{align*}
    and the \textit{scaling maps}
    \begin{align*}
        h_\alpha: V &\to V \\
        v &\mapsto \alpha v
    \end{align*}
    for all $\alpha \in \FF$.
\end{example}

\begin{definition}
    Fix $k \in \PP$.
    A \textit{set-valued $k$-ary function} is a $k$-ary function of the form
    \[
        h: A^k \to \powerset{\_}.
    \]
\end{definition}



\begin{definition}
Let $A$ be a set, $B \subseteq A$, and now let $\mathcal{H}$ be a collection of \textit{set-valued} $\_$-ary functions on $A$.

    We call $(A, B, \mathcal{H})$ a \textit{generating system}.

    Again, $\mathcal{H}_k$ denotes all $h \in \mathcal{H}$ of arity $k \in \PP$.
\end{definition}

\begin{example}[Subfields generated by a set]
    Let $\FF$ be a field, and let $B \subseteq \FF$ such that $0,1 \in B$.

    Let $\mathcal{H}$ be the collection of functions 
    \begin{align*}
        h_1: \FF^2 &\to \powerset{\FF} \\
        (a,b) &\mapsto \{a + b\} \\
        \\
        h_2: \FF^2 &\to \powerset{\FF} \\
        (a,b) &\mapsto \{a \cdot b\} \\
        \\
        h_3: \FF &\to \powerset{\FF} \\
        a &\mapsto \{-a\} \\
        \\
        h_4: \FF &\to \powerset{\FF} \\
        a &\mapsto \begin{cases}
            \{a^{-1}\} & a \neq 0 \\
            \varnothing & a = 0
        \end{cases}.
    \end{align*}

    Then $(\FF, B, \mathcal{H})$ identifies the \textit{subfield of $\FF$ generated by $B$}.
\end{example}

\begin{example}[Subgraphs generated by reachability]
    If $G = (V,E)$ is a directed graph, reachability can be characterized by $\mathcal{H} = \{h\}$, where
    \begin{align*}
        h: V &\to \powerset{V} \\
        v &\mapsto \{w \in V: (v,w) \in E\}.
    \end{align*}
\end{example}

We remark that every simple generating system can be represented as a generating system, by converting each $k$-ary function $h$ on $A$ to a set-valued $k$-ary function $h'$ on $A$ by putting
\begin{align*}
    h': A^k &\to \powerset{A} \\
    (a_1,\ldots,a_k) &\mapsto \{h(a_1,\ldots,a_k)\}.
\end{align*}

Now we explicitly define what it is exactly that a generating system generates.

\subsubsection*{From above}

\begin{definition}
    Let $(A,B,\mathcal{H})$ be a generating system.

    We call a subset $J$ of $A$ \textit{inductive} if
    \begin{enumerate}[label=(\roman*)]
        \item $B \subseteq J$.
        \item If $k \in \PP$, $h \in \mathcal{H}_k$, and $a_1,\ldots,a_k \in J$, then $h(a_1,\ldots,a_k) \subseteq J$.
    \end{enumerate}
\end{definition}

We do a common set-theoretic trick here--- we take intersections to get the smallest set in a family.

\begin{proposition}
    Let $(A, B, \mathcal{H})$ be a generating system.
    Then there exists a unique inductive set $I$ such that $I \subseteq J$ for all inductive sets $J$.
\end{proposition}

\begin{proof}
    We claim that
    \[
        I
        =
        \bigcap_{J\text{ is an inductive set}}J.
    \]
    Clearly, this satisfies $I \subseteq J$ for all inductive sets $J$.
    Next, we prove that it is inductive.

    Since $B \subseteq J$ for all inductive sets $J$, we have that $B \subseteq I$.

    Fix $k \in \PP$, and take some $h \in \mathcal{H}_k$.
    If $a_1,\ldots,a_k \in I$, then $a_1,\ldots,a_k \in J$ for all inductive sets $J$.
    Hence $h(a_1,\ldots,a_k) \subseteq J$ for all inductive sets $J$.
    By the same principle from which we concluded $B \subseteq I$, we must conclude that $h(a_1,\ldots,a_k) \subseteq I$.

    Finally, uniqueness follows from the fact that if $I_1$ and $I_2$ are inductive sets such that $I_1 \subseteq J$ and $I_2 \subseteq J$ for all inductive sets $J$, it must be that $I_1 \subseteq I_2$ and $I_2 \subseteq I_1$, and therefore $I_1 = I_2$.
\end{proof}


\subsubsection*{From below: levels}
    
This approach is more in the spirit of induction.

\begin{definition}
    Let $(A, B, \mathcal{H})$ be a generating system.
    We define a sequence $\{V_n\}_{n=0}^\infty$ of subsets of $A$ recursively as follows
    \begin{align*}
        V_0 &= B \\
        V_{n+1} &= V_n \cup \{c\in A; c = h(a_1,\ldots,a_k) \text{ for some } a_1,\ldots,a_k \in V_n, h \in \mathcal{H}_k\}.
    \end{align*}
\end{definition}

Namely, the $n$-th subset in the sequence is all the elements that can be obtained by applying functions in $\mathcal{H}$ \textit{at most $n$ times}.

\subsubsection*{From below: witnessing sequences}

\subsection{Step induction}

\begin{definition}[Step induction]
    Let $(A, B, \mathcal{H})$ be a generating system.
    If $X \subseteq A$ satisfies
    \begin{enumerate}[label=(\roman*)]
        \item $B \subseteq X$
        \item $h(a_1,\ldots,a_k) \in X$ whenever $k \in \PP$, $h \in \mathcal{H}_k$, and $a_1,\ldots,a_k \in X$,
    \end{enumerate}
    then $G \subseteq X$.
\end{definition}

This implies that if $X \subseteq G$ additionally, it must be that $X = G$.


\subsection{Freeness and step recursion}

The problem with attempting ``definition by recursion'' with something like a generating system is that \textit{there might be conflicting witnessing sequences}.


\end{document}
